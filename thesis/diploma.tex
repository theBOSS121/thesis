\documentclass[a4paper, 12pt]{book}
% \documentclass[a4paper, 12pt, draft]{book} % Nalogo preverite tudi z opcijo draft, ki vam bo pokazala, katere vrstice so predolge!



\usepackage[utf8x]{inputenc}   % omogoča uporabo slovenskih črk kodiranih v formatu UTF-8
\usepackage[slovene,english]{babel}    % naloži, med drugim, slovenske delilne vzorce
\usepackage[pdftex]{graphicx}  % omogoča vlaganje slik različnih formatov
\usepackage{fancyhdr}          % poskrbi, na primer, za glave strani
\usepackage{amssymb}           % dodatni simboli
\usepackage{amsmath}           % eqref, npr.
%\usepackage{hyperxmp}
\usepackage[hyphens]{url}  % dodal Solina
\usepackage{comment}       % dodal Solina

\usepackage[pdftex, colorlinks=true,
						citecolor=black, filecolor=black, 
						linkcolor=black, urlcolor=black,
						pagebackref=false, 
						pdfproducer={LaTeX}, pdfcreator={LaTeX}, hidelinks]{hyperref}

\usepackage{color}       % dodal Solina
\usepackage{soul}       % dodal Solina
\usepackage[numbers]{natbib}  % dodal Solina

\usepackage[lined, ruled]{algorithm2e}
\usepackage{longtable}
\usepackage{tikz}
\usepackage{enumitem}
\usepackage{listings}% http://ctan.org/pkg/listings
\lstset{
    basicstyle=\ttfamily,
    mathescape,
    inputencoding = utf8,  % Input encoding
    extendedchars = true,  % Extended ASCII
    literate      =        % Support additional characters
      {Č}{{\v{C}}}1  {č}{{\v{c}}}1  
      {Š}{{\v{S}}}1  {š}{{\v{s}}}1  
      {Ž}{{\v{Z}}}1  {ž}{{\v{z}}}1
}
\allowdisplaybreaks

%%%%%%%%%%%%%%%%%%%%%%%%%%%%%%%%%%%%%%%%
%	DIPLOMA INFO
%%%%%%%%%%%%%%%%%%%%%%%%%%%%%%%%%%%%%%%%
\newcommand{\ttitle}{Inverzije permutacij, permutacijski grafi in tekmovalnostni grafi}
\newcommand{\ttitleEn}{Inversions of permutations, permutation graphs and competitivity graphs}
\newcommand{\tsubject}{\ttitle}
\newcommand{\tsubjectEn}{\ttitleEn}
\newcommand{\tauthor}{Luka Uranič}
\newcommand{\tkeywords}{permutacije, inverzije permutacij, rangiranja, permutacijski grafi, tekmovalnostni grafi}
\newcommand{\tkeywordsEn}{permutations, inversions of permutations, rankings, permutation graphs, competitivity graphs}


%%%%%%%%%%%%%%%%%%%%%%%%%%%%%%%%%%%%%%%%
%	HYPERREF SETUP
%%%%%%%%%%%%%%%%%%%%%%%%%%%%%%%%%%%%%%%%
\hypersetup{pdftitle={\ttitle}}
\hypersetup{pdfsubject=\ttitleEn}
\hypersetup{pdfauthor={\tauthor, lu3748@student.uni-lj.is}}
\hypersetup{pdfkeywords=\tkeywordsEn}


 


%%%%%%%%%%%%%%%%%%%%%%%%%%%%%%%%%%%%%%%%
% postavitev strani
%%%%%%%%%%%%%%%%%%%%%%%%%%%%%%%%%%%%%%%%  

\addtolength{\marginparwidth}{-20pt} % robovi za tisk
\addtolength{\oddsidemargin}{40pt}
\addtolength{\evensidemargin}{-40pt}

\renewcommand{\baselinestretch}{1.3} % ustrezen razmik med vrsticami
\setlength{\headheight}{15pt}        % potreben prostor na vrhu
\renewcommand{\chaptermark}[1]%
{\markboth{\MakeUppercase{\thechapter.\ #1}}{}} \renewcommand{\sectionmark}[1]%
{\markright{\MakeUppercase{\thesection.\ #1}}} \renewcommand{\headrulewidth}{0.5pt} \renewcommand{\footrulewidth}{0pt}
\fancyhf{}
\fancyhead[LE,RO]{\sl \thepage} 
%\fancyhead[LO]{\sl \rightmark} \fancyhead[RE]{\sl \leftmark}
\fancyhead[RE]{\sc \tauthor}              % dodal Solina
\fancyhead[LO]{\sc Diplomska naloga}     % dodal Solina


\newcommand{\BibTeX}{{\sc Bib}\TeX}

%%%%%%%%%%%%%%%%%%%%%%%%%%%%%%%%%%%%%%%%
% naslovi
%%%%%%%%%%%%%%%%%%%%%%%%%%%%%%%%%%%%%%%%  


\newcommand{\autfont}{\Large}
\newcommand{\titfont}{\LARGE\bf}
\newcommand{\clearemptydoublepage}{\newpage{\pagestyle{empty}\cleardoublepage}}
\setcounter{tocdepth}{1}	      % globina kazala

%%%%%%%%%%%%%%%%%%%%%%%%%%%%%%%%%%%%%%%%
% konstrukti
%%%%%%%%%%%%%%%%%%%%%%%%%%%%%%%%%%%%%%%%  
\newtheorem{definicija}{Definicija}[chapter]
\newtheorem{lema}{Lema}[chapter]
\newtheorem{izrek}{Izrek}[chapter]
\newtheorem{trditev}{Trditev}[chapter]
\newtheorem{posledica}{Posledica}[chapter]
\newtheorem{domneva}{Domneva}[chapter]
\newtheorem{primer}{Primer}[chapter]
\newtheorem{opomba}{Opomba}[chapter]
\newenvironment{dokaz}{\emph{Dokaz.}\ }{\hspace{\fill}{$\Box$}}

%%%%%%%%%%%%%%%%%%%%%%%%%%%%%%%%%%%%%%%%%%%%%%%%%%%%%%%%%%%%%%%%%%%%%%%%%%%%%%%
%% PDF-A
%%%%%%%%%%%%%%%%%%%%%%%%%%%%%%%%%%%%%%%%%%%%%%%%%%%%%%%%%%%%%%%%%%%%%%%%%%%%%%%


%%%%%%%%%%%%%%%%%%%%%%%%%%%%%%%%%%%%%%%% 
% define medatata
%%%%%%%%%%%%%%%%%%%%%%%%%%%%%%%%%%%%%%%% 
\def\Title{\ttitle}
\def\Author{\tauthor, lu3748@student.uni-lj.si}
\def\Subject{\ttitleEn}
\def\Keywords{\tkeywordsEn}

%%%%%%%%%%%%%%%%%%%%%%%%%%%%%%%%%%%%%%%% 
% \convertDate converts D:20080419103507+02'00' to 2008-04-19T10:35:07+02:00
%%%%%%%%%%%%%%%%%%%%%%%%%%%%%%%%%%%%%%%% 
\def\convertDate{%
    \getYear
}

{\catcode`\D=12
 \gdef\getYear D:#1#2#3#4{\edef\xYear{#1#2#3#4}\getMonth}
}
\def\getMonth#1#2{\edef\xMonth{#1#2}\getDay}
\def\getDay#1#2{\edef\xDay{#1#2}\getHour}
\def\getHour#1#2{\edef\xHour{#1#2}\getMin}
\def\getMin#1#2{\edef\xMin{#1#2}\getSec}
\def\getSec#1#2{\edef\xSec{#1#2}\getTZh}
\def\getTZh +#1#2{\edef\xTZh{#1#2}\getTZm}
\def\getTZm '#1#2'{%
    \edef\xTZm{#1#2}%
    \edef\convDate{\xYear-\xMonth-\xDay T\xHour:\xMin:\xSec+\xTZh:\xTZm}%
}

\expandafter\convertDate\pdfcreationdate 

%%%%%%%%%%%%%%%%%%%%%%%%%%%%%%%%%%%%%%%%
% get pdftex version string
%%%%%%%%%%%%%%%%%%%%%%%%%%%%%%%%%%%%%%%% 
\newcount\countA
\countA=\pdftexversion
\advance \countA by -100
\def\pdftexVersionStr{pdfTeX-1.\the\countA.\pdftexrevision}


%%%%%%%%%%%%%%%%%%%%%%%%%%%%%%%%%%%%%%%%
% XMP data
%%%%%%%%%%%%%%%%%%%%%%%%%%%%%%%%%%%%%%%%  
\usepackage{xmpincl}
\includexmp{pdfa-1b}

%%%%%%%%%%%%%%%%%%%%%%%%%%%%%%%%%%%%%%%%
% pdfInfo
%%%%%%%%%%%%%%%%%%%%%%%%%%%%%%%%%%%%%%%%  
\pdfinfo{%
    /Title    (\ttitle)
    /Author   (\tauthor, lu3748@student.uni-lj.si)
    /Subject  (\ttitleEn)
    /Keywords (\tkeywordsEn)
    /ModDate  (\pdfcreationdate)
    /Trapped  /False
}


%%%%%%%%%%%%%%%%%%%%%%%%%%%%%%%%%%%%%%%%%%%%%%%%%%%%%%%%%%%%%%%%%%%%%%%%%%%%%%%
%%%%%%%%%%%%%%%%%%%%%%%%%%%%%%%%%%%%%%%%%%%%%%%%%%%%%%%%%%%%%%%%%%%%%%%%%%%%%%%
\let\ab\allowbreak

\begin{document}
\selectlanguage{slovene}
\frontmatter
\setcounter{page}{1} %
\renewcommand{\thepage}{}       % preprecimo težave s številkami strani v kazalu
\newcommand{\sn}[1]{"`#1"'}                    % dodal Solina (slovenski narekovaji)

%%%%%%%%%%%%%%%%%%%%%%%%%%%%%%%%%%%%%%%%
%naslovnica
 \thispagestyle{empty}%
   \begin{center}
    {\large\sc Univerza v Ljubljani\\%
%      Fakulteta za elektrotehniko\\% za študijski program Multimedija
%      Fakulteta za upravo\\% za študijski program Upravna informatika
      Fakulteta za računalništvo in informatiko\\%
      Fakulteta za matematiko in fiziko\\% za študijski program Računalništvo in matematika
     }
    \vskip 10em%
    {\autfont \tauthor\par}%
    {\titfont \ttitle \par}%
    {\vskip 3em \textsc{DIPLOMSKO DELO\\[5mm]         % dodal Solina za ostale študijske programe
%    VISOKOŠOLSKI STROKOVNI ŠTUDIJSKI PROGRAM\\ PRVE STOPNJE\\ RAČUNALNIŠTVO IN INFORMATIKA}\par}%
%    UNIVERZITETNI  ŠTUDIJSKI PROGRAM\\ PRVE STOPNJE\\ RAČUNALNIŠTVO IN INFORMATIKA}\par}%
%    INTERDISCIPLINARNI UNIVERZITETNI\\ ŠTUDIJSKI PROGRAM PRVE STOPNJE\\ MULTIMEDIJA}\par}%
%    INTERDISCIPLINARNI UNIVERZITETNI\\ ŠTUDIJSKI PROGRAM PRVE STOPNJE\\ UPRAVNA INFORMATIKA}\par}%
    INTERDISCIPLINARNI UNIVERZITETNI\\ ŠTUDIJSKI PROGRAM PRVE STOPNJE\\ RAČUNALNIŠTVO IN MATEMATIKA}\par}%
    \vfill\null%
% izberite pravi habilitacijski naziv mentorja!
    {\large \textsc{Mentorica}: izr. prof. dr. Polona Oblak\par}%
%    {\large \textsc{Somentor}:  viš. pred./doc./izr. prof./prof. dr.  Martin Krpan \par}%
    {\vskip 2em \large Ljubljana, 2023 \par}%
\end{center}
% prazna stran
%\clearemptydoublepage      % dodal Solina (izjava o licencah itd. se izpiše na hrbtni strani naslovnice)

%%%%%%%%%%%%%%%%%%%%%%%%%%%%%%%%%%%%%%%%
%copyright stran
\thispagestyle{empty}
\vspace*{8cm}

\noindent
{\sc Copyright}. 
Rezultati diplomske naloge so intelektualna lastnina avtorja in matične fakultete Univerze v Ljubljani.
Za objavo in koriščenje rezultatov diplomske naloge je potrebno pisno privoljenje avtorja, fakultete ter mentorja.

\begin{center}
\mbox{}\vfill
\emph{Besedilo je oblikovano z urejevalnikom besedil \LaTeX.}
\end{center}
% prazna stran
\clearemptydoublepage

%%%%%%%%%%%%%%%%%%%%%%%%%%%%%%%%%%%%%%%%
% stran 3 med uvodnimi listi
\thispagestyle{empty}
\
\vfill

\bigskip
\noindent\textbf{Kandidat:} Luka Uranič\\
\noindent\textbf{Naslov:} \ttitle\\
% vstavite ustrezen naziv študijskega programa!
\noindent\textbf{Vrsta naloge:} Diplomska naloga na univerzitetnem programu prve stopnje Računalništvo in matematika \\
% izberite pravi habilitacijski naziv mentorja!
\noindent\textbf{Mentorica:}  izr. prof. dr. Polona Oblak\\
% \noindent\textbf{Somentor:} isto kot za mentorja

\bigskip
\noindent\textbf{Opis:}\\
V delu preučimo različne kombinatorične interpretacije inverzij permutacij. Za permutacijo predstavimo njen graf inverzij in karakteriziramo permutacijske grafe. Nato si ogledamo njihovo posplošitev, tekmovalnostne grafe, in si ogledamo, kako bi jih lahko uporabili za gručenje tekmovalcev.

\bigskip
\noindent\textbf{Title:} \ttitleEn

\bigskip
\noindent\textbf{Description:}\\
We present different combinatorial interpretations of inversions of permutations. We introduce a permutation graph, in which the edges correspond to the inversions of a given permutation. We characterise permutation graphs. We also present competitivity graphs and see how they could be used to cluster the competitors.

\vfill



\vspace{2cm}

% prazna stran
\clearemptydoublepage

% zahvala
\thispagestyle{empty}\mbox{}\vfill\null\it%
\noindent
Zahvaljujem se mentorici izr. prof. dr. Poloni Oblak za pomoč in usmeritvi pri izdelavi diplomskega dela.\\
\rm\normalfont

% prazna stran
\clearemptydoublepage

%%%%%%%%%%%%%%%%%%%%%%%%%%%%%%%%%%%%%%%%
% posvetilo, če sama zahvala ne zadošča :-)
% \thispagestyle{empty}\mbox{}{\vskip0.20\textheight}\mbox{}\hfill\begin{minipage}{0.55\textwidth}%
% Svoji dragi Alenčici.
% \normalfont\end{minipage}

% % prazna stran
% \clearemptydoublepage


%%%%%%%%%%%%%%%%%%%%%%%%%%%%%%%%%%%%%%%%
% kazalo
\pagestyle{empty}
\def\thepage{}% preprecimo tezave s stevilkami strani v kazalu
\tableofcontents{}


% prazna stran
\clearemptydoublepage

%%%%%%%%%%%%%%%%%%%%%%%%%%%%%%%%%%%%%%%%
% seznam kratic

% \chapter*{Seznam uporabljenih kratic}  % spremenil Solina, da predolge vrstice ne gredo preko desnega roba

% \begin{comment}
% \begin{tabular}{l|l|l}
%   {\bf kratica} & {\bf angleško} & {\bf slovensko} \\ \hline
%   % after \\: \hline or \cline{col1-col2} \cline{col3-col4} ...
%   {\bf CA} & classification accuracy & klasifikacijska točnost \\
%   {\bf DBMS} & database management system & sistem za upravljanje podatkovnih baz \\
%   {\bf SVM} & support vector machine & metoda podpornih vektorjev \\
%   \dots & \dots & \dots \\
% \end{tabular}
% \end{comment}

% \noindent\begin{tabular}{p{0.1\textwidth}|p{.4\textwidth}|p{.4\textwidth}}    % po potrebi razširi prvo kolono tabele na račun drugih dveh!
%   {\bf kratica} & {\bf angleško}                             & {\bf slovensko} \\ \hline
%   {\bf CA}      & classification accuracy               & klasifikacijska točnost \\
%   {\bf DBMS} & database management system & sistem za upravljanje podatkovnih baz \\
%   {\bf SVM}   & support vector machine              & metoda podpornih vektorjev \\
% %  \dots & \dots & \dots \\
% \end{tabular}


% % prazna stran
% \clearemptydoublepage

%%%%%%%%%%%%%%%%%%%%%%%%%%%%%%%%%%%%%%%%
% povzetek
\addcontentsline{toc}{chapter}{Povzetek}
\chapter*{Povzetek}

\noindent\textbf{Naslov:} \ttitle
\bigskip

\noindent\textbf{Avtor:} \tauthor
\bigskip

%\noindent\textbf{Povzetek:} 
\noindent V diplomski nalogi si najprej ogledamo inverzije permutacij, njihove lastnosti, rodovno funkcijo za število permutacij množice $[n]$ z $i$ inverzijami in Bruhatovi delni urejenosti.
Nato pokažemo kako, predstavimo permutacije s permutacijskimi grafi, karakteriziramo permutacijske grafe s pomočjo kohezivnega zaporedja vozlišč, pokažemo, da so gosenice edina drevesa, ki so permutacijski grafi, ter pokažemo, da za vsako gosenico obstajata natanko dve permutaciji, ki generirata permutacijski graf izomorfen tej gosenici. Potem si ogledamo, kaj so tekmovalnostni grafi, množice tekmovalcev, množice posrednih in neposrednih tekmovalcev ter algoritem za izračun množic posrednih in neposrednih tekmovalcev, ki ne potebuje konstrukcije tekmovalnostega grafa. Na koncu uporabimo algoritem na primeru z resničnimi podatki.

% \noindent V vzorcu je predstavljen postopek priprave diplomskega dela z uporabo okolja \LaTeX. Vaš povzetek mora sicer vsebovati približno 100 besed, ta tukaj je odločno prekratek.
% Dober povzetek vključuje: (1) kratek opis obravnavanega problema, (2) kratek opis vašega pristopa za reševanje tega problema in (3) (najbolj uspešen) rezultat ali prispevek magistrske naloge.

\bigskip

\noindent\textbf{Ključne besede:} \tkeywords.
% prazna stran
\clearemptydoublepage

%%%%%%%%%%%%%%%%%%%%%%%%%%%%%%%%%%%%%%%%
% abstract
\selectlanguage{english}
\addcontentsline{toc}{chapter}{Abstract}
\chapter*{Abstract}

\noindent\textbf{Title:} \ttitleEn
\bigskip

\noindent\textbf{Author:} \tauthor
\bigskip

%\noindent\textbf{Abstract:} 
\noindent 
In the thesis, we first examine inversions of permutations, their properties, the generating function for the number of permutations of the set $[n]$ with $i$ inversions, and Bruhat partial orders. Then, we present how permutations can be represented with permutation graphs and characterize permutation graphs using a cohesive vertex-set order. We show that caterpillars are the only trees that are permutation graphs, and prove that for every caterpillar graph, there are exactly two permutations that generate a permutation graph isomorphic to the caterpillar. Next, we explore competitivity graphs, sets of competitors, sets of eventual competitors, and an algorithm to calculate the sets of eventual competitors that does not require the construction of a competitivity graph. Finally, we apply the algorithm to a concrete example featuring real-world data.
\bigskip

\noindent\textbf{Keywords:} \tkeywordsEn.
\selectlanguage{slovene}
% prazna stran
\clearemptydoublepage

%%%%%%%%%%%%%%%%%%%%%%%%%%%%%%%%%%%%%%%%
\mainmatter
\setcounter{page}{1}
\pagestyle{fancy}

\chapter {Uvod}

Inverzija permutacije je par elementov, ki sta v obratnem vrstem redu kot v permutaciji $(1, 2,\dots, n)$. Število inverzij nam meri stopnjo neurejenosti oziroma oddaljenost permutacije od urejenega zaporedja števil $1, 2, \dots, n$. Permutacijski graf permutacije $\pi$ je graf z vozlišči $\{ 1, 2, \dots, n \}$, kjer je $xy$ povezava natanko tedaj, ko je par elementov $(x, y)$ ali $(y, x)$ inverzija permutacije $\pi$. 

Leta 2012 so Gervacio, Rapanut in Ramos \cite{charectarizationPermutationGraphs} karakterizirali permutacijske grafe s pomočjo kohezivnega zaporedja vozlišč. Poleg tega so pokazali, da so gosenice permutacijski grafi, da so gosenice edina drevesa, ki so permutacijski grafi, ter kako lahko na enostaven način konstruiramo permutacijske grafe. Leta 2023 sta Brualdi in Dahl \cite{weakBruhatOrder} pokazala, da obstajata natanko dve permutaciji iz $S_n$, katerih permutacijski graf je izomorfen neki dani gosenici na $n$ vozliščih ($n \geq 3$). Poleg tega sta pokazala tudi, kako so permutacije urejene v šibki Bruhatovi delni urejenosti. Definicija (krepke) Bruhatove delne urejenosti na permutacijah je bila podana v članku \cite{bruhatOrder}. V \cite{generatingFunction} je Margolius prikazal, kakšna je rodovna funkcija za število permutacij dolžine $n$ z danim številom inverzij.

Vsak permutacijski graf je tudi tekmovalnostni. Tekmovalnostni graf je generiran z množico permutacij oziroma rangiranj $R$. Vozlišča tekmovalnostnega grafa so $\{ 1, 2, \dots, n \}$, kjer je $xy$ povezava natanko tedaj, ko je par elementov $(x, y)$ v dveh permutacijah iz $R$ v različnem vrstnem redu. 

Leta 2015 so Criado, García, Pedroche in Romance v \cite{setsOfRankings} predstavili in analizirali nekaj pomembnih množic vozlišč tekmovalnostnega grafa. Na primer, množica tekmovalcev je množica vozlišč, kjer vsako vozlišče iz množice tekmuje z vsemi ostalimi vozlišči iz množice. Množica posrednih in neposrednih tekmovalcev je množica vozlišč, kjer vsako vozlišče iz množice posredno ali neposredno tekmuje z vsemi vozlišči iz množice preko neke poti v grafu, kjer med seboj tekmujeta vsaki sosednji vozlišči na poti. Poleg tega predstavijo algoritem za izračun množic posrednih in neposrednih tekmovalcev, ki ne potrebuje konstrukcije tekmovalnostnega grafa.

Ostale informacije o permutacijah, inverzijah, permutacijskih grafih in faktorskem številskem sistemu so bile pridobljene iz \cite{algorithmDesign, dsSkripta, inversionCoding, factorialNumberSystem, algorithmGraphTheoryAndPerfectGraphs}.

V drugem poglavju bomo predstavili permutacije in inverzije permutacij. Pogledali si bomo njihove lastnosti ter različne urejenosti množice permutacij $S_n$. V tretjem poglavju bomo definirali permutacijske grafe in jih karakterizirali s pomočjo kohezivnega zaporedja vozlišč. Pokazali bomo, da so gosenice edina drevesa, ki so permutacijski grafi. V četrtem poglavju bomo predstavili tekmovalnostne grafe, množice tekmovalcev, množice posrednih in neposrednih tekmovalcev ter algoritem za izračun množice posrednih in neposrednih tekmovalcev, ki ne potebuje konstrukcije tekmovalnostega grafa.

%%%%%%%%%%%%%%%%%%%%%%%%%%%%%%%%%%%%%%%%%
\section{ Splošne oznake in definicije }

V delu bomo uporabili naslednje oznake in definicije.
Graf $G = (V(G), E(G))$ ima množico vozlišč $V(G)$ in množico povezav $E(G)$. Soseščina $N_G(u)$ vozlišča $u$ iz grafa $G$ so vsa vozlišča $v \in V(G)$, za katera velja $uv \in E(G)$. Soseščina je zaprta, če vsebuje tudi vozlišče $u$, kar označimo z oznako $N_G[u]$. Premer grafa je najdaljša pot med dvema vozliščema. Disjunktna unija grafov je združitev dveh grafov v večji graf tako, da naredimo disjunktno unijo množic vozlišč in disjunktno unijo množic povezav. Komplement grafa $G$ označimo z $\overline{G}$ (nepovezave grafa $G$ so povezave grafa $\overline{G}$). $K_n$ je poln graf na $n$ vozliščih, $\overline{K_n}$ je nepovezan graf na $n$ vozliščih, $P_n$ je pot na $n$ vozliščih, $K_{m, n}$ je dvodelen graf z $m$ vozlišči v eni in $n$ vozlišči v drugi množici. Drevo je (neusmerjen) povezan graf brez ciklov. Določiti smer povezave $uv$ grafa $G$ pomeni spremeniti povezavo $uv$ v urejen par $(u, v)$ ali $(v, u)$. Usmerjen graf je graf, ki ima vse povezave usmerjene. Orientacija grafa $G$ je usmerjen graf, ki je pridobljen tako, da vsaki povezavi grafa $G$ določimo smer.

Relacija $R$ na neprazni množici $A$ je množica urejenih parov elementov iz $A$, to pomeni $R\subseteq A\times A$. Oznako $(x, y) \in R$ preberemo kot $x$ je v relaciji $R$ z $y$, kar označimo z $xRy$. Relacija $R$ je:
\begin{itemize}
    \item refleksivna, če $xRx$ za vsak $x\in A$,
    \item irefleksivna, če $\neg xRx$ za vsak $x\in A$,
    \item simetrična, če iz $xRy$ sledi, $yRx$ za vsaka $x,y\in A$,
    \item asimetrična, če iz $xRy$ sledi, $\neg yRx$ za vsaka $x,y\in A$,
    \item antisimetrična, če iz $xRy$ sledi, da je $x = y$ ali $\neg yRx$ za vsaka $x,y\in A$,
    \item tranzitivna, če iz $xRy$ in $yRz$ sledi, da je $xRz$ za vse $x,y,z\in A$,
    \item sovisna, če iz $x \neq y$ sledi, da $xRy$ ali $yRx$ za vsaka $x,y\in A$,
    \item strogo sovisna, če $xRy$ ali $yRx$ za vsaka $x,y\in A$.
\end{itemize}
Za relacijo $R$ rečemo, da je:
\begin{itemize}
    \item delna urejenost, če je $R$ refleksivna, antisimetrična in tranzitivna,
    \item linearna urejenost, če je $R$ antisimetrična, strogosovisna, transitivna,
    \item stroga delna urejenost, če je $R$ asimetrična in tranzitivna,
    \item stroga linearna urejenost, če je $R$ asimetrična, sovisna in transitivna.
\end{itemize}
Če je $R$ linearna urejenost, potem je delna urejenost. Če je $R$ stroga linearna urejenost, potem je stroga delna urejenost.


\begin{definicija}[definicija grupe]
    Naj bo A množica in $\cdot$ operacija, ki vsakemu urejenemu paru elementov iz A priredi natančno določen element iz množice A:
    \[
        \cdot:A \times A \rightarrow A 
    \]
    Par $(A, \cdot)$ je grupa če veljajo naslednje trditve:
    \begin{enumerate}
        \item Za vsake $a, b, c \in A$ velja $(a \cdot b) \cdot c = a \cdot (b \cdot c)$ (asociativnost)
        \item Obstaja tak element $e \in A$, da za vsak $a \in A$ velja $a \cdot e = e \cdot a = a$ (obstoj enote)
        \item Za vsak $a \in A$ obstaja tak element $a^{-1} \in A$, da velja $a \cdot a^{-1} = a^{-1} \cdot a = e$ (obstoj inverza)
        
    \end{enumerate}
\end{definicija}

%%%%%%%%%%%%%%%%%%%%%%%%%%%%%%%%%%%%%%%%%
\chapter{ Permutacije in inverzije }

V tem poglavju bomo definirali permutacije, pogledali kako jih lahko predstavimo in pokazali, da so permutacije množice $[n] = \{ 1, 2, \dots, n\}$ skupaj z operacijo kompozitum grupa. Nato bomo definirali inverzije permutacije, pogledali, kako sta definirani Bruhatovi delni urejenosti, pokazali, kako zapišemo rodovno funkcijo za število permutacij množice $[n]$ z $i$ inverzijami, in si ogledali, kako lahko uredimo množico $S_n$ in tako permutacije identificiramo s celimi števili.

\section{ Permutacije }

Permutacije so prerazporeditve elementov neke končne množice. Elemente te množice lahko oštevilčimo s števili $1, 2, \dots, n$, za nek $n$, zato bomo brez škode za splošnost permutacije gledali na množici $[n]$. 
Permutacije so pomembne v matematiki, računalništvu in na številnih drugih področjih. 



\begin{definicija}
Bijektivni preslikavi $\pi: [n] \rightarrow [n]$ rečemo permutacija. $S_n$ je množica vseh permutacij na množici $[n]$.
\end{definicija}

\subsection{ Zapis permutacij }
Permutacijo $\pi$ lahko zapišemo z vodoravno tabelo:
\[
    \pi = \begin{pmatrix}
        1 & 2 & \cdot\cdot\cdot & n \\
        \pi(1) & \pi(2) & \cdot\cdot\cdot & \pi(n)
    \end{pmatrix}.
\]
Ker ima množica $[n]$ naravno urejenost $1 \leq 2 \leq \cdot\cdot\cdot \leq n$, lahko zgornjo vrstico izpustimo in $\pi$ predstavimo zgolj s spodnjo vrstico:
\[
    \pi = (\pi(1), \pi(2), \cdot\cdot\cdot, \pi(n)) = (\pi_1, \pi_2, \cdot\cdot\cdot, \pi_n),
\]
kjer je $\pi_i = \pi(i)$.
Temu zapisu bomo rekli enovrstični zapis permutacije.
Permutacijo lahko zapišemo tudi s produktom disjunktnih ciklov:
\[
    \pi = (a_1 a_2 \cdot\cdot\cdot a_{i_1})(b_1 b_2 \cdot\cdot\cdot b_{i_2}) \cdot\cdot\cdot (c_1 c_2 \cdot\cdot\cdot c_{i_k}).
\]
Ta zapis nam pove, da je:
\begin{align*}
    \pi(a_1) &= a_2, & \pi(a_2) &= a_3, & \cdot&\cdot\cdot & \pi(a_{i_1-1}) &= a_{i_1}, & \pi(a_{i_1}) &= a_1 \\
    \pi(b_1) &= b_2, & \pi(b_2) &= b_3, & \cdot&\cdot\cdot & \pi(b_{i_2-1}) &= b_{i_2}, & \pi(b_{i_2}) &= b_1 \\
    &&&& \cdot&\cdot\cdot \\
    \pi(c_1) &= c_2, & \pi(c_2) &= c_3, & \cdot&\cdot\cdot & \pi(c_{i_k-1}) &= c_{i_k}, & \pi(c_{i_k}) &= c_1.
\end{align*}

\begin{primer}
    Naj bo $\pi \in S_5$, $\pi(1) = 3$, $\pi(2) = 5$, $\pi(3) = 1$, $\pi(4) = 4$ in $\pi(5) = 2$ (slika \ref{bijektivna_preslikava_n_n}). 
    \begin{figure}[h]
        \begin{center}        
            \begin{tikzpicture}[shorten >=1pt,-]
                \tikzstyle{vertex}=[circle,draw=black!0,fill=white!25,minimum size=20pt,inner sep=0pt]
                \foreach \num/\y in {1/1.1, 2/2.2, 3/3.3, 4/4.4, 5/5.5}
                    \node[vertex] (V-\num) at (0,5-\y) {$\num$};
                \foreach \num/\y in {1/1.1, 2/2.2, 3/3.3, 4/4.4, 5/5.5}
                    \node[vertex] (VV-\num) at (5, 5-\y) {$\num$};
            
                \foreach \from/\to in {1/3, 2/5, 3/1, 4/4, 5/2}
                    \draw[-latex] (V-\from) -- (VV-\to);
            \end{tikzpicture}     
        \end{center}
        \caption{Primer bijektivne preslikave (permutacije) $\pi = (3, 5, 1, 4, 2)$.}
        \label{bijektivna_preslikava_n_n}
    \end{figure}
    Na naslednji način zapišemo permutacijo $\pi$ z vodoravno tabelo in enovrstičnim zapisom:
    \[
        \pi = \begin{pmatrix}
            1 & 2 & 3 & 4 & 5 \\
            3 & 5 & 1 & 4 & 2
        \end{pmatrix} = (3, 5, 1, 4, 2).
    \]
    Če so vsi elementi permutacije manjši od 10, bomo med številkami izpustili vejice. Pri tem moramo vedeti, da to ni zapis permutacije z disjunktnimi cikli. Včasih izpustimo tudi oklepaje:
    \[
        \pi = (3 5 1 4 2) = 35142.
    \]
    Permutacijo $\pi$ zapišemo s produktom disjunktnih ciklov na naslednji način:
    \[
        \pi = (1 3)(2 5)(4).
    \]
    Če vemo koliko elementov ima permutacija, lahko cikle dolžine ena izpustimo: 
    \[
        \pi = (1 3)(2 5).
    \]
    Zapis permutacije $\pi$ kot produkt disjunktnih ciklov ni enoličen, saj lahko na začetek vsakega cikla postavimo poljuben element iz tega cikla, poleg tega pa disjunktni cikli komutirajo:
    \[
        \pi = (3 1)(5 2) = (5 2)(3 1).
    \]
\end{primer}

V nadaljevanju bomo za zapis permutacije uporabljali enovrstični zapis razen, kjer bo navedeno drugače.

\subsection{ Simetrična in permutacijska grupa }

Naj bo permutacija $id \in S_n$ podana s predpisom $id(a) = a$ za vsak $a \in [n]$.

Kompozitum preslikav, ki ga označimo z $\circ$, je operacija na množici preslikav. Kompozitum preslikav $\pi \circ \sigma$ je taka preslikava, ki najprej element preslika z $\sigma$, nato pa dobljeni element preslika še s $\pi$.

\begin{trditev}
    $(S_n, \circ)$ je grupa.
\end{trditev}
\begin{dokaz}
    \begin{enumerate}
        \item Asociativnost: Naj bodo $\pi, \sigma, \tau \in S_n$. Za vsak $i \in [n]$ velja: 
        \[
            ((\pi \circ \sigma) \circ \tau)(i) = (\pi \circ \sigma)(\tau(i)) = \pi(\sigma(\tau(i))),
        \]
        \[
            (\pi \circ (\sigma \circ \tau))(i) = \pi((\sigma \circ \tau)(i)) = \pi(\sigma(\tau(i))).
        \]
        \item Obstoj enote: Za vsaka $\pi \in S_n$ in $i \in [n]$ velja:
        \[
            (\pi \circ id)(i) = \pi(id(i)) = \pi(i),
        \]
        \[
            (id \circ \pi)(i) = id(\pi(i)) = \pi(i).
        \]
        \item Obstoj inverza: Naj bo $\pi \in S_n$. Ker je $\pi$ bijekcija, obstaja $\pi^{-1} \in S_n$:
        \[
            \pi \circ \pi^{-1} = \pi^{-1} \circ \pi = id.
        \]
    \end{enumerate}
    S pomočjo zgornjih lastnosti smo pokazali, da je $(S_n, \circ)$ grupa.
\end{dokaz}

\begin{definicija}
    Grupi $(S_n, \circ)$ rečemo simetrična grupa. Vsaki podgrupi simetrične grupe rečemo permutacijska grupa.
\end{definicija}
Po Cayleyevem izreku \cite{dsSkripta} je vsaka grupa izomorfna neki permutacijski grupi.


%%%%%%%%%%%%%%%%%%%%%%%%%%%%%%%%%%%%%%%%%
\section{ Inverzije permutacij }

V tem podpoglavju bomo najprej definirali inverzijo permutacije, ki je par elementov, ki sta v obratnem vrstem redu kot v identični permutaciji. Število inverzij nam meri stopnjo neurejenosti permutacije oziroma oddaljenost permutacije od identične permutacije. 

\begin{definicija}
\label{definicija_inverzije}
    Inverzija permutacije $\sigma = (a_1, a_2,\dots a_n) \in S_n$ je urejen par $(a_i, a_j)$, kjer je $i < j$ in $a_i > a_j$. Množico vseh inverzij permutacije $\sigma$ označimo z $I_{\sigma}$. Pozicijski zapis inverzije $(a_i, a_j)$ je $(i, j)$.
\end{definicija}

\begin{figure}[h]
    \begin{center}
        \begin{tikzpicture}[shorten >=1pt,-]
            \tikzstyle{vertex}=[circle,draw=black!0,fill=white!25,minimum size=20pt,inner sep=0pt]
            \foreach \num/\y in {1/1, 2/1.8, 3/2.6, 4/3.4}
                \node[vertex] (V-\num) at (0,4-\y) {$\num$};
            \foreach \num/\y in {1/1, 2/1.8, 3/2.6, 4/3.4}
                \node[vertex] (VV-\num) at (4, 4-\y) {$\num$};
        
            
            \node[vertex] (V-0) at (0.3,3.5) {};
            \node[vertex] (VV-0) at (3.7,3.5) {};
            \node[vertex] (V-5) at (0.3,0) {};
            \node[vertex] (VV-5) at (3.7,0) {};

            \foreach \from/\to in {1/4, 2/2, 3/1, 4/3}
                \draw[latex-latex] (V-\from) -- (VV-\to);

            \draw[-latex] (V-0) .. controls (1.25, 4.4) and (2.75, 4.4) .. (VV-0) node[midway, above]{$\sigma$};
            \draw[latex-] (V-5) .. controls (1.25, -0.9) and (2.75, -0.9) .. (VV-5) node[midway, above]{$\sigma^{-1}$};

        \end{tikzpicture}
    \end{center}
    \caption{Permutacija $\sigma = (4, 2, 1, 3)$ in njen inverz $\sigma^{-1} = (3, 2, 4, 1)$.}
    \label{permutacija_4213}
\end{figure}

Naj bo $(a_i, a_j)$ inverzija permutacije $\sigma$. Potem po definiciji \ref{definicija_inverzije} veljata pogoja: 
\[
    \sigma^{-1}(a_i) = i < j = \sigma^{-1}(a_j) \quad \text{in} \quad \sigma(i) = a_i > a_j = \sigma(j).
\]
Torej je $(j, i)$ inverzija permutacije $\sigma^{-1}$.
Še več, če ima permutacija $\sigma$ inverzije $(a_{i_1}, a_{j_1}), \dots, (a_{i_k}, a_{j_k})$, potem ima $\sigma^{-1}$ inverzije $(j_1, i_1), \dots, (j_k, i_k)$. 
Število inverzij permutacije $\sigma$ je enako številu inverzij permutacije $\sigma^{-1}$.

\begin{primer}
    Naj bo $\sigma = (4, 2, 1, 3)$ kot na sliki \ref{permutacija_4213}. Inverzije permutacije $\sigma$ so $(4, 2), (4, 1), (4, 3), (2, 1)$. Pozicijski zapisi inverzij permutacije $\sigma$ so $(1, 2), (1, 3), (1, 4), (2, 3)$. Inverz permutacije $\sigma$ je $\sigma^{-1} = (3, 2, 4, 1)$. Inverzije permutacije $\sigma^{-1}$ so $(3, 2), (3, 1), (2, 1), (4, 1)$. To so ravno obrnjeni pozicijski zapisi inverzij permutacije $\sigma$.
\end{primer}

Identična permutacija $id = (1, 2, \dots, n)$ nima inverzij. Največ inverzij ima permutacija $(n, n-1, \dots, 1)$. V tem primeru je vsak par različnih števil v inverziji. Število izborov dveh elementov izmed $n$ je ravno $\binom{n}{2}$, torej je $|I_{(n, n-1, \dots, 1)}| = \binom{n}{2}$.

Število inverzij je enako številu presečišč v puščičnem diagramu permutacije (slika \ref{permutacija_4213}). To je res, saj vsaka inverzija $(a_i, a_j)$ ustreza presečišču puščic, ki izhajajata iz $i$ in $j$ ter gresta proti $a_i$ in $a_j$, kjer je $a_i > a_j$ in $i < j$.

Standardne primerjalne algoritme razvrščanja, kot je na primer merge sort, lahko prilagodimo tako, da izračunamo število inverzij neke permutacije iz $S_n$ v času $O(n \cdot \log(n))$, \cite{algorithmDesign}.

\section{ Bruhatovi delni urejenosti permutacij}

\begin{definicija}    
    (Krepka) Bruhatova  delna urejenost na množici $S_n$ je tranzitivno refleksivna ovojnica relacije $\preceq_{B}$, ki jo definiramo kot $\sigma \preceq_{B} (ij) \cdot \sigma$, če je $|I_{(ij) \cdot \sigma}| = |I_{\sigma}| + 1$, kjer je $(ij)$ zapis transpozicije elementov na pozicijah $i$ in $j$.
\end{definicija}

\begin{figure}[h]
    \begin{center}        
        \begin{tikzpicture}[shorten >=1pt,-]
            \tikzstyle{vertex}=[circle,draw=black!0,fill=white!25,minimum size=20pt,inner sep=0pt]
            \foreach \num/\x in {1234/5}
                \node[vertex] (V-\num) at (\x, 0) {$\num$};
            \foreach \num/\x in {1243/3, 1324/5, 2134/7}
                \node[vertex] (V-\num) at (\x, 1.75) {$\num$};                
            \foreach \num/\x in {1423/1, 1342/3, 2143/5, 3124/7, 2314/9}
                \node[vertex] (V-\num) at (\x, 3.5) {$\num$};
            \foreach \num/\x in {1432/0, 4123/2, 2413/4, 3142/6, 3214/8, 2341/10}
                \node[vertex] (V-\num) at (\x, 5.25) {$\num$};                
            \foreach \num/\x in {4132/1, 4213/3, 3412/5, 2431/7, 3241/9}
                \node[vertex] (V-\num) at (\x, 7) {$\num$};            
            \foreach \num/\x in {4312/3, 4231/5, 3421/7}
                \node[vertex] (V-\num) at (\x, 8.75) {$\num$};
            \foreach \num/\x in {4321/5}
                \node[vertex] (V-\num) at (\x, 10.5) {$\num$};

            \foreach \from/\to in {1234/1243, 1234/1324, 1234/2134, 1243/1423, 1243/2143, 1324/1342, 1324/3124, 2134/2143, 2134/2314, 1423/1432, 1423/4123, 1342/1432, 1342/3142, 2143/2413, 3124/3142, 3124/3214, 2314/3214, 2314/2341, 1432/4132, 4123/4132, 4123/4213, 2413/4213, 2413/2431, 3142/3412, 3214/3241, 2341/2431, 2341/3241, 4132/4312, 4213/4231, 3412/4312, 3412/3421, 2431/4231, 3241/3421, 4312/4321, 4231/4321, 3421/4321, 1243/1342, 1324/1423, 1324/2314, 2134/3124, 1423/2413, 1342/2341, 2143/4123, 2143/3142, 2143/2341, 3124/4123, 2314/2413, 1432/3412, 1432/2431, 2413/3412, 3142/4132, 3142/3241, 3214/4213, 3214/3412, 4132/4231, 4213/4312, 2431/3421, 3241/4231}
                \draw (V-\from) -- (V-\to);
        \end{tikzpicture}     
    \end{center}
    \caption{Hessejev diagram Bruhatove delne urejenosti množice $S_4$.}
    \label{bruhatova_urejenost_S4}
\end{figure}

\begin{primer}
    Naj bo $\sigma = (1, 2, 4, 3) \in S_4$ in $\pi = (2, 3, 4, 1) \in S_4$ (slika \ref{bruhatova_urejenost_S4}). Permutacija $\sigma$ je manjša od permutacije $\pi$ v (krepki) Bruhatovi delni urejenosti, ker velja:
    \[
        \sigma = (1, 2, 4, 3) \preceq_{B} (2, 1, 4, 3) \preceq_{B} (2, 3, 4, 1) = \pi.
    \]
\end{primer}

\begin{definicija}
\label{definicija_sibke_bruhatove_urejenosti}
Naj bosta $\sigma, \pi \in S_n$. Permutacija $\sigma$ je manjša ali enaka od permutacije $\pi$ v šibki Bruhatovi delni urejenosti, kar označimo z $\sigma \preceq_b \pi$, če je $I_{\sigma} \subseteq I_{\pi}$. 
\end{definicija}

\begin{figure}[h]
    \begin{center}        
        \begin{tikzpicture}[shorten >=1pt,-]
            \tikzstyle{vertex}=[circle,draw=black!0,fill=white!25,minimum size=20pt,inner sep=0pt]
            \foreach \num/\x in {1234/5}
                \node[vertex] (V-\num) at (\x, 0) {$\num$};
            \foreach \num/\x in {1243/3, 1324/5, 2134/7}
                \node[vertex] (V-\num) at (\x, 1.75) {$\num$};                
            \foreach \num/\x in {1423/1, 1342/3, 2143/5, 3124/7, 2314/9}
                \node[vertex] (V-\num) at (\x, 3.5) {$\num$};
            \foreach \num/\x in {1432/0, 4123/2, 2413/4, 3142/6, 3214/8, 2341/10}
                \node[vertex] (V-\num) at (\x, 5.25) {$\num$};                
            \foreach \num/\x in {4132/1, 4213/3, 3412/5, 2431/7, 3241/9}
                \node[vertex] (V-\num) at (\x, 7) {$\num$};            
            \foreach \num/\x in {4312/3, 4231/5, 3421/7}
                \node[vertex] (V-\num) at (\x, 8.75) {$\num$};
            \foreach \num/\x in {4321/5}
                \node[vertex] (V-\num) at (\x, 10.5) {$\num$};

            \foreach \from/\to in {1234/1243, 1234/1324, 1234/2134, 1243/1423, 1243/2143, 1324/1342, 1324/3124, 2134/2143, 2134/2314, 1423/1432, 1423/4123, 1342/1432, 1342/3142, 2143/2413, 3124/3142, 3124/3214, 2314/3214, 2314/2341, 1432/4132, 4123/4132, 4123/4213, 2413/4213, 2413/2431, 3142/3412, 3214/3241, 2341/2431, 2341/3241, 4132/4312, 4213/4231, 3412/4312, 3412/3421, 2431/4231, 3241/3421, 4312/4321, 4231/4321, 3421/4321}
                \draw (V-\from) -- (V-\to);
        \end{tikzpicture}     
    \end{center}
    \caption{Hessejev diagram šibke Bruhatove delne urejenosti množice $S_4$.}
    \label{sibka_bruhatova_urejenost_S4}
\end{figure}

Iz definicije \ref{definicija_sibke_bruhatove_urejenosti} sledi, da lahko permutacijo $\sigma$ pridobimo iz permutacije $\pi$ z zaporedjem transpozicij sosednih elementov, pri čemer vsaka transpozicija zmanjša število inverzij za ena.


\begin{primer}
\label{primer_nepreimenovanje_elementov_mnozice_inverzij}
    Naj bo $\pi = (4, 2, 1, 3)$ in $\sigma = (2, 1, 3, 4)$. Potem sta:
    \[I_{\pi} = \{ (4, 2), (4, 1), (4, 3), (2, 1) \}, \ I_{\sigma} = \{ (2, 1) \}\]
    in zato $\sigma \preceq_b \pi$ (slika \ref{sibka_bruhatova_urejenost_S4}). Permutacijo $\sigma$ pridobimo iz $\pi$ z zaporedjem treh transpozicij sosednjih elementov:
    \[
        (4, 2, 1, 3) \overset{(12)}{\rightarrow} (2, 4, 1, 3) \overset{(23)}{\rightarrow} (2, 1, 4, 3) \overset{(34)}{\rightarrow} (2, 1, 3, 4).
    \]
    Opazimo, da je množica:
    \[
        I_{\pi} \setminus I_{\sigma} = \{ (4, 3), (4, 2), (4, 1) \}
    \]
    ravno množica inverzij $I_{\tau}$ za permutacijo $\tau = (4, 1, 2, 3)$.
\end{primer}

Naslednja lema nam pove, kaj se zgodi z množicami inverzij $I_{\pi}$ in $I_{\sigma}$ permutacij $\pi$ in $\sigma$, kjer eno pridobimo iz druge s poljubno transpozicijo.

\begin{lema}
    Naj bo permutacija $\pi = (i_1, \dots, i_{k-1}, i_k, \dots, i_l, i_{l+1}, \dots, i_n) \in S_n$ in permutacija $\sigma = (i_1, \dots, i_{k-1}, i_l, \dots, i_k, i_{l+1}, \dots, i_n) \in S_n$, pridobljena iz $\pi$ s transpozicijo elementov $i_k$ in $i_l$, kjer je $i_k > i_l$ in $1 \leq k < l \leq n$. Poglejmo si particijo množice $L = \{ k, k+ 1, \dots, l\}$ v množice $L_1, L_2, L_3$ in $\{ k, l \}$, kjer so:
    \[
        L_1 = \{ s \in L: i_s > i_k \}, \ L_2 = \{ s \in L: i_k > i_s > i_l \}, \ L_3 = \{ s \in L: i_s < i_l \}.
    \]
    Množici inverzij $I_{\pi}, I_{\sigma}$ permutacij $\pi, \sigma$ imata razliki:
    \begin{align}
        I_{\pi} \setminus I_{\sigma} &= \{ (i_k, i_l) \} \cup \{ (i_k, i_s): s \in L_2 \cup L_3 \} \cup \{ (i_s, i_l): s \in L_1 \cup L_2 \}, \notag \\
        I_{\sigma} \setminus I_{\pi} &= \{ (i_l, i_s): s \in L_3 \} \cup \{ (i_s, i_k): s \in L_1 \}. \notag
    \end{align}
    Vidimo, da je $\sigma \preceq_b \pi$ ($I_{\sigma} \subseteq I_{\pi}$) natanko tedaj, ko za vsak $s$, kjer je $k < s < l$, velja $i_k > i_s > i_l$ ($I_{\sigma} \setminus I_{\pi} = \emptyset$). Zato je $\sigma \preceq_b \pi$ natanko tedaj, ko lahko pridobimo $\sigma$ iz $\pi$ z zaporedjem transpozicij sosednjih elementov, ki zmanjšajo število inverzij za ena.
\end{lema}
\begin{dokaz}
    Naj bo $1 \leq k < l \leq n$, $\pi = (i_1, \dots, i_{k-1}, i_k, \dots, i_l, i_{l+1}, \dots, i_n) \in S_n$, kjer je $i_k > i_l$, in $\sigma = (i_1, \dots, i_{k-1}, i_l, \dots, i_k, i_{l+1}, \dots, i_n) \in S_n$, pridobljena iz $\pi$ s transpozicijo elementov $i_k$ in $i_l$. Inverzije oblike $(i_a, i_b), (i_{b_1}, i_{b_2})$,  $(i_b, i_c)$ in $(i_a, i_c)$, kjer je $a < k$, $k \leq b \leq l$, $k < b_1 < l$, $k < b_2 < l$ in $l < c$, so v obeh permutacijah. Zato jih v razlikah $I_{\pi} \setminus I_{\sigma}$ in $I_{\sigma} \setminus I_{\pi}$ ni. Razliki sta zato ravno:
    \begin{align}
        I_{\pi} \setminus I_{\sigma} &= \{ (i_k, i_l) \} \cup \{ (i_k, i_s): s \in L_2 \cup L_3 \} \cup \{ (i_s, i_l): s \in L_1 \cup L_2 \}, \notag \\
        I_{\sigma} \setminus I_{\pi} &= \{ (i_l, i_s): s \in L_3 \} \cup \{ (i_s, i_k): s \in L_1 \}. \notag
    \end{align}
    Sledi, da je $\sigma \preceq_b \pi$ ($I_{\sigma} \subseteq I_{\pi}$) natanko tedaj, ko za vsak $s$, kjer je $k < s < l$, velja $i_k > i_s > i_l$ ($I_{\sigma} \setminus I_{\pi} = \emptyset$).
\end{dokaz}

\begin{opomba}
    Če je $\sigma \preceq_b \pi$, potem $I_{\pi} \setminus I_{\sigma}$ ni vedno množica inverzij neke permutacije (glej primera \ref{primer_nepreimenovanje_elementov_mnozice_inverzij} in \ref{primer_preimenovanje_elementov_mnozice_inverzij}). Drži pa, da z ustreznim preimenovanjem elementov dobimo množico inverzij neke permutacije (glej primer \ref{primer_preimenovanje_elementov_mnozice_inverzij}). V splošnem, če je $\sigma \preceq_b \pi$, potem lahko $I_{\pi} \setminus I_{\sigma}$ vedno identificiramo z množico inverzij neke permutacije v smislu, da je vsak interval $[\sigma, \pi]$ v Hessejevem diagramu, kjer je $\pi = \tau \circ \sigma$, v šibki Bruhatovi urejenosti izomorfen nekemu intervalu oblike $[id, \tau]$, kjer je $\tau = \tau \circ id$. To je res, saj je $\tau$ po definiciji šibke Bruhatove urejenosti takšna, da doda nekaj novih inverzij, vendar ohrani vse inverzije permutacije $\sigma$. Če $\tau$ uporabimo na $id$ prav tako pridobimo enako število inverzij, ki so med seboj v enakih razmerjih, kot novo pridobljene inverzije permutacije $\pi$.
\end{opomba}

\begin{primer}
\label{primer_preimenovanje_elementov_mnozice_inverzij}
    Naj bo $\pi = (3, 1, 4, 2)$ in $\sigma = (1, 3, 2, 4)$. Potem sta:
    \[I_{\pi} = \{ (3, 1), (3, 2), (4, 2) \}, \ I_{\sigma} = \{ (3, 2) \}\]
    in zato $\sigma \preceq_b \pi$. Opazimo, da množica:
    \[
        I_{\pi} \setminus I_{\sigma} = \{ (3, 1), (4, 2) \}
    \]
    ni množica inverzij $I_{\tau}$ za nobeno permutacijo $\tau$. Če bi bila, bi $4$ morala biti desno od $3$ in $1$ ter levo od $2$. Iz tega bi sledilo, da je $(3, 2)$ tudi inverzija. Ampak za preimenovanje:
    \[
        1 \rightarrow 1, \ 2 \rightarrow 3, \ 3 \rightarrow 2, \ 4 \rightarrow 4 
    \]
    dobimo množico inverzij $I_{\tau} = \{ (2, 1), (4, 3)\}$ permutacije $\tau = (2, 1, 4, 3)$.
\end{primer}

Kot poseben primer je $id \preceq_b \pi$, za vsak $\pi \in S_n$. Zato je urediti permutacijo (jo preoblikovati v identično permutacijo) s $k$ inverzijami vedno mogoče. To lahko storimo z zaporedjem $k$ transpozicij sosednjih elementov. Na vsakem koraku izberemo transpozicijo $i$ in $i+1$, če je element na poziciji $i+1$ manjši od elementa na poziciji $i$. Na ta način zmanjšamo število inverzij za $1$. To ponavljamo, dokler ne pridemo do identične permutacije.

\begin{primer}
    Postopek ureditve permutacije $\sigma = (4, 2, 1, 3)$, ki ima $4$ inverzije: 
    \[
        (4, 2, 1, 3) \overset{(12)}{\rightarrow} (2, 4, 1, 3) \overset{(23)}{\rightarrow} (2, 1, 4, 3) \overset{(34)}{\rightarrow} (2, 1, 3, 4) \overset{(12)}{\rightarrow} (1, 2, 3, 4).
    \]
\end{primer}

\section{ Rodovne funkcije permutacij }
Naj bo $f_n(x)$ rodovna funkcija s koeficienti $a_i$ pred $x^i$, ki štejejo število permutacij množice $[n]$ z $i$ inverzijami:
\[
    f_n(x) = \sum_{i=0}^{\binom{n}{2}} a_i x^i.
\]
Število vseh permutacij množice $[n]$ je $n!$, zato velja:
\[
    \sum_{i=0}^{\binom{n}{2}} a_i = n! .
\]
Sedaj si poglejmo, kako rekurzivno konstruiramo rodovno funkcijo $f_n(x)$. Začnimo z rodovno funkcijo $f_1$. Edina permutacija iz $S_1$ je $(1)$. Ta permutacija nima nobene inverzije. Tako dobimo rodovno funkcijo:
\[
    f_1(x) = 1.
\]
Sedaj iz permutacije iz $S_1$ naredimo permutacijo iz $S_2$ tako, da vstavimo dvojko na prvo ali drugo mesto. Vidimo da, če jo vstavimo na prvo mesto, dobimo permutacijo $(2, 1)$, ki ima eno inverzijo. V drugem primeru pa dobimo permutacijo $(1, 2)$, ki nima inverzij. Tako dobimo rodovno funkcijo:
\[
    f_2(x) = 1 + x = f_1(x) \cdot (1 + x).
\]
Sedaj iz permutacije iz $S_2$ na podoben način naredimo permutacijo iz $S_3$. Imamo dve različni permutaciji iz $S_2$. Permutacija $(1, 2)$ je brez inverzij. Ko vstavimo trojko na poljubno mesto, ustvarimo permutacijo z dvema, eno ali nič inverzijami. Druga permutacija je $(2, 1)$ z eno inverzijo. Ko vstavimo trojko na poljubno mesto, ustvarimo permutacijo s tremi, dvema ali eno inverzijo, torej zopet dve, eno ali nič novih inverzij. Tako dobimo rodovno funkcijo:
\[
    f_3(x) = 1 + 2x + 2x^2 + x^3 = 1 \cdot (1 + x + x^2) + x \cdot (1 + x + x^2) = f_2(x) \cdot (1 + x + x^2).
\]
Vidimo da, ko v permutacijo iz $S_{n-1}$ vstavimo element $n$, lahko naredimo med $0$ in $n-1$ novih inverzij ($1 + x + \cdot\cdot\cdot + x^{n-1}$) odvisno od tega, kam vstavimo element $n$. Prav tako vse inverzije, ki so bile del permutacije iz $S_{n-1}$, ostanejo. Tako iz $a_i$ permutacij iz $S_{n-1}$ z $i$ inverzijami dobimo $a_i$ permutacij iz $S_n$ z $i$ inverzijami (vstavimo $n$ na zadnje mesto), $a_i$ permutacij iz $S_n$ z $i+1$ inverzijami (vstavimo $n$ na predzadnje mesto), $\dots$, $a_i$ permutacij iz $S_n$ z $i+n-1$ inverzijami (vstavimo $n$ na prvo mesto). Se pravi iz člena $a_i x^i$ v rodovni funkciji $f_{n-1}$ dobimo člene $a_i x^i \cdot (1 + x + \cdot\cdot\cdot + x^{n-1})$ v rodovni funkciji $f_n$.
Zato, velja rekurzivna zveza:
\[
    f_n(x) = f_{n-1}(x) \cdot (1 + x + \cdot\cdot\cdot + x^{n-1}).
\]
In tako dobimo eksplicitno formulo za rodovno funkcijo:
\[
    f_n(x) = \prod_{m=1}^{n}\sum_{i=0}^{m-1} x^i = 1 (1 + x) (1 + x + x^2) \cdot\cdot\cdot (1 + x + \cdot\cdot\cdot + x^{n-1}).
\]
Naslednja formula iz \cite{generatingFunction} nam pove, kako iz rodovne funkcije $f_n$ izrazimo koeficient $a_i$:
\[
    a_i = \binom{n+i-1}{i} + \sum_{j=1}^{\infty} (-1)^j \left( \binom{n+i-u_j-j-1}{i-u_j-j} + \binom{n+i-u_j-1}{i-u_j} \right),
\]
kjer so $u_j = \frac{j(3j-1)}{2}$ petkotniška števila. Če je v binomskem simbolu spodaj negativno število, je vrednost binomskega simbola enaka $0$. Zato je vsota končna.

\section{ Lehmerjeva koda in vektor inverzij}
Množico $S_n$ lahko uredimo na različne načine. Zato lahko vsaki permutaciji iz množice $S_n$ dodelimo celo število $N$, kjer je $0 \leq N \leq n!$. To je ravno njena zaporedna številka v neki ureditvi. V tem podpoglavju si bomo pogledali ureditvi s pomočjo Lehmerjeve kode in vektorja inverzij. Lehmerjeva koda in vektor inverzij sta zapisa števil v faktorskem številskem sistemu. Uporabimo ju kot vmesni korak med pretvarjanjem števila v permutacijo in obratno (slika \ref{pretvarjanje_permutacija_stevilo}).
\begin{figure}[h]
    \begin{center}        
        \begin{tikzpicture}[shorten >=1pt,-]
            \tikzstyle{vertex}=[rectangle,draw=black!0,fill=white!25,minimum size=20pt,inner sep=0pt]
            
            \node[vertex] (V-1) at (0, 1.5) {Permutacija};
            \node[vertex] (V-3) at (5, 1.5) {Faktorski številski sistem};
            \node[vertex] (V-2a) at (5, 0) {Lehmerjeva koda};
            \node[vertex] (V-2b) at (5, 3) {Vektor inverzij};
            \node[vertex] (V-4) at (10, 1.5) {Število};
        
            \foreach \from/\to in {1/2a, 2a/3, 1/2b, 2b/3, 3/4}
                \draw[latex-latex] (V-\from) -- (V-\to);

        \end{tikzpicture}     
    \end{center}
    \caption{Pretvorba med permutacijo in številom z vmesnim korakom.}
    \label{pretvarjanje_permutacija_stevilo}
\end{figure}

\subsection{ Faktorski številski sistem }

Faktorski številski sistem je številski sistem, kjer teže pozicije števk niso geometrijska vrsta nekega števila, temveč so fakultete naravnih števil (primer \ref{primer_faktorski_stevilski_sistem1}). Naj ima število v faktorskem številskem sistemu zapis $d_nd_{n-1}...d_2d_1$, potem ima števka $d_i$ na poziciji $i$ težo $(i-1)!$. Števka $d_i$ je nenegativno celo število manjše od $i$ (pri tem lahko izpustimo $d_1$, saj je $d_1$ vedno $0$). 

Število zapisano v faktorskem številskem sistemu pretvorimo v desetiški številski sistem tako, da seštejemo produkt vseh števk s pripadajočo težo pozicije števke (primer \ref{primer_faktorski_stevilski_sistem1}).

\begin{primer}
\label{primer_faktorski_stevilski_sistem1}
    Vzemimo za primer število $341010_{!}$ v faktorskem številskem sistemu. Ker je:
    \[
        341010_{!} = 3 \cdot 5! + 4 \cdot 4! + 1 \cdot 3! + 0 \cdot 2! + 1 \cdot 1! + 0 \cdot 0! = 463_{10} 
    \]
    je $341010_{!} = 463_{10}$
\end{primer}

Naj bo $x$ pozitivno celo število zapisano v desetiškem številskem sistemu. Število $x$ bi radi zapisali v faktorskem številskem sistemu kot
\[
    x = \sum_{i=1}^{n} d_i \cdot (i-1)!
\]
Če $x$ delimo z 1, potem je 
\[
    x = r_1 + 1 \cdot x^{(1)},
\]
kjer je $x = x^{(1)}$ in $r_1 = 0 < 1$. Potem lahko rekurzivno delimo $x^{(1)}$ z 2 in dobimo
\[
    x^{(1)} = r_2 + 2 \cdot x^{(2)}
\]
in zato
\[
    x = r_1 + 1 \cdot (r_2 + 2 \cdot x^{(2)}),
\]
kjer je $x^{(2)} < x^{(1)}$ in $r_2 < 2$. Na ta način nadaljujemo rekurzijo, ki ima končno število korakov in dobimo:
\[
    x = r_1 + 1 \cdot (r_2 + 2 \cdot(r_3 + 3 \cdot(\cdot\cdot\cdot + (n-1) \cdot r_n))),
\]
kjer je $0 \leq r_i < i$ za $i = 1, 2, \dots, n$. Se pravi $d_i = r_i$. Tako dobimo faktorski zapis števila $x$.

Pretvorbo iz desetiškega v faktorski številski sistem torej naredimo tako, da število zaporedoma delimo s števili $1, 2, 3, \dots$ in si zapisujemo ostanke pri deljenju, dokler ne dobimo $0$ kot rezultat deljenja. Zapis števila so ostanki pri deljenju v vrstem redu od zadnjega deljenja proti prvemu (primer \ref{primer_faktorski_stevilski_sistem2}). 

\begin{primer}
\label{primer_faktorski_stevilski_sistem2}
    Vzemimo za primer število $463$ v desetiškem številskem sistemu. Ker je:
    \begin{align}
        463 / 1 = 463,&\ ostanek = 0, \quad 463 = 0 \cdot 0! + 463 \cdot 1! \notag\\
        463 / 2 = 231,&\ ostanek = 1, \quad 463 = 1 \cdot 1! + 231 \cdot 2! \notag\\
        231 / 3 = 77,&\ ostanek = 0, \quad 463 = 1 \cdot 1! + 0 \cdot 2! + 77 \cdot 3!  \notag\\
        77 / 4 = 19,&\ ostanek = 1, \quad 463 = 1 \cdot 1! +  1 \cdot 3! + 19 \cdot 4! \notag\\
        19 / 5 = 3,&\ ostanek = 4, \quad 463 = 1 \cdot 1! + 1 \cdot 3! + 4 \cdot 4! + 3 \cdot 5! \notag\\
        3 / 6 = 0,&\ ostanek = 3, \quad 463 = 1 \cdot 1! + 1 \cdot 3! + 4 \cdot 4! + 3 \cdot 5! + 0 \cdot 6! \notag
    \end{align}
    je $463_{10} = 341010_{!}$
\end{primer}


\subsection{ Pretvorba med Lehmerjevo kodo ali vektorjem inverzij in številom }
Lehmerjeva koda in vektor inverzij sta zapisa števil v faktorskem številskem sistemu, zato je pretvorba med Lehmerjevo kodo ali vektorjem inverzij in številom ravno pretvorba med faktorskim in desetiškim številskim sistemom.

\subsection{ Pretvorba med Lehmerjevo kodo ali vektorjem inverzij in permutacijo }
Poglejmo si najprej pretvorbi permutacije v Lehmerjevo kodo in vektor inverzij. Naj bo $\sigma = (\sigma_1, \sigma_2, \dots, \sigma_n) \in S_n$ in $d_nd_{n-1}...d_2d_1$ zapis števila v faktorskem številskem sistemu (Lehmerjeva koda ali vektor inverzij), ki pripada permutaciji $\sigma$.

V Lehmerjevi kodi permutacije $\sigma$ števka $d_n$ predstavlja $\sigma_1 - 1$. To je število elementov manjših od $\sigma_1$, ki so v inverziji s $\sigma_1$. Števka $d_{n-1}$ predstavlja število elementov, ki so manjši od $\sigma_2$ in so v inverziji s $\sigma_2$. V splošnem, števka $d_{n-i+1}$ predstavlja število elementov, ki so manjši od $\sigma_i$ in so v inverziji s $\sigma_i$.

Vektor inverzij permutacije $\sigma$ je podoben zapis. Števka $d_{n-j+1}$ nam pove, koliko je inverzij oblike $(i, j)$, kjer je $j$ manjša vrednost para števil v inverziji.

Obe kodiranji lahko prikažemo z Rothejevim diagramom, kjer so pike postavljene na pozicijah $(i, \sigma_i)$, križi pa predstavljajo inverzije permutacije. Lehmerjeva koda nam šteje število križev v vsaki vrstici, vektor inverzij pa nam šteje število križev v vsakem stolpcu. Ker ima inverzna permutacija ravno transponiran Rothejev diagram, sledi, da je vektor inverzij ravno Lehmerjeva koda inverzne permutacije in Lehmerjeva koda je ravno vektor inverzij inverzne permutacije. Primer Rothejevega diagrama je prikazan v tabeli \ref{tbl:rothejev_diagram}.

\begin{table}[h]
    \begin{center}
        \begin{tabular}{ |c|c|c|c|c|c|c|c|c|c|c| } 
        \hline
            $i \setminus \sigma_i$ & 1 & 2 & 3 & 4 & 5 & 6 & 7 & 8 & 9 & Lehmerjeva koda  \\ 
        \hline
            1 & $\times$ & $\times$ & $\times$ & $\times$ & $\times$ & $\cdot$ & & & & $d_9 = 5$  \\ 
        \hline
            2 & $\times$ & $\times$ & $\cdot$ & & & & & & & $d_8 = 2$  \\ 
        \hline
            3 & $\times$ & $\times$ & & $\times$ & $\times$ & & $\times$ & $\cdot$ & & $d_7 = 5$  \\ 
        \hline
            4 & $\cdot$ & & & & & & & & & $d_6 = 0$  \\ 
        \hline
            5 & & $\times$ & & $\cdot$ & & & & & & $d_5 = 1$  \\ 
        \hline
            6 & & $\times$ & & & $\times$ & & $\times$ & & $\cdot$ & $d_4 = 3$  \\ 
        \hline
            7 & & $\times$ & & & $\times$ & & $\cdot$ & & & $d_3 = 2$  \\ 
        \hline
            8 & & $\cdot$ & & & & & & & & $d_2 = 0$  \\ 
        \hline
            9 & & & & & $\cdot$ & & & & & $d_1 = 0$  \\ 
        \hline
            Vektor inverzij & 3 & 6 & 1 & 2 & 4 & 0 & 2 & 0 & 0 &  \\ 
        \hline
        \end{tabular}
    \end{center}
    \caption{ Rothejev diagram za permutacijo $\sigma = (6, 3, 8, 1, 4, 9, 7, 2, 5)$. }
    \label{tbl:rothejev_diagram}
\end{table}

Sedaj si poglejmo še pretvorbi Lehmerjeve kode in vektorja inverzij v permutacijo (primeri za permutacije iz $S_4$ so v tabeli \ref{tbl:permutacije4vektorji}).

Da bi pretvorili Lehmerjevo kodo $d_nd_{n-1}...d_1$ v permutacijo, najprej uredimo števila $1, 2, \dots, n$ v vrsto. $\sigma_1$ je enak elementu v vrsti, ki je za $d_{n}$ elementi. Nato ta element izbrišemo iz vrste. $\sigma_2$ je enak elementu v spremenjeni vrsti, ki je za $d_{n-1}$ elementi. Nato ta element izbrišemo iz vrste in ponovimo postopek za $\sigma_3, \dots, \sigma_n$ (primer \ref{primer_lehmerjeva_koda_permutacija}). Ta postopek je inverzen prej opisanemu, saj ko izberemo element za $\sigma_i$, bo ta element vedno imel natanko $d_{n-i+1}$ elementov, ki so manjši od $\sigma_i$ in so v inverziji s $\sigma_i$.


\begin{primer}
\label{primer_lehmerjeva_koda_permutacija}
    Vzemimo za primer Lehmerjevo kodo 525013200, kot v primeru iz tabele \ref{tbl:rothejev_diagram}. Ker je
    \begin{align}
        &d_9 = 5, \quad [ \ 1, \ 2, \ 3, \ 4, \ 5, \ 6, \ 7, \ 8, \ 9 \ ]  \ &\Rightarrow& \ &\sigma_1 = 6, \notag \\
        &d_8 = 2, \quad [ \ 1, \ 2, \ 3, \ 4, \ 5, \ 7, \ 8, \ 9 \ ] \ &\Rightarrow& \ &\sigma_2 = 3, \notag \\
        &d_7 = 5, \quad [ \ 1, \ 2, \ 4, \ 5, \ 7, \ 8, \ 9 \ ] \ &\Rightarrow& \ &\sigma_3 = 8, \notag \\
        &d_6 = 0, \quad [ \ 1, \ 2, \ 4, \ 5, \ 7, \ 9 \ ] \ &\Rightarrow& \ &\sigma_4 = 1, \notag \\
        &d_5 = 1, \quad [ \ 2, \ 4, \ 5, \ 7, \ 9 \ ] \ &\Rightarrow& \ &\sigma_5 = 4, \notag \\
        &d_4 = 3, \quad [ \ 2, \ 5, \ 7, \ 9 \ ] \ &\Rightarrow& \ &\sigma_6 = 9, \notag \\
        &d_3 = 2, \quad [ \ 2, \ 5, \ 7 \ ] \ &\Rightarrow& \ &\sigma_7 = 7, \notag \\
        &d_2 = 0, \quad [ \ 2, \ 5 \ ] \ &\Rightarrow& \ &\sigma_8 = 2, \notag \\
        &d_1 = 0, \quad [ \ 5 \ ] \ &\Rightarrow& \ &\sigma_9 = 5, \notag
    \end{align}
    je $\sigma = (6, 3, 8, 1, 4, 9, 7, 2, 5)$ permutacija Lehmerjeve kode 525013200.
\end{primer}

Da bi pretvorili tabelo inverzij $d_nd_{n-1}...d_1$ v permutacijo, imejmo najprej prazno vrsto. Najprej vzemimo $n$ in ga vstavimo v vrsto za $d_1$ elementi (vedno $0$). Nato vzamemo $n-1$ in ga vstavimo v vrsto za $d_{2}$ elementi, $\dots$, vzamemo $1$ in ga vstavimo v vrsto za $d_{n}$ elementi (primer \ref{primer_vektor_inverzij_permutacija}). Ta postopek je inverzen prej opisanemu, saj ko vstavimo element $j$ za $d_{n-j+1}$ elementi, bo $j$ vedno v inverziji oblike $(i, j)$ z natanko $d_{n-j+1}$ elementi, kjer je $j$ manjša vrednost para števil v inverziji.

\begin{primer}
\label{primer_vektor_inverzij_permutacija}
    Vzemimo za primer vektor inverzij 361240200, kot v primeru iz tabele \ref{tbl:rothejev_diagram}. Ker je
    \begin{align}
        &d_1 = 0& \ &\Rightarrow& \ &[ \ 9 \ ], \notag \\
        &d_2 = 0& \ &\Rightarrow& \ &[ \ 8, \ 9 \ ], \notag \\
        &d_3 = 2& \ &\Rightarrow& \ &[ \ 8, \ 9, \ 7 \ ], \notag \\
        &d_4 = 0& \ &\Rightarrow& \ &[ \ 6, \ 8, \ 9, \ 7 \ ], \notag \\
        &d_5 = 4& \ &\Rightarrow& \ &[ \ 6, \ 8, \ 9, \ 7, \ 5 \ ], \notag \\
        &d_6 = 2& \ &\Rightarrow& \ &[ \ 6, \ 8, \ 4, \ 9, \ 7, \ 5 \ ], \notag \\
        &d_7 = 1& \ &\Rightarrow& \ &[ \ 6, \ 3, \ 8, \ 4, \ 9, \ 7, \ 5 \ ], \notag \\
        &d_8 = 6& \ &\Rightarrow& \ &[ \ 6, \ 3, \ 8, \ 4, \ 9, \ 7, \ 2, \ 5 \ ], \notag \\
        &d_9 = 3& \ &\Rightarrow& \ &[ \ 6, \ 3, \ 8, \ 1, \ 4, \ 9, \ 7, \ 2, \ 5 \ ], \notag
    \end{align}
    je $\sigma = (6, 3, 8, 1, 4, 9, 7, 2, 5)$ permutacija vektorja inverzij 361240200.
\end{primer}

Vsota števk v Lehmerjevi kodi ali vektorju inverzij nam pove število inverzij permutacije, saj vsak križ v Rothejevem diagramu predstavlja ravno eno inverzijo in vsota števk Lehmerjeve kode ali vekorja inverzij je ravno število vseh križev v Rothejevem diagramu. 

%Parnost vsote pa nam pove znak permutacije.
\begin{table}[h]
    \begin{center}
        \begin{tabular}{ |c|c|c|c| } 
        \hline
            $\sigma$ & Lehmerjeva koda & Vektor inverzij & Število inverzij \\ 
        \hline
            1234 & 0000 & 0000 & 0 \\ 
        \hline
            1243 & 0010 & 0010 & 1 \\ 
        \hline
            1324 & 0100 & 0100 & 1 \\ 
        \hline
            1342 & 0110 & 0200 & 2 \\ 
        \hline
            1423 & 0200 & 0110 & 2 \\ 
        \hline
            1432 & 0210 & 0210 & 3 \\ 
        \hline
            2134 & 1000 & 1000 & 1 \\ 
        \hline
            2143 & 1010 & 1010 & 2 \\ 
        \hline
            2314 & 1100 & 2000 & 2 \\ 
        \hline
            2341 & 1110 & 3000 & 3 \\ 
        \hline
            2413 & 1200 & 2010 & 3 \\ 
        \hline
            2431 & 1210 & 3010 & 4 \\ 
        \hline
            3124 & 2000 & 1100 & 2 \\ 
        \hline
            3142 & 2010 & 1200 & 3 \\ 
        \hline
            3214 & 2100 & 2100 & 3 \\ 
        \hline
            3241 & 2110 & 3100 & 4 \\ 
        \hline
            3412 & 2200 & 2200 & 4 \\ 
        \hline
            3421 & 2210 & 3200 & 5 \\ 
        \hline
            4123 & 3000 & 1110 & 3 \\ 
        \hline
            4132 & 3010 & 1210 & 4 \\ 
        \hline
            4213 & 3100 & 2110 & 4 \\ 
        \hline
            4231 & 3110 & 3110 & 5 \\
        \hline
            4312 & 3200 & 2210 & 5 \\ 
        \hline
            4321 & 3210 & 3210 & 6 \\ 
        \hline
        \end{tabular}
    \end{center}
    \caption{ Permutacije iz $S_4$ s pripadajočimi Lehmerjevimi kodami in vektorji inverzij. }
    \label{tbl:permutacije4vektorji}
\end{table}

%%%%%%%%%%%%%%%%%%%%%%%%%%%%%%%%%%%%%%%%%
\chapter{ Permutacijski grafi }

V tem poglavju bomo definirali permutacijske grafe in si ogledali njihovo karakterizacijo s kohezivnim zaporedjem grafa. Nato bomo pokazali, da so gosenice edina drevesa, ki so permutacijski grafi, in si ogledali, koliko je permutacij, katerih permutacijski graf je izomorfen neki poti ali gosenici. Na koncu pa bomo pokazali še, kako lahko konstruiramo permutacijske grafe.

\section{ Karakterizacija permutacijskih grafov }

\begin{definicija}
    Naj bo $\sigma \in S_n$. Graf inverzij permutacije $\sigma$, ki ga označimo z $G_{\sigma}$, je neusmerjen graf z $V(G_{\sigma}) = [n]$, kjer je $xy \in E(G_{\sigma})$ natanko tedaj, ko je $(x, y)$ ali $(y, x)$ inverzija permutacije $\sigma$. Vsak graf izomorfen grafu $G_{\sigma}$ za neko permutacijo $\sigma$ imenujemo permutacijski graf.
\end{definicija}

\begin{primer}
    Naj bo $\sigma = (4, 2, 5, 1, 7, 6, 3) \in S_7$ permutacija in $V(G_{\sigma}) =[7]$ množica vozlišč grafa inverzij permutacije $\sigma$. Množica inverzij permutacije $\sigma$ je
    $I_{\sigma} = \{ (4, 2), (4, 1), (4, 3), (2, 1), (5, 1), (5, 3), (7, 6), (7, 3), (6, 3) \}$,
    zato je $E(G_{\sigma}) = \{ 42, 41, 43, 21, 51, 53, 76, 73, 63 \}$  množica povezav grafa inverzij permutacije $\sigma$. Graf $G_{\sigma}$ je prikazan na sliki \ref{graf_inverzij}.
\end{primer}

\begin{figure}[h]
    \begin{center}        
        \begin{tikzpicture}[shorten >=1pt,-]
            \tikzstyle{vertex}=[circle,draw=black,fill=white!25,minimum size=20pt,inner sep=0pt]
            \foreach \num/\x in {4/1, 2/3, 5/5, 1/7, 7/9, 6/11, 3/13}
                    \node[vertex] (V-\num) at (\x,0) {$\num$};

                \foreach \from/\to in {4/2, 5/1, 7/6, 6/3}
                    \draw (V-\from) -- (V-\to);

                \draw (V-2) .. controls (3.5, 0.75) and (6.5, 0.75) .. (V-1);
                \draw (V-7) .. controls (9.5, 0.75) and (12.5, 0.75) .. (V-3);
                \draw (V-4) .. controls (1.5, 1.5) and (6.5, 1.5) .. (V-1);
                \draw (V-5) .. controls (5.5, 1.5) and (12.5, 1.5) .. (V-3);
                \draw (V-4) .. controls (1, 2.5) and (13, 2.5) .. (V-3);
        \end{tikzpicture}        
    \end{center}
    \caption{Primer grafa inverzij permutacije $\sigma = (4, 2, 5, 1, 7, 6, 3)$.}
    \label{graf_inverzij}
\end{figure}

Če je graf permutacijski graf, potem lahko veliko problemov, ki so na poljubnih grafih NP-polni, rešimo v polinomskem času. Na primer iskanje največjega podgrafa, ki je poln graf, je ekvivalentno iskanju največjega padajočega zaporedja v permutaciji, ki definira permutacijski graf \cite{algorithmGraphTheoryAndPerfectGraphs}.

\begin{definicija}[Kohezivno zaporedje grafa]
\label{def_kohezivno_zaporedje}
    Naj bo $G$ neusmerjen graf na $n$ vozliščih. 
    Zaporedju vozlišč $l = (v_1, v_2, \dots, v_n)$ rečemo kohezivno vozliščno zaporedje grafa $G$ (ali enostavneje kohezivno zaporedje grafa $G$), če sta za poljubne $i, j, k$, kjer je $1 \leq i < k < j \leq n$, izpolnjena naslednja pogoja (slika \ref{graf_kohezivno_zaporedje_ab}):
    \begin{enumerate}[label=(\alph*)]
        \item Če je $v_iv_k \in E(G)$, $v_kv_j \in E(G)$, potem je $v_iv_j \in E(G)$.
        \item Če je $v_iv_j \in E(G)$, potem je $v_iv_k \in E(G)$ ali $v_kv_j \in E(G)$.
    \end{enumerate}
\end{definicija}

\begin{figure}[h]
    \begin{center}        
        \begin{tikzpicture}[shorten >=1pt,-]
            \tikzstyle{vertex}=[circle,draw=black,fill=white!25,minimum size=20pt,inner sep=0pt]
            \foreach \num/\x in {i/1, k/3, j/5}
                    \node[vertex] (V-\num) at (\x,0) {$v_{\num}$};
            \foreach \num/\x in {i/7, k/9, j/11}
                    \node[vertex] (VV-\num) at (\x,0) {$v_{\num}$};

                \foreach \from/\to in {i/k, k/j}
                    \draw (V-\from) -- (V-\to);
                \foreach \from/\to in {i/k, k/j}
                    \draw[dashed] (VV-\from) -- (VV-\to);


                \draw[dashed] (V-i) .. controls (1.5, 1) and (4.5, 1) .. (V-j);
                \draw (VV-i) .. controls (7.5, 1) and (10.5, 1) .. (VV-j);
        \end{tikzpicture}        
    \end{center}
    \caption{Pogoja za kohezivno zaporedje grafa $G$.}
    \label{graf_kohezivno_zaporedje_ab}
\end{figure}

\begin{lema}
\label{lema0}
    Naj bo $G$ graf. Zaporedje vozlišč $l$ je kohezivno zaporedje grafa $G$ natanko tedaj, ko je $l$ kohezivno zaporedje grafa $\overline{G}$. 
\end{lema}
\begin{dokaz}
    $(\Rightarrow)$ Naj bo $l = (v_1, v_2, \dots, v_n)$ kohezivno zaporedje grafa $G$. Trdimo, da je $l$ kohezivno zaporedje grafa $\overline{G}$. 
    \begin{enumerate}[label=(\alph*)]
        \item Naj bosta $v_iv_k, v_kv_j \in E(\overline{G})$ taki povezavi, da $i < k < j$. Potem, po definiciji komplementa $v_iv_k, v_kv_j \notin E(G)$. Če pogoj (b) iz definicije \ref{def_kohezivno_zaporedje} negiramo ($v_iv_k, v_kv_j \notin E(G) \Rightarrow v_iv_j \notin E(G)$), sledi, da $v_iv_j \notin E(G)$. Kar pomeni $v_iv_j \in E(\overline{G})$.
        \item Naj bo $v_iv_j \in E(\overline{G})$ taka povezava, da $i < j$ in $k$ tako naravno število, da je $i < k < j$. Potem $v_iv_j \notin E(G)$. Če pogoj (a) iz definicije \ref{def_kohezivno_zaporedje} negiramo, vidimo, da $v_iv_k \notin E(G)$ ali $v_kv_j \notin E(G)$. Zato sledi, da je $v_iv_k \in E(\overline{G})$ ali $v_kv_j \in E(\overline{G})$. 
    \end{enumerate}

    $(\Leftarrow)$ Obratna smer dokaza sledi iz dejstva, da je $\overline{\overline{G}} = G$.
\end{dokaz}

\begin{izrek}
\label{izrek_sigma_kohezivno_zaporedje}
    Naj bo $\sigma \in S_n$. Zaporedje vozlišč $(\sigma(1), \sigma(2), \dots, \sigma(n))$ je kohezivno zaporedje permutacijskega grafa $G_{\sigma}$.
\end{izrek}
\begin{dokaz}
    Naj bo $\sigma = (\sigma(1), \sigma(2), \dots, \sigma(n)) \in S_n$. Trdimo, da je $\sigma$ kohezivno zaporedje grafa $G_{\sigma}$. 
    \begin{enumerate}[label=(\alph*)]
        \item Če je $i < k < j$ in $\sigma(i)\sigma(k), \sigma(k)\sigma(j) \in E(G_{\sigma})$, potem sta $(\sigma(i),\sigma(k))$ in $(\sigma(k),\sigma(j))$ inverziji permutacije $\sigma$. To pomeni $\sigma(i) > \sigma(k) > \sigma(j)$. Zato je tudi $(\sigma(i),\sigma(j))$ inverzija permutacije $\sigma$ in $\sigma(i)\sigma(j) \in E(G_{\sigma})$.        

        \item Naj bo $\sigma(i)\sigma(j) \in E(G_{\sigma})$ in $k$ tak, da $i < k < j$. Potem je $(\sigma(i),\sigma(j))$ inverzija permutacije $\sigma$ in $\sigma(i) > \sigma(j)$. Če je $\sigma(i) > \sigma(k)$ je $(\sigma(i),\sigma(k))$ inverzija permutacije $\sigma$ in $\sigma(i)\sigma(k) \in E(G_{\sigma})$. Če je $\sigma(k) > \sigma(i)$, potem je tudi $\sigma(k) > \sigma(j)$ in je $(\sigma(k), \sigma(j))$ inverzija permutacije $\sigma$ in $\sigma(k)\sigma(j) \in E(G_{\sigma})$. To pomeni, da je 
        $\sigma(i)\sigma(k) \in E(G_{\sigma})$ ali $\sigma(k)\sigma(j) \in E(G_{\sigma})$.
    \end{enumerate}
\end{dokaz}

Zaporedje vozlišč $(v_1, v_2, \dots, v_n)$ je kohezivno zaporedje grafa $G$ natanko tedaj, ko je zaporedje vozlišč $(v_n, v_{n-1}, \dots, v_1)$ kohezivno zaporedje grafa $G$, saj sta oba pogoja (a) in (b) iz definicije \ref{def_kohezivno_zaporedje} hkrati izpolnjena ali neizpolnjena za obe zaporedji.

Pri dokazu izreka \ref{izrek_permutacijski_graf_kohezivno_zaporedje} bomo uporabili tranzitivne turnirje, zato si poglejmo, kaj je turnir in kaj je tranzitiven graf.

Za usmerjen graf $D$ rečemo, da je tranzitiven, če je $(x, z)$ usmerjena povezava grafa $D$, kadar sta $(x, y)$ in $(y, z)$ usmerjeni povezavi grafa $D$. 

Polnemu orientiranemu grafu rečemo turnir. Rezultat vozlišča $x$ v turnirju je izhodna stopnja vozlišča $x$. Označimo ga s $s(x)$. Rezultatsko zaporedje turnirja je zaporedje rezultatov vozlišč turnirja v nepadajočem vrstnem redu.

Obstaja natanko en tranzitiven turnir na $n$ vozliščih (do izomorfizma natančno). Tranzitiven turnir na $n$ vozliščih je izomorfen permutacijskemu grafu permutacije $\sigma = (n, n-1, \dots, 1)$ z usmerjenimi povezavami $x \rightarrow y$, če je $(x, y)$ inverzija. Opazimo tudi, da v tranzitivnem turnirju ni usmerjenih ciklov. 

\begin{figure}[h]
    \begin{center}        
        \begin{tikzpicture}[shorten >=1pt,-]
            \tikzstyle{vertex}=[circle,draw=black,fill=white!25,minimum size=20pt,inner sep=0pt]
            \foreach \num/\x in {1/1, 2/3}
                    \node[vertex] (V-\num) at (\x,0) {$\num$};
            \foreach \num/\x in {3/1, 4/3}
                    \node[vertex] (V-\num) at (\x,2) {$\num$};
            \foreach \num/\x in {4/5, 3/7, 2/9, 1/11}
                    \node[vertex] (VV-\num) at (\x,0) {$\num$};

                \foreach \from/\to in {1/2, 2/4, 4/3, 1/3, 3/2, 4/1}
                    \draw[-latex] (V-\from) -- (V-\to);
                \foreach \from/\to in {4/3, 3/2, 2/1}
                    \draw[-latex] (VV-\from) -- (VV-\to);

                \draw[-latex] (VV-4) .. controls (5.5, 1) and (8.5, 1) .. (VV-2);
                \draw[-latex] (VV-3) .. controls (7.5, 1) and (10.5, 1) .. (VV-1);
                \draw[-latex] (VV-4) .. controls (5.5, 1.5) and (10.5, 1.5) .. (VV-1);
        \end{tikzpicture}
    \end{center}
    \caption{Levo je turnir, desno je tranzitiven turnir na $4$ vozliščih.}
    \label{graf_turnirja_in_tranzitivnega_turnirja}
\end{figure}

\begin{izrek}
\label{izrek_tranzitiven_turnir}
    Naj bo $T$ turnir na $n$ vozliščih. Naslednje trditve so ekvivalentne:
    \begin{enumerate}
        \item $T$ je tranzitiven.
        \item Za vsaka $x,y \in V(T)$ velja, da če je $(x, y)$ usmerjena povezava v $T$, potem je $s(x) > s(y)$.
        \item Za vsaka $x,y \in V(T)$  velja, da če je $s(x) > s(y)$, potem je $(x, y)$ usmerjena povezava v $T$.
        \item Rezultatsko zaporedje turnirja $T$ je $(0, 1, 2, \dots, n-1)$.
    \end{enumerate}
\end{izrek}
\begin{dokaz}
    Tranzitiven turnir $T$ na $n$ vozliščih je izomorfen grafu permutacije $\sigma = (n, n-1, \dots, 1)$ z usmerjenimi povezavami $x \rightarrow y$, če je $(x, y)$ inverzija. Če uredimo vozlišča od leve proti desni tako, kot so v permutaciji $\sigma$, vidimo, da ima vsako vozlišče povezave do vseh vozlišč desno od njega (slika \ref{graf_turnirja_in_tranzitivnega_turnirja} desno). Iz tega sledijo vse lastnosti iz izreka.
\end{dokaz}

\begin{izrek}
\label{izrek_permutacijski_graf_kohezivno_zaporedje}
    Graf $G$ je permutacijski graf natanko tedaj, ko ima kohezivno zaporedje.
\end{izrek}
\begin{dokaz}
    $(\Rightarrow)$ Vsak permutacijski graf $G$ je po definiciji izomorfen nekemu grafu $G_{\sigma}$ za neko permutacijo $\sigma$. Po izreku \ref{izrek_sigma_kohezivno_zaporedje} je $\sigma = (\sigma(1), \dots, \sigma(n))$ kohezivno zaporedje grafa $G_{\sigma}$. Naj bo $f$ izomorfizem, ki graf $G$ slika v graf $G_{\sigma}$. Potem je $g = f^{-1}$ izomorfizem, ki graf $G_{\sigma}$ slika v graf $G$. Sledi, da je $\pi = (g(\sigma(1)), \dots, g(\sigma(n)))$ kohezivno zaporedje grafa $G$, saj je $\sigma$ kohezivno zaporedje grafa $G_{\sigma}$ (slika \ref{graf_izomorfizem_f_k3}).

    \begin{figure}[h]
        \begin{center}        
            \begin{tikzpicture}[shorten >=1pt,-]
                \tikzstyle{vertex}=[circle,draw=black,fill=white!25,minimum size=20pt,inner sep=0pt]
                \foreach \num/\x in {1/1, 3/3}
                    \node[vertex] (V-\num) at (\x,0) {$v_\num$};
                    
                    \node[vertex] (V-2) at (2, 1.3) {$v_2$};
                
                \foreach \num/\x in {3/5.5, 2/7, 1/8.5}
                    \node[vertex] (VV-\num) at (\x,0) {$\num$};
    
                    \foreach \from/\to in {1/2, 2/3, 3/1}
                        \draw (V-\from) -- (V-\to);
                    \foreach \from/\to in {3/2, 2/1}
                        \draw (VV-\from) -- (VV-\to);
    
                    \draw (VV-3) .. controls (6, 1) and (8, 1) .. (VV-1);
            \end{tikzpicture}
        \end{center}
        \caption{Izomorfna grafa $G$ in $G_{\sigma}$.}
        \label{graf_izomorfizem_f_k3}
    \end{figure}

    $(\Leftarrow)$ Naj bo $G$ graf s kohezivnim zaporedjem $\pi = (v_1, v_2, \dots, v_n)$ (slika \ref{graf_s_kohezivnim_zaporedjem}). 
    \begin{figure}[h]
        \begin{center}
            \begin{tikzpicture}[shorten >=1pt,-]
                \tikzstyle{vertex}=[circle,draw=black,fill=white!25,minimum size=20pt,inner sep=0pt]
                \foreach \num/\x in {1/1, 2/3, 3/5, 4/7}
                        \node[vertex] (V-\num) at (\x,0) {$v_{\num}$};
            
                    \foreach \from/\to in {1/2, 3/4}
                        \draw (V-\from) -- (V-\to);
            
                    \draw (V-1) .. controls (1.5, 1.5) and (6.5, 1.5) .. (V-4);
            \end{tikzpicture}
        \end{center}
        \caption{Graf $G$ s kohezivnim zaporedjem $(v_1, v_2, v_3, v_4)$.}
        \label{graf_s_kohezivnim_zaporedjem}
    \end{figure}
    Orientirajmo graf $G$ tako, da vse povezave usmerimo od vozlišča z manjšim indeksom proti vozlišču z večjim indeksom. Če je $v_iv_j \in E(G)$ in $i < j$, potem dobimo $(v_i, v_j)$. Označimo usmerjen graf, ki ga na ta način dobimo z $D$. Spomnimo se, da je zaradi pogoja $(a)$ iz definicije \ref{def_kohezivno_zaporedje} graf $D$ tranzitiven. Orientirajmo še komplement $\overline{G}$ grafa $G$. Povezave $v_iv_j \in E(\overline{G})$, kjer je $i < j$, usmerimo od večjega indeksa k manjšemu in tako dobimo $(v_j, v_i)$. Označimo dobljeni graf z $\overline{D}$. Po lemi \ref{lema0} je $\pi$ kohezivno zaporedje grafa $\overline{G}$. Po definiciji \ref{def_kohezivno_zaporedje} je tudi usmerjen graf $\overline{D}$ tranzitiven. Unija grafov $T = D \cup \overline{D}$ je turnir, to je orientacija polnega grafa $G \cup \overline{G}$ (slika \ref{graf_tranzitivni_turnir}).
    \begin{figure}[h]
        \begin{center}
            \begin{tikzpicture}[shorten >=1pt,-]
                \tikzstyle{vertex}=[circle,draw=black,fill=white!25,minimum size=20pt,inner sep=0pt]
                \foreach \num/\x in {1/1, 2/3, 3/5, 4/7}
                        \node[vertex] (V-\num) at (\x,0) {$v_{\num}$};
            
                    \foreach \from/\to in {1/2, 3/4}
                        \draw[-latex] (V-\from) -- (V-\to);
                    \draw[-latex] (V-1) .. controls (1.5, 1.5) and (6.5, 1.5) .. (V-4);
                    
                    
                    \draw[dashed, -latex] (V-3) .. controls (4, 0.85) and (2, 0.85) .. (V-1);
                    \draw[dashed, -latex] (V-4) .. controls (6, 0.85) and (4, 0.85) .. (V-2);
                    \draw[dashed, -latex] (V-3) -- (V-2);
            \end{tikzpicture}
        \end{center}
        \caption{ Tranzitiven turnir $T$.}
        \label{graf_tranzitivni_turnir}
    \end{figure}
    Radi bi pokazali, da je $T$ tranzitiven turnir. 
    
    Naj bosta $(x, y)$ in $(y, z)$ usmerjeni povezavi v grafu $T$. Če bi obe pripadali istemu grafu $D$ ali $\overline{D}$, bi sledilo, da je $(x, z)$ usmerjena povezava v $T$, saj sta $D$ in $\overline{D}$ tranzitivna. Zato brez škode za splošnost privzamimo, da je $(x, y) \in E(D) $ in $(y, z) \in E(\overline{D})$. Če je $(x, z) \in E(D)$ smo končali, saj je potem $(x, z) \in E(T)$. Zato privzamimo da $(x, z) \notin E(D)$. Poglejmo, ali je lahko $(z, x) \in E(D)$. Zaradi tranzitivnosti grafa $D$ bi to pomenilo, da je tudi $(z, y) \in E(D)$, kar je v protislovju s tem, da je  $(y, z) \in E(\overline{D})$. Potem je $(z, x) \in E(\overline{D})$ ali $(x, z) \in E(\overline{D})$. Če je $(z, x) \in E(\overline{D})$, potem zaradi tranzitivnosti $\overline{D}$ in $(y, z), (z, x) \in E(\overline{D})$ sledi, da je $(y, x) \in E(\overline{D})$. To je v protislovju z $(x, y) \in E(D)$. Zato je $(x, z) \in E(\overline{D})$. Sledi, da je $(x, z) \in E(T)$ in $T$ je tranzitiven turnir. Po izreku \ref{izrek_tranzitiven_turnir} je rezultatsko zaporedje tranzitivnega turnirja $T$ enako $(0, 1, 2, \dots, n-1)$. 
    \begin{figure}[h]
        \begin{center}
            \begin{tikzpicture}[shorten >=1pt,-]
                \tikzstyle{vertex}=[circle,draw=black,fill=white!25,minimum size=20pt,inner sep=0pt]
                \foreach \num/\x in {3/1, 1/3, 4/5, 2/7}
                        \node[vertex] (V-\num) at (\x,0) {\num};
            
                    \foreach \from/\to in {3/1, 4/2}
                        \draw (V-\from) -- (V-\to);
                    \draw (V-3) .. controls (1.5, 1.5) and (6.5, 1.5) .. (V-2);
                    
                    
                    % \draw[dotted, -latex] (V-3) .. controls (4, 0.85) and (2, 0.85) .. (V-1);
                    % \draw[dotted, -latex] (V-4) .. controls (6, 0.85) and (4, 0.85) .. (V-2);
                    % \draw[dotted, -latex] (V-3) -- (V-2);
            \end{tikzpicture}
        \end{center}
        \caption{Permutacijski graf $G_\sigma$, $\sigma = (3, 1, 4, 2)$.}
        \label{graf_permutacije3142}
    \end{figure}
    Rezultat vozlišča $v_i$ tranzitivnega turnirja $T$ je $s(v_i)$ (slika \ref{graf_tranzitivni_turnir}).
    
    Naj bo $\sigma(i) = 1 + s(v_i)$ (slika \ref{graf_permutacije3142}). Radi bi pokazali, da je preslikava $f: v_i \rightarrow 1 + s(v_i) = \sigma(i)$ izomorfizem, ki slika graf $G$ v graf $G_{\sigma}$. Preslikava $f$ je bijektivna, saj imajo vozlišča različne rezultate. Pokazati moramo še, da $f$ ohranja sosednosti vozlišč. Naj bo $v_iv_j \in E(G)$, kjer je $i < j$. Potem je $(v_i, v_j) \in E(D)$. Ker je $T$ tranzitiven turnir, je $s(v_i) > s(v_j)$ (izrek \ref{izrek_tranzitiven_turnir}). Sledi, da je $\sigma(v_i) = 1 + s(v_i) > 1 + s(v_j) = \sigma(v_j)$. Zato je $(\sigma(i), \sigma(j))$ inverzija v $\sigma$ in $f(v_i)f(v_j) \in E(G_{\sigma})$. Obratno, naj bo $xy \in E(G_{\sigma})$. Potem je $(x, y)$ ali $(y, x)$ inverzija v $\sigma$. Privzemimo, da je $(x, y)$ inverzija v $\sigma$. Potem je $x = \sigma(i) = 1 + s(v_i)$ in $y = \sigma(j) = 1 + s(v_j)$, $i < j$. Ker je $(x, y)$ inverzija, je $x > y$. Potem je tudi $s(v_i) > s(v_j)$ in $(v_i, v_j) \in E(T)$ (izrek \ref{izrek_tranzitiven_turnir}). Ker je $i < j$, je $(v_i, v_j) \in E(D)$ in posledično $v_iv_j \in E(G)$.
\end{dokaz}


\begin{izrek}
\label{izrek_ekvivalentne_trditve_permutacijski_graf}
    Naj bo $G$ neusmerjen graf. Naslednje trditve so ekvivalentne:
    \begin{enumerate}[label=(\alph*)]
        \item $G$ je permutacijski graf.
        \item $\overline{G}$ je permutacijski graf.
        \item Vsak induciran podgraf grafa $G$ je permutacijski graf.
        \item Vsaka povezana komponenta grafa $G$ je permutacijski graf.
    \end{enumerate}
\end{izrek}
\begin{dokaz}
    Ekvivalentnost trditve $(a)$ in $(b)$ sledi iz leme \ref{lema0} in izreka \ref{izrek_permutacijski_graf_kohezivno_zaporedje}. Naj bo $G$ permutacijski graf. Po izreku \ref{izrek_permutacijski_graf_kohezivno_zaporedje} ima graf $G$ kohezivno zaporedje $(v_1, v_2, \dots, v_n)$. Induciran podgraf z vozlišči $\{ v_{i_1}, v_{i_2}, \dots, v_{i_k} \}$, kjer $i_1 < i_2 < \cdot\cdot\cdot < i_k$, ima kohezivno zaporedje $(v_{i_1}, v_{i_2}, \dots, v_{i_k})$, saj sta izpolnjena pogoja $(a)$ in $(b)$ iz definicije \ref{def_kohezivno_zaporedje}. Torej je vsak induciran podgraf permutacijski in iz $(a)$ sledi $(c)$. Iz $(c)$ sledi $(d)$, saj je vsaka povezana komponenta induciran podgraf. Pokazati moramo še, da iz $(d)$ sledi $(a)$. Naj bo $G$ graf, ki ima povezane komponente $G_1, G_2, \dots, G_k$. Naj bo $n_i$ število vozlišč grafa $G_i$. Ker je vsaka povezana komponenta grafa $G$ permutacijski graf, ima kohezivno zaporedje. Naj bo $l_i = (v_1^i, v_2^i, \dots, v_{n_i}^i)$ kohezivno zaporedje povezane komponente $G_i$. Potem je 
    \[
        l = (l_1, l_2, \dots, l_k) = (v_1^1, v_2^1, \dots, v_{n_1}^1, v_1^2, v_2^2, \dots, v_{n_2}^2, \dots, v_1^k, v_2^k, \dots, v_{n_k}^k)
    \]
    kohezivno zaporedje grafa $G$ in graf $G$ permutacijski. 
\end{dokaz}

\section{ Drevesa }

V tem podpoglavju si bomo pogledali, katera drevesa so permutacijski grafi. Pokazali bomo, da za $n \geq 3$ obstajata natanko dve permutaciji iz $S_n$, katerih permutacijski graf je pot na $n$ vozliščih.

\begin{trditev}
\label{trditev_zvezde_permutacijski_graf}
    Zvezda $K_{1,n}$ je permutacijski graf.
\end{trditev}
\begin{dokaz}
    Permutacija $\pi = (n+1, 1, 2, \dots, n)$ je kohezivno zaporedje zvezde $K_{1, n}$ (slika \ref{graf_kohezivnega_zaporedja_zvezda}), saj sta izpolnjena pogoja $(a)$ in $(b)$ iz definicije kohezivnega zaporedja \ref{def_kohezivno_zaporedje}. Po izreku \ref{izrek_permutacijski_graf_kohezivno_zaporedje} je zvezda $K_{1, n}$ permutacijski graf, saj ima kohezivno zaporedje.
\end{dokaz}

\begin{figure}[h]
    \begin{center}        
        \begin{tikzpicture}[shorten >=1pt,-]
            \tikzstyle{vertex}=[circle,draw=black,fill=white!25,minimum size=28pt,inner sep=0pt]
            \foreach \num/\x in {1/3, 2/5, n/9}
                    \node[vertex] (V-\num) at (\x,0) {$\num$};
                    
                \node[vertex] (V-n+1) at (1,0) {$n\!+\!1$};
                \foreach \from/\to in {n+1/1}
                    \draw (V-\from) -- (V-\to);

                \draw (V-n+1) .. controls (1.5, 0.75) and (4.5, 0.75) .. (V-2);
                \draw (V-n+1) .. controls (1.5, 1.5) and (8.5, 1.5) .. (V-n);

                
                \filldraw [black] (6.5,0) circle (1pt);
                \filldraw [black] (7,0) circle (1pt);
                \filldraw [black] (7.5,0) circle (1pt);
        \end{tikzpicture} 
    \end{center}
    \caption{Primer kohezivnega zaporedja za zvezdo $K_{1, n}$.}
    \label{graf_kohezivnega_zaporedja_zvezda}
\end{figure}

\begin{figure}[h]
    \begin{center}        
        \begin{tikzpicture}[shorten >=1pt,-]
            \tikzstyle{vertex}=[circle,draw=black,fill=white!25,minimum size=20pt,inner sep=0pt]
            \foreach \num/\x in {v_2/1, v_1/3, v_4/5, v_3/7, v_6/9, v_5/11}
                    \node[vertex] (V-\num) at (\x,0) {$\num$};
                    
                \foreach \from/\to in {v_2/v_1, v_4/v_3, v_6/v_5}
                    \draw (V-\from) -- (V-\to);
        
                \draw (V-v_2) .. controls (1.5, 1.5) and (6.5, 1.5) .. (V-v_3);
                \draw (V-v_4) .. controls (5.5, 1.5) and (10.5, 1.5) .. (V-v_5);
                
                \filldraw [black] (9.2,0.5) circle (1pt);
                \filldraw [black] (9.6,0.8) circle (1pt);
                \filldraw [black] (10.1,1) circle (1pt);
                \filldraw [black] (10.6,1.1) circle (1pt);
                \filldraw [black] (11.1,1.18) circle (1pt);
                \filldraw [black] (11.6,1.1) circle (1pt);
                \filldraw [black] (12.1,1) circle (1pt);
                \filldraw [black] (12.6,0.8) circle (1pt);
                \filldraw [black] (13,0.5) circle (1pt);
        \end{tikzpicture}     
    \end{center}
    \caption{Primer kohezivnega zaporedja za pot $P_n$.}
    \label{graf_kohezivnega_zaporedja_pot}
\end{figure}

\begin{trditev}
\label{trditev_pot_permutacijski_graf}
    Pot $P_n$ je permutacijski graf.
\end{trditev}
\begin{dokaz}
    Naj bo $P_n$ pot na $n$ vozliščih. Naj bo prvo vozlišče na poti označeno z $v_1$, drugo z $v_2$, $\dots$, in zadnje z $v_n$ (slika \ref{graf_kohezivnega_zaporedja_pot}). Če je $n$ sodo število, potem je $(v_2, v_1, v_4, v_3,\dots,v_{n}, v_{n-1})$ kohezivno zaporedje poti $P_n$. Če je $n$ liho število, potem je $(v_2, v_1, v_4, v_3,\dots,v_{n-1}, v_{n-2}, v_{n})$ kohezivno zaporedje poti $P_n$. To je res, saj sta izpolnjena pogoja $(a)$ in $(b)$ iz definicije \ref{def_kohezivno_zaporedje}. Po izreku \ref{izrek_permutacijski_graf_kohezivno_zaporedje} je pot $P_n$ permutacijski graf, saj ima kohezivno zaporedje.
\end{dokaz}

% \begin{figure}[h]
%     \begin{center}        
%         \begin{tikzpicture}[shorten >=1pt,-]
%             \tikzstyle{vertex}=[circle,draw=black,fill=white!25,minimum size=20pt,inner sep=0pt]
%             \foreach \num/\x in {3/1, 1/2, 5/3, 2/4, 7/5, 4/6, 9/7, 6/8, 11/9, 8/10, 10/12}
%                     \node[vertex] (V-\num) at (\x,0) {$\num$};
        
%                 \foreach \from/\to in {3/1, 5/2, 7/4, 9/6, 11/8}
%                     \draw (V-\from) -- (V-\to);
        
%                 \draw (V-3) .. controls (1.5, 0.75) and (3.5, 0.75) .. (V-2);
%                 \draw (V-5) .. controls (3.5, 0.75) and (5.5, 0.75) .. (V-4);
%                 \draw (V-7) .. controls (5.5, 0.75) and (7.5, 0.75) .. (V-6);
%                 \draw (V-9) .. controls (7.5, 0.75) and (9.5, 0.75) .. (V-8);
%                 \draw (V-11) .. controls (9.5, 0.75) and (11.5, 0.75) .. (V-10);
%                 % \draw (V-13) .. controls (11.5, 0.75) and (12.5, 0.75) .. (V-12);
                
%                 \filldraw [black] (11,0) circle (.5pt);
%                 \filldraw [black] (11.33,0) circle (.5pt);
%                 % \filldraw [black] (11.5,0) circle (.5pt);
%                 % \filldraw [black] (12.5,0) circle (.5pt);
%                 % \filldraw [black] (12.75,0) circle (.5pt);
%                 % \filldraw [black] (13,0) circle (.5pt);
                
%                 \filldraw [black] (11,0) circle (.5pt);
%                 \filldraw [black] (11.6,0.53) circle (.5pt);
%                 \filldraw [black] (12.2,0.63) circle (.5pt);
%                 \filldraw [black] (12.8,0.63) circle (.5pt);
%                 \filldraw [black] (13.4,0.53) circle (.5pt);
%                 \filldraw [black] (14,0) circle (.5pt);
%         \end{tikzpicture}     
%     \end{center}
%     \caption{Primer kohezivnega zaporedja za pot $P_n$.}
%     \label{graf_kohezivnega_zaporedja_pot}
% \end{figure}


% Grafi poti $P_n$ in zvezd $K_{1,n}$ so permutacijski grafi, saj imajo kohezivno zaporedje (sliki \ref{graf_kohezivnega_zaporedja_pot} in \ref{graf_kohezivnega_zaporedja_zvezda}). 
\begin{primer}
Kohezivni zaporedji poti $P_{11}$ sta permutaciji:
\[
    \sigma = (3, 1, 5, 2, 7, 4, 9, 6, 11, 8, 10) \quad \text{in} \quad \pi = (2, 4, 1, 6, 3, 8, 5, 10, 7, 11, 9). 
\]
Permutacijski graf permutacije $\sigma$ je tak, kot v dokazu izreka \ref{trditev_pot_permutacijski_graf}. Permutacijski graf permutacije $\pi$ je zrcalna slika permutacijskega grafa permutacije $\sigma$.
Vidimo, da obstajata vsaj dve permutaciji, katerih permutacijski graf je pot $P_{11}$. V izreku \ref{izrek_dve_permutaciji_poti} bomo pokazali, da sta to tudi edini permutaciji, katerih permutacijski graf je pot $P_{11}$.
\end{primer}
% Vidimo, da je pogoj $(a)$ iz definicije kohezivnega zaporedja \ref{def_kohezivno_zaporedje} za poti $P_n$ na prazno izpolnjen, saj ni dveh povezav $v_iv_k$ in $v_kv_j$, kjer so $i < k < j$. Pogoj $(b)$ pa je izpolnjen, saj je vedno, ko je $v_iv_j$ povezava in $k$ tak, da $i < k < j$, v grafu ena od povezav $v_iv_k$ ali $v_kv_j$. Podobno je pogoj $(a)$ izpolnjen na prazno za zvezde. Pogoj $(b)$ pa je izpolnjen, saj so vse povezave v grafu oblike $v_{n+1}v_k$ za $k \in \{ 1, 2, \dots, n \}$. Zato katerokoli povezavo vzamemo, bodo vsa vozlišča med krajiščema izbrane povezave $v_{n+1}v_k$ povezana z vozliščem $v_{n+1}$.



% \begin{tikzpicture}[shorten >=1pt,-]
%     \tikzstyle{vertex}=[circle,draw=black,fill=white!25,minimum size=20pt,inner sep=0pt]
%     \foreach \num/\x in {2/1, 4/2, 1/3, 6/4, 3/5, 8/6, 5/7, 10/8, 7/9, 12/10, 9/11, 13/12, 11/13}
%             \node[vertex] (V-\num) at (\x,0) {$\num$};

%         \foreach \from/\to in {4/1, 6/3, 8/5, 10/7, 12/9, 13/11}
%             \draw (V-\from) -- (V-\to);

%         \draw (V-2) .. controls (1.5, 0.75) and (2.5, 0.75) .. (V-1);
%         \draw (V-4) .. controls (2.5, 0.75) and (4.5, 0.75) .. (V-3);
%         \draw (V-6) .. controls (4.5, 0.75) and (6.5, 0.75) .. (V-5);
%         \draw (V-8) .. controls (6.5, 0.75) and (8.5, 0.75) .. (V-7);
%         \draw (V-10) .. controls (8.5, 0.75) and (10.5, 0.75) .. (V-9);
%         \draw (V-12) .. controls (10.5, 0.75) and (12.5, 0.75) .. (V-11);
% \end{tikzpicture}



Poti in zvezde so primeri dreves, ki so permutacijski grafi. Ampak niso vsa drevesa permutacijski grafi. 
Drevo $K_{1,3}^*$ je graf pridobljen s subdivizijo vseh povezav zvezde $K_{1,3}$ (slika \ref{graf_subdivizija_k13}). Trditev \ref{trditev_k13*_ni_permutacijski_graf} nam pove, da drevo $K_{1,3}^*$ ni permutacijski graf.

\begin{figure}[h]
    \begin{center}        
        \begin{tikzpicture}[shorten >=1pt,-]
            \tikzstyle{vertex}=[circle,draw=black,fill=white!25,minimum size=20pt,inner sep=0pt]
                    \node[vertex] (V-0) at (4,4){};
                    \node[vertex] (V-1) at (4,1.75){};
                    \node[vertex] (V-2) at (2,0){};
                    \node[vertex] (V-3) at (6,0){};
        
                \foreach \from/\to in {0/1, 1/2, 1/3}
                    \draw (V-\from) -- (V-\to);
        \end{tikzpicture}\qquad
        \begin{tikzpicture}[shorten >=1pt,-]
            \tikzstyle{vertex}=[circle,draw=black,fill=white!25,minimum size=20pt,inner sep=0pt]
                    \node[vertex] (V-0) at (4,4){};
                    \node[vertex] (V-01) at (4,2.90){};
                    \node[vertex] (V-1) at (4,1.75){};
                    \node[vertex] (V-12) at (3, 0.875){};
                    \node[vertex] (V-2) at (2,0){};
                    \node[vertex] (V-13) at (5,0.875){};
                    \node[vertex] (V-3) at (6,0){};
        
                \foreach \from/\to in {0/01, 12/2, 13/3, 01/1, 1/12, 1/13}
                    \draw (V-\from) -- (V-\to);
        \end{tikzpicture}
    \end{center}
    \caption{Grafa $K_{1, 3}$ in $K_{1, 3}^*$.}
    \label{graf_subdivizija_k13}
\end{figure}

\begin{trditev}
\label{trditev_k13*_ni_permutacijski_graf}
    Drevo $K_{1,3}^*$ ni permutacijski graf.
\end{trditev}
\begin{dokaz}
    Če hočemo pokazati, da graf $K_{1,3}^*$ ni permutacijski, moramo pokazati, da nima kohezivnega zaporedja. 
    
    Najprej si oglejmo, kakšno kohezivno zaporedje ima podgraf $K_{1, 3}$ grafa $K_{1, 3}^*$. Iz dokaza trditve \ref{trditev_zvezde_permutacijski_graf} dobimo kohezivno zaporedje zvezde prikazano na sliki \ref{graf_kohezivnega_zaporedja_zvezda}. Opazimo, da ima zvezda $K_{1, 3}$ poleg kohezivnega zaporedja, kjer so vsi listi desno od vozlišča s stopnjo $3$ tudi kohezivno zaporedje, kjer so vsi listi levo od vozlišča s stopnjo $3$. To sta edini možnosti, saj če bi bili listi levo in desno od vozlišča s stopnjo $3$, potem pogoj $(a)$ iz definicije \ref{def_kohezivno_zaporedje} ne bi bil izpolnjen. Zaradi simetrije si lahko poglejmo samo primer, ko so listi desno od vozlišča s stopnjo $3$ (slika \ref{graf_kohezivnega_zaporedja_zvezda}). 
    
    Da bi našli kohezivno zaporedje grafa $K_{1,3}^*$, moramo kohezivnemu zaporedju iz prejšnega odstavka dodati še preostala vozlišča (liste grafa $K_{1,3}^*$). Če vozlišče, ki je sosednje srednjemu listu podgrafa $K_{1,3}$ v kohezivnem zaporedju dodamo na katerokoli mesto (slika \ref{slika_k12*_ni}) ne dobimo kohezivnega zaporedja, saj nikoli nista izpolnjena oba pogoja iz definicije \ref{def_kohezivno_zaporedje}. Zato tudi graf $K_{1,3}^*$ nima kohezivnega zaporedja in ni permutacijski graf.
\end{dokaz}

\begin{figure}[h]
    \begin{center}        
        \begin{tikzpicture}[shorten >=1pt,-]
            \tikzstyle{vertex}=[circle,draw=black,fill=white!25,minimum size=20pt,inner sep=0pt]
            \tikzstyle{vertex2}=[circle,draw=black,dotted,fill=white!25,minimum size=20pt,inner sep=0pt]
            \foreach \num/\x in {1/1, 2/3, 3/5, 4/7}
                    \node[vertex] (V-\num) at (\x,0) {};

            \foreach \num/\x in {5/0, 6/2, 7/4, 8/6, 9/8}
                    \node[vertex2] (V-\num) at (\x,0) {};
                    
                \foreach \from/\to in {3/7, 3/8}
                    \draw[dotted] (V-\from) -- (V-\to);

                \draw (V-1) .. controls (1.5, 0.75) and (2.5, 0.75) .. (V-2);
                \draw (V-1) .. controls (1.5, 1.1) and (4.5, 1.1) .. (V-3);
                \draw (V-1) .. controls (1.5, 1.5) and (6.5, 1.5) .. (V-4);
                
                \draw[dotted] (V-5) .. controls (0.5, -1.5) and (4.5, -1.5) .. (V-3);
                \draw[dotted] (V-6) .. controls (2.5, -1) and (4.5, -1) .. (V-3);
                \draw[dotted] (V-3) .. controls (5.5, -1) and (7.5, -1) .. (V-9);

        \end{tikzpicture} 
    \end{center}
    \caption{Nobena od pozicij za pikčasto vozlišče nam ne da kohezivnega zaporedja grafa, saj nikoli nista izpolnjena oba pogoja iz definicije \ref{def_kohezivno_zaporedje}.}
    \label{slika_k12*_ni}
\end{figure}

\begin{figure}[h]
    \begin{center}        
        \begin{tikzpicture}[shorten >=1pt,-]
            \tikzstyle{vertex}=[circle,draw=black,fill=white!25,minimum size=20pt,inner sep=0pt]
                    \node[vertex] (V-0) at (0, 1){};
                    \node[vertex] (V-1) at (2, 1){};
                    \node[vertex] (V-2) at (4, 1){};
                    \node[vertex] (V-3) at (6, 1){};
                    \node[vertex] (V-4) at (8, 1){};
                    \node[vertex] (V-5) at (0, 2){};
                    \node[vertex] (V-6) at (2, 2){};
                    \node[vertex] (V-7) at (1, 2){};
                    \node[vertex] (V-8) at (3, 2){};
                    \node[vertex] (V-9) at (4, 0){};
                    \node[vertex] (V-10) at (5, 0){};
                    \node[vertex] (V-11) at (8, 0){};
                    \node[vertex] (V-12) at (7, 2){};
                    \node[vertex] (V-13) at (8, 2){};
                    \node[vertex] (V-14) at (9, 2){};
        
                \foreach \from/\to in {0/1, 1/2, 2/3, 3/4, 0/5, 1/6, 1/7, 1/8, 2/9, 2/10, 4/11, 4/12, 4/13, 4/14}
                    \draw (V-\from) -- (V-\to);
        \end{tikzpicture}
    \end{center}
    \caption{Primer gosenice z $10$ listi.}
    \label{graf_gosenice_10_listov}
\end{figure}

\begin{definicija}
    Drevo je gosenica, če po odstranitvi vseh listov dobimo pot.
\end{definicija}

\begin{lema}
\label{lema_gosenica_ne_vsebuje_k13}
    Drevo je gosenica natanko tedaj, ko ne vsebuje podgrafa $K_{1,3}^*$.
\end{lema}
\begin{dokaz}
    Če je drevo gosenica, potem po odstranitvi vseh listov dobimo pot. Če drevesu $K_{1,3}^*$ odstranimo vse liste, ne dobimo poti. Torej tudi če drevo vsebuje $K_{1,3}^*$ kot podgraf, nam po odstranitvi listov ostane graf, ki ni pot. Torej gosenica ne vsebuje podgrafa $K_{1,3}^*$. Če drevo ne vsebuje podgrafa $K_{1,3}^*$, potem ima vsako vozlišče največ dva soseda, ki nista lista (če bi neko vozlišče imelo tri sosede, ki niso listi, potem je $K_{1,3}^*$ podgraf drevesa). Prav tako graf ne vsebuje ciklov, saj je drevo. Po odstranitvi listov drevesa vedno dobimo povezan graf in stopnja vsakega vozlišča je $\leq 2$, zato po odstranitvi listov tako dobimo pot. Torej je drevo, ki ne vsebuje podgrafa $K_{1,3}^*$, gosenica.
\end{dokaz}

% \begin{figure}[h]
%     \begin{center}        
%         \begin{tikzpicture}[shorten >=1pt,-]
%             \tikzstyle{vertex}=[circle,draw=black,fill=white!25,minimum size=20pt,inner sep=0pt]
%             \foreach \num/\x in {l_1/0, 2/1, l_2/2, 1/3, l_3/4, 4/5, l_4/6, 3/7, l_5/8, 6/9, l_6/10, 5/11}
%                     \node[vertex] (V-\num) at (\x,0) {$\num$};
        
%                 \foreach \from/\to in {2/l_2, 4/l_4, 6/l_6}
%                     \draw[dashed] (V-\from) -- (V-\to);
        
%                 \draw[dashed] (V-l_1) .. controls (0.5, 1) and (2.5, 1) .. (V-1);
%                 \draw[dashed] (V-l_3) .. controls (4.5, 1) and (6.5, 1) .. (V-3);
%                 \draw[dashed] (V-l_5) .. controls (8.5, 1) and (10.5, 1) .. (V-5);
    
%                 \draw (V-2) .. controls (1.5, 0.75) and (2.5, 0.75) .. (V-1);
%                 \draw (V-4) .. controls (5.5, 0.75) and (6.5, 0.75) .. (V-3);
%                 \draw (V-6) .. controls (9.5, 0.75) and (10.5, 0.75) .. (V-5);
%                 \draw (V-2) .. controls (1.5, 1.5) and (6.5, 1.5) .. (V-3);
%                 \draw (V-4) .. controls (5.5, 1.5) and (10.5, 1.5) .. (V-5);
%         \end{tikzpicture}
%     \end{center}
%     \caption{Primer kohezivnega zaporedja gosenice.}
%     \label{graf_kohezivnega_zaporedja_gosenice}
% \end{figure}
\begin{figure}[h]
    \begin{center}        
        \begin{tikzpicture}[shorten >=1pt,-]
            \tikzstyle{vertex}=[circle,draw=black,fill=white!25,minimum size=20pt,inner sep=0pt]
            \foreach \num/\x in {l_1/0, v_2/1, v_1/2, l_2/3, l_3/4, v_4/5, v_3/6, l_4/7, l_5/8, v_6/9, v_5/10, l_6/11}
                    \node[vertex] (V-\num) at (\x,0) {$\num$};
        
                \foreach \from/\to in {v_2/v_1, v_4/v_3, v_6/v_5}
                    \draw (V-\from) -- (V-\to);
        
                \draw[dashed] (V-l_1) .. controls (0.5, 0.75) and (1.5, 0.75) .. (V-v_1);
                \draw[dashed] (V-l_3) .. controls (4.5, 0.75) and (5.5, 0.75) .. (V-v_3);
                \draw[dashed] (V-l_5) .. controls (8.5, 0.75) and (9.5, 0.75) .. (V-v_5);
    
                \draw[dashed] (V-v_2) .. controls (1.5, 0.75) and (2.5, 0.75) .. (V-l_2);
                \draw[dashed] (V-v_4) .. controls (5.5, 0.75) and (6.5, 0.75) .. (V-l_4);
                \draw[dashed] (V-v_6) .. controls (9.5, 0.75) and (10.5, 0.75) .. (V-l_6);
                \draw (V-v_2) .. controls (1.5, 1.33) and (5.5, 1.33) .. (V-v_3);
                \draw (V-v_4) .. controls (5.5, 1.33) and (9.5, 1.33) .. (V-v_5);
        \end{tikzpicture}
    \end{center}
    \caption{Primer kohezivnega zaporedja gosenice.}
    \label{graf_kohezivnega_zaporedja_gosenice}
\end{figure}
\begin{izrek}
\label{izrek_gosenica_permutacijski_graf}
    Drevo je permutacijski graf natanko tedaj, ko je gosenica.
\end{izrek}
\begin{dokaz}
    $(\Rightarrow)$ Če drevo ni gosenica, potem po lemi \ref{lema_gosenica_ne_vsebuje_k13} vsebuje $K_{1,3}^*$ kot podgraf. Drevo, ki vsebuje $K_{1,3}^*$, ni permutacijski graf, saj $K_{1,3}^*$ ni permutacijski graf. 
    
    $(\Leftarrow)$ Potrebno je še pokazati, da je vsaka gosenica permutacijski graf. 
    Po izreku \ref{izrek_permutacijski_graf_kohezivno_zaporedje} je graf permutacijski graf natanko tedaj, ko ima kohezivno zaporedje, zato bomo pokazali, da ima vsaka gosenica kohezivno zaporedje. Naj bo $C$ gosenica in naj bo $P_n$ pot, ki jo pridobimo iz $C$ tako, da odstranimo liste. Če je $n=1$, potem je $C$ zvezda $K_{1,k}$ za nek $k \geq 0$. Po trditvi \ref{trditev_zvezde_permutacijski_graf} so zvezde permutacijski grafi. Zato predpostavimo, da je $n \geq 2$. Kohezivno zaporedje poti $P_n$ je $(v_2, v_1, v_4, v_3, \dots)$, kot v dokazu trditve \ref{trditev_pot_permutacijski_graf}. Liste lihega vozlišča $v_i$ na poti ($i$ je liho število) vstavimo levo od vozlišča $v_{i+1}$ na poti $P_n$. Vse liste sodega vozlišča $v_i$ na poti ($i$ je sodo število) vstavimo desno od vozlišča $v_{i-1}$ na poti $P_n$. Rezultat je kohezivno zaporedje $(l_1, v_2, v_1, l_2, l_3, v_4, v_3, l_4, \dots)$, kjer $l_i$ predstavlja liste vozlišča $v_i$ (slika \ref{graf_kohezivnega_zaporedja_gosenice}). Zato je gosenica $C$ po izreku \ref{izrek_permutacijski_graf_kohezivno_zaporedje} res permutacijski graf.    
\end{dokaz}

Imejmo tako gosenico, da ko ji odstranimo vse liste, dobimo pot na $k$ vozliščih. Vozlišče $u_i$ na poti naj ima $m_i$ listov, $m_i \in \mathbb{N}_0$. Označimo tako gosenico s $C_k(m_1, m_2, \dots, m_k)$.

\begin{trditev}
\label{trditev_gosenica_vsaj_dve_permutaciji}
    Gosenica $C_k(m_1, m_2, \dots, m_k)$ je permutacijski graf vsaj dveh permutacij.
\end{trditev}
\begin{dokaz}
    Vsako drevo, z vsaj eno povezavo ima natanko dve tranzitivni orientaciji. To sledi iz dejstva, da ko usmerimo eno povezavo smo že določili tranzitivno orientacijo drevesa. Zato ima tudi gosenica $C_k(m_1, m_2, \dots, m_k)$ natanko dve tranizitivni orientaciji (slika \ref{tranzitivni_orientaciji_gosenice}).
    Naj bo $n = k + m_1 + m_2 + \cdot\cdot\cdot + m_k$ število vozlišč grafa $C_k(m_1, m_2, \dots, m_k)$. Priredimo števila $1, 2, \dots, n$ vozliščem grafa tako, da če je povezava $uv$ orientirana od $u$ proti $v$, potem mora biti število, ki ga priredimo vozlišču $u$, večje od števila, ki ga priredimo vozlišču $v$ (primer na sliki \ref{tranzitivni_orientaciji_gosenice}). To bomo naredili induktivno na dva načina (slike \ref{tranzitivni_orientaciji_gosenice}, \ref{gosenica_dve_permutaciji1}, \ref{gosenica_dve_permutaciji2}). 

    \begin{figure}[h]
        \begin{center}        
            \begin{tikzpicture}[shorten >=1pt,-]
                \tikzstyle{vertex}=[circle,draw=black,fill=white!25,minimum size=20pt,inner sep=0pt]
                \tikzstyle{vertex2}=[circle,draw=black!,fill=black!10,minimum size=20pt,inner sep=0pt]
                \foreach \num/\x in {5/1, 1/2, 8/5, 4/6}
                        \node[vertex] (V-\num) at (\x,0) {$\num$};
                \foreach \num/\x in {2/0, 3/3, 6/4, 7/7}
                        \node[vertex2] (V-\num) at (\x,0) {$\num$};
            
                    \foreach \from/\to in {5/1, 8/4}
                        \draw[-latex] (V-\from) -- (V-\to);
            
                    \draw[-latex] (V-2) .. controls (0.5, 0.75) and (1.5, 0.75) .. (V-1);
                    \draw[-latex] (V-6) .. controls (4.5, 0.75) and (5.5, 0.75) .. (V-4);
        
                    \draw[-latex] (V-5) .. controls (1.5, 0.75) and (2.5, 0.75) .. (V-3);
                    \draw[-latex] (V-8) .. controls (5.5, 0.75) and (6.5, 0.75) .. (V-7);
                    \draw[-latex] (V-5) .. controls (1.5, 1.33) and (5.5, 1.33) .. (V-4);
            \end{tikzpicture}\\\quad\\                    
            \begin{tikzpicture}[shorten >=1pt,-]
                \tikzstyle{vertex}=[circle,draw=black,fill=white!25,minimum size=20pt,inner sep=0pt]
                \tikzstyle{vertex2}=[circle,draw=black!,fill=black!10,minimum size=20pt,inner sep=0pt]
                \foreach \num/\x in {3/0, 7/3, 2/4, 6/7}
                        \node[vertex] (V-\num) at (\x,0) {$\num$};
                \foreach \num/\x in {1/1, 4/2, 5/5, 8/6}
                        \node[vertex2] (V-\num) at (\x,0) {$\num$};
            
                    \foreach \from/\to in {3/1, 7/2, 8/6}
                        \draw[-latex] (V-\from) -- (V-\to);
            
                    \draw[-latex] (V-4) .. controls (2.5, 0.75) and (3.5, 0.75) .. (V-2);
                    \draw[-latex] (V-7) .. controls (3.5, 0.75) and (4.5, 0.75) .. (V-5);
    
                    \draw[-latex] (V-3) .. controls (0.5, 1.1) and (3.5, 1.1) .. (V-2);
                    \draw[-latex] (V-7) .. controls (3.5, 1.1) and (6.5, 1.1) .. (V-6);
            \end{tikzpicture}
        \end{center}
        \caption{Tranzitivni orientaciji gosenice $C_4(1, 1, 1, 1)$ na osmih vozliščih. Gosenica $C_4(1, 1, 1, 1)$ je permutacijski graf permutacij $(2, 5, 1, 3, 6, 8, 4, 7)$ in $(3, 1, 4, 7, 2, 5, 8, 6)$. }
        \label{tranzitivni_orientaciji_gosenice}
    \end{figure}
    \begin{figure}[h]
        \begin{center}        
            % \begin{tikzpicture}[shorten >=1pt,-]
            %     \tikzstyle{vertex}=[circle,draw=black,fill=white!25,minimum size=20pt,inner sep=0pt]
            %     \tikzstyle{vertex2}=[circle,draw=black!50,fill=white!25,minimum size=20pt,inner sep=0pt]
            %     \foreach \num/\x in {5/1, 1/2, 8/5, 4/6}
            %             \node[vertex] (V-\num) at (\x,0) {$\num$};
            %     \foreach \num/\x in {2/0, 3/3, 6/4, 7/7}
            %             \node[vertex2] (V-\num) at (\x,0) {$\num$};
            
            %         \foreach \from/\to in {5/1, 8/4}
            %             \draw (V-\from) -- (V-\to);
            
            %         \draw (V-2) .. controls (0.5, 0.75) and (1.5, 0.75) .. (V-1);
            %         \draw (V-6) .. controls (4.5, 0.75) and (5.5, 0.75) .. (V-4);
        
            %         \draw (V-5) .. controls (1.5, 0.75) and (2.5, 0.75) .. (V-3);
            %         \draw (V-8) .. controls (5.5, 0.75) and (6.5, 0.75) .. (V-7);
            %         \draw (V-5) .. controls (1.5, 1.33) and (5.5, 1.33) .. (V-4);
            % \end{tikzpicture}\\\quad\\                    
            % \begin{tikzpicture}[shorten >=1pt,-]
            %     \tikzstyle{vertex}=[circle,draw=black,fill=white!25,minimum size=20pt,inner sep=0pt]
            %     \tikzstyle{vertex2}=[circle,draw=black!50,fill=white!25,minimum size=20pt,inner sep=0pt]
            %     \foreach \num/\x in {3/0, 7/3, 2/4, 6/7}
            %             \node[vertex] (V-\num) at (\x,0) {$\num$};
            %     \foreach \num/\x in {1/1, 4/2, 5/5, 8/6}
            %             \node[vertex2] (V-\num) at (\x,0) {$\num$};
            
            %         \foreach \from/\to in {3/1, 7/2, 8/6}
            %             \draw (V-\from) -- (V-\to);
            
            %         \draw (V-4) .. controls (2.5, 0.75) and (3.5, 0.75) .. (V-2);
            %         \draw (V-7) .. controls (3.5, 0.75) and (4.5, 0.75) .. (V-5);
    
            %         \draw (V-3) .. controls (0.5, 1.1) and (3.5, 1.1) .. (V-2);
            %         \draw (V-7) .. controls (3.5, 1.1) and (6.5, 1.1) .. (V-6);
            % \end{tikzpicture}\\\quad\\
            \begin{tikzpicture}[shorten >=1pt,-]
                \tikzstyle{vertex}=[circle,draw=black,fill=white!25,minimum size=20pt,inner sep=0pt]
                \tikzstyle{vertex2}=[circle,draw=black!,fill=black!10,minimum size=20pt,inner sep=0pt]
                \foreach \num/\x in {5/1, 1/2, 9/5, 4/6, 8/9}
                        \node[vertex] (V-\num) at (\x,0) {$\num$};
                \foreach \num/\x in {2/0, 3/3, 6/4, 7/7, 10/8}
                        \node[vertex2] (V-\num) at (\x,0) {$\num$};
            
                    \foreach \from/\to in {5/1, 9/4, 10/8}
                        \draw (V-\from) -- (V-\to);
            
                    \draw (V-2) .. controls (0.5, 0.75) and (1.5, 0.75) .. (V-1);
                    \draw (V-6) .. controls (4.5, 0.75) and (5.5, 0.75) .. (V-4);
        
                    \draw (V-5) .. controls (1.5, 0.75) and (2.5, 0.75) .. (V-3);
                    \draw (V-9) .. controls (5.5, 0.75) and (6.5, 0.75) .. (V-7);
    
                    \draw (V-5) .. controls (1.5, 1.33) and (5.5, 1.33) .. (V-4);
                    \draw (V-9) .. controls (5.5, 1.1) and (8.5, 1.1) .. (V-8);
            \end{tikzpicture}\\\quad\\                    
            \begin{tikzpicture}[shorten >=1pt,-]
                \tikzstyle{vertex}=[circle,draw=black,fill=white!25,minimum size=20pt,inner sep=0pt]
                \tikzstyle{vertex2}=[circle,draw=black!,fill=black!10,minimum size=20pt,inner sep=0pt]
                \foreach \num/\x in {3/0, 7/3, 2/4, 10/7, 6/8}
                        \node[vertex] (V-\num) at (\x,0) {$\num$};
                \foreach \num/\x in {1/1, 4/2, 5/5, 8/6, 9/9}
                        \node[vertex2] (V-\num) at (\x,0) {$\num$};
            
                    \foreach \from/\to in {3/1, 7/2, 10/6}
                        \draw (V-\from) -- (V-\to);
            
                    \draw (V-4) .. controls (2.5, 0.75) and (3.5, 0.75) .. (V-2);
                    \draw (V-7) .. controls (3.5, 0.75) and (4.5, 0.75) .. (V-5);        
                    \draw (V-8) .. controls (6.5, 0.75) and (7.5, 0.75) .. (V-6);
                    \draw (V-10) .. controls (7.5, 0.75) and (8.5, 0.75) .. (V-9);
    
                    \draw (V-3) .. controls (0.5, 1.1) and (3.5, 1.1) .. (V-2);
                    \draw (V-7) .. controls (3.5, 1.33) and (7.5, 1.33) .. (V-6);
            \end{tikzpicture}\\\quad\\
        \end{center}
        \caption{Gosenica $C_5(1, 1, 1, 1, 1)$ je permutacijski graf na desetih vozliščih permutacij $(2, 5, 1, 3, 6, 9, 4, 7, 10, 8)$ in $(3, 1, 4, 7, 2, 5, 8, 10, 6, 9)$.}
        \label{gosenica_dve_permutaciji1}
    \end{figure}
    \begin{figure}[h]
        \begin{center}
            \begin{tikzpicture}[shorten >=1pt,-]
                \tikzstyle{vertex}=[circle,draw=black,fill=white!25,minimum size=20pt,inner sep=0pt]
                \tikzstyle{vertex2}=[circle,draw=black!,fill=black!10,minimum size=20pt,inner sep=0pt]
                \foreach \num/\x in {5/1, 1/2, 9/5, 4/6, 12/9, 8/10}
                        \node[vertex] (V-\num) at (\x,0) {$\num$};
                \foreach \num/\x in {2/0, 3/3, 6/4, 7/7, 10/8, 11/11}
                        \node[vertex2] (V-\num) at (\x,0) {$\num$};
            
                    \foreach \from/\to in {5/1, 9/4, 12/8}
                        \draw (V-\from) -- (V-\to);
            
                    \draw (V-2) .. controls (0.5, 0.75) and (1.5, 0.75) .. (V-1);
                    \draw (V-6) .. controls (4.5, 0.75) and (5.5, 0.75) .. (V-4);
                    \draw (V-10) .. controls (8.5, 0.75) and (9.5, 0.75) .. (V-8);
    
                    \draw (V-5) .. controls (1.5, 0.75) and (2.5, 0.75) .. (V-3);
                    \draw (V-9) .. controls (5.5, 0.75) and (6.5, 0.75) .. (V-7);
                    \draw (V-12) .. controls (9.5, 0.75) and (10.5, 0.75) .. (V-11);
    
                    \draw (V-5) .. controls (1.5, 1.33) and (5.5, 1.33) .. (V-4);
                    \draw (V-9) .. controls (5.5, 1.33) and (9.5, 1.33) .. (V-8);
            \end{tikzpicture}\\\quad\\                    
            \begin{tikzpicture}[shorten >=1pt,-]
                \tikzstyle{vertex}=[circle,draw=black,fill=white!25,minimum size=20pt,inner sep=0pt]
                \tikzstyle{vertex2}=[circle,draw=black!,fill=black!10,minimum size=20pt,inner sep=0pt]
                \foreach \num/\x in {3/0, 7/3, 2/4, 11/7, 6/8, 10/11}
                        \node[vertex] (V-\num) at (\x,0) {$\num$};
                \foreach \num/\x in {1/1, 4/2, 5/5, 8/6, 9/9, 12/10}
                        \node[vertex2] (V-\num) at (\x,0) {$\num$};
            
                    \foreach \from/\to in {3/1, 7/2, 11/6, 12/10}
                        \draw (V-\from) -- (V-\to);
            
                    \draw (V-4) .. controls (2.5, 0.75) and (3.5, 0.75) .. (V-2);
                    \draw (V-7) .. controls (3.5, 0.75) and (4.5, 0.75) .. (V-5);        
                    \draw (V-8) .. controls (6.5, 0.75) and (7.5, 0.75) .. (V-6);
                    \draw (V-11) .. controls (7.5, 0.75) and (8.5, 0.75) .. (V-9);
    
                    \draw (V-3) .. controls (0.5, 1.1) and (3.5, 1.1) .. (V-2);
                    \draw (V-7) .. controls (3.5, 1.33) and (7.5, 1.33) .. (V-6);
                    \draw (V-11) .. controls (7.5, 1.1) and (10.5, 1.1) .. (V-10);
            \end{tikzpicture}
        \end{center}
        \caption{ Gosenica $C_6(1, 1, 1, 1, 1, 1)$ je permutacijski graf permutacij $(2, 5, 1, 3, 6, 9, 4, 7, 10, 12, 8, 11)$ in $(3, 1, 4, 7, 2, 5, 8, 11, 6, 9, 12, 10)$.}
        \label{gosenica_dve_permutaciji2}
    \end{figure}

    (1. način) Če je $k = 1$, priredimo števila $2, 3,\dots, m_1+1$ listom, ki so sosedi vozlišča $u_1$, in  število $1$ vozlišču $u_1$. Na ta način dobimo permutacijo $(2, 3,\dots, m_1+1, 1)$, katere permutacijski graf je gosenica $C_1(m_1)$ . Naj bo $k$ sodo število in si poglejmo, kako iz $C_k(m_1, m_2, \dots, m_k)$ konstruiramo $C_{k+1}(m_1, m_2, \dots, m_{k+1})$. Število $p$, ki je prirejeno vozlišču $u_k$, povečamo na $p+1$ in priredimo število $p$ vozlišču $u_{k+1}$. Števila $p+2, p+3, \dots, p+m_{k+1}+1$ priredimo listom, ki so sosedi vozlišča $u_{k+1}$. Se pravi, da smo permutacijo posodobili tako, da najprej na koncu zaporedja prejšnje permutacije priključimo nova števila v zaporedju $p+2, p+3, \dots, p+m_{k+1}+1, p+1$ in dobimo $(\dots, p, \dots, p+2, p+3, \dots, p+m_{k+1}+1, p+1)$, potem pa zamenjamo števili $p$ in $p+1$ in dobimo $(\dots, p+1, \dots, p+2, p+3, \dots, p+m_{k+1}+1, p)$. Naj bo sedaj $k$ liho število in si poglejmo, kako iz $C_k(m_1, m_2, \dots, m_k)$ konstruiramo $C_{k+1}(m_1, m_2, \dots m_{k+1})$. Naj bo $p_k = k + m_1 + m_2 + \cdot\cdot\cdot + m_k$. Priredimo števila $p_k+1, p_k+2, \dots, p_k+m_{k+1}$ listom, ki so sosedi vozlšča $u_{k+1}$. Število $p_k+m_{k+1}+1$ pa priredimo vozlišču $u_{k+1}$. Se pravi, da smo permutacijo posodobili tako, da najprej na koncu zaporedja prejšne permutacije priključimo nova števila v zaporedju $p_k + m_{k+1} + 1, p_k+1, p_k+2, \dots,p_k+m_{k+1}$ in dobimo $(\dots, p_k - m_k - 1, \dots, p_k + m_{k+1} + 1, p_k+1, p_k+2,\dots,p_k+m_{k+1})$, potem pa zamenjamo števili $p_k - m_k - 1$ in $p_k + m_{k+1} + 1$, ki pripadata vozliščema $u_k$ in $u_{k+1}$. 

    (2. način) Če je $k = 1$ priredimo število $m_1 + 1$ vozlišču $u_1$ in števila $1, 2, \dots, m_1$ listom, ki so sosedi vozlišča $u_1$. Na ta način dobimo permutacijo $(m_1+1, 1, 2, \dots, m_1)$. Naj bo $k$ sodo število in si poglejmo, kako iz $C_k(m_1, m_2, \dots, m_k)$ konstruiramo $C_{k+1}(m_1, m_2, \dots, m_{k+1})$. To naredimo na enak način kot pri 1. načinu v primeru, ko je bilo $k$ liho število. Naj bo $k$ liho število in si poglejmo, kako iz $C_k(m_1, m_2, \dots, m_k)$ konstruiramo $C_{k+1}(m_1, m_2, \dots, m_{k+1})$. To naredimo na enak način kot pri 1. načinu v primeru, ko je bilo $k$ sodo število. Tako smo pokazali, da je $C_k(m_1, m_2, \dots, m_k)$ permutacijski graf vsaj dveh permutacij.
\end{dokaz}

\begin{figure}[h]
    \begin{center}        
        \begin{tikzpicture}[shorten >=1pt,-]
            \tikzstyle{vertex}=[circle,draw=black!0,fill=white!25,minimum size=20pt,inner sep=0pt]
            
            \node[vertex] (V-0) at (3.5, 8) {$\sigma$};
            \node[vertex] (V-11) at (-1, 7) {$\sigma_1$};
            \node[vertex] (V-22) at (-1, 6) {$\sigma_2$};
            \node[vertex] (V-33) at (-1, 5) {$\sigma_3$};
            \node[vertex] (V-44) at (-1, 4) {$\sigma_4$};
            \node[vertex] (V-55) at (-1, 3) {$\sigma_5$};
            \node[vertex] (V-66) at (-1, 2) {$\sigma_6$};
            \node[vertex] (V-77) at (-1, 1) {$\sigma_7$};
            \node[vertex] (V-88) at (-1, -0.15) {};
            \node[vertex] (V-111) at (0.5, 7) {};
            \node[vertex] (V-222) at (0.5, 6) {};
            \node[vertex] (V-333) at (0.5, 5) {};
            \node[vertex] (V-444) at (0.5, 4) {};
            \node[vertex] (V-555) at (0.5, 3) {};
            \node[vertex] (V-666) at (0.5, 2) {};
            \node[vertex] (V-777) at (0.5, 1) {};

            \node[vertex] (V-12) at (3, 7) {$2$};
            \node[vertex] (V-23) at (2, 6) {$3$};
            \node[vertex] (V-24) at (3, 6) {$4$};
            \node[vertex] (V-31l) at (2, 5) {$1$};
            \node[vertex] (V-31d) at (3, 5) {$1$};
            \node[vertex] (V-45) at (2, 4) {$5$};
            \node[vertex] (V-43) at (1, 4) {$3$};
            \node[vertex] (V-46) at (3, 4) {$6$};
            \node[vertex] (V-53l) at (2, 3) {$3$};
            \node[vertex] (V-53d) at (3, 3) {$3$};
            \node[vertex] (V-68) at (3, 2) {$8$};
            \node[vertex] (V-67) at (2, 2) {$7$};
            \node[vertex] (V-65) at (1, 2) {$5$};
            \node[vertex] (V-75l) at (2, 1) {$5$};
            \node[vertex] (V-75d) at (3, 1) {$5$};
            \node[vertex] (V-80) at (3, 0) {};
            \node[vertex] (V-800) at (2, 0) {};
            \node[vertex] (V-8000) at (1, 0) {};

            \node[vertex] (V-13) at (4, 7) {$3$};
            \node[vertex] (V-21) at (4, 6) {$1$};
            \node[vertex] (V-35) at (4, 5) {$5$};
            \node[vertex] (V-34) at (5, 5) {$4$};
            \node[vertex] (V-32) at (6, 5) {$2$};
            \node[vertex] (V-42l) at (4, 4) {$2$};
            \node[vertex] (V-42d) at (5, 4) {$2$};
            \node[vertex] (V-57) at (4, 3) {$7$};
            \node[vertex] (V-56) at (5, 3) {$6$};
            \node[vertex] (V-54) at (6, 3) {$4$};
            \node[vertex] (V-64l) at (4, 2) {$4$};
            \node[vertex] (V-64d) at (5, 2) {$4$};
            \node[vertex] (V-70) at (4, 1) {};
            \node[vertex] (V-700) at (5, 1) {};
            \node[vertex] (V-7000) at (6, 1) {};
            

            \foreach \from/\to in {0/12, 0/13, 12/23, 12/24, 23/31l, 24/31d, 31d/43, 31d/45, 31d/46, 45/53l, 46/53d, 53d/65, 53d/67, 53d/68, 67/75l, 68/75d, 
            13/21, 21/35, 21/34, 21/32, 35/42l, 34/42d, 42l/56, 42l/57, 42l/54, 57/64l, 56/64d}
                \draw (V-\from) -- (V-\to);

            \draw[dotted] (V-75d) -- (V-80);
            \draw[dotted] (V-75d) -- (V-800);
            \draw[dotted] (V-75d) -- (V-8000);
            \draw[dotted] (V-64l) -- (V-70);
            \draw[dotted] (V-64l) -- (V-700);
            \draw[dotted] (V-64l) -- (V-7000);            
            \draw[dotted] (V-77) -- (V-88);

            \foreach \from/\to in {11/111, 22/222, 33/333, 44/444, 55/555, 66/666, 77/777}
                \draw[-latex] (V-\from) -- (V-\to);
            \end{tikzpicture}     
    \end{center}
    \caption{Slika ponazarja odločitveno drevo iz dokaza izreka \ref{izrek_dve_permutaciji_poti} za permutaciji $(2, 4, 1, 6, 3, 8, 5, \dots)$ in $(3, 1, 5, 2, 7, 4, \dots)$, katerih permutacijski graf je $P_n$.}
    \label{dve_permutaciji_poti}
\end{figure}

\begin{izrek}
\label{izrek_dve_permutaciji_poti}
    Za $n \geq 3$ obstajata natanko dve permutaciji iz $S_n$, katerih permutacijski graf je pot na $n$ vozliščih.
\end{izrek}
\begin{dokaz}
    Naj bo $n \geq 3$ in $\sigma = (\sigma_1, \sigma_2,\dots, \sigma_n) \in S_n$ takšna permutacija, da je $G_{\sigma} = P_n$. Opazimo naslednji lastnosti permutacije $\sigma$:
    \begin{enumerate}[label=(\alph*)]
        \item $(\sigma_1, \sigma_2, \dots, \sigma_k) \notin S_k$ za $k < n$.
        \item $i-2 \leq \sigma_i \leq i+2$ za $i \leq n$
    \end{enumerate}
    
    Dokaz lastnosti (a): Naj bo $k < n$. Če je $(\sigma_1, \sigma_2, \dots, \sigma_k) \in S_k$, potem graf $G_{\sigma}$ ni povezan. To je v nasprotju s tem, da je $G_{\sigma}$ pot.
    
    Dokaz lastnosti (b): Če je $\sigma_i \leq i-3$, potem je največ $i-4$ števil, ki so manjše od $\sigma_i$, (med $i-1$ števili) pred $\sigma_i$ v permutaciji $\sigma$. Zato so vsaj 3 števila pred $\sigma_i$ v permutaciji $\sigma$ večja od $\sigma_i$. To pomeni, da ima $\sigma_i$ vsaj $3$ sosede v grafu $G_{\sigma}$, kar pa je v protislovju s tem, da je $G_{\sigma}$ pot. Simetrično, če je $\sigma_i \geq i+3$, potem so za $\sigma_i$ v permutaciji $\sigma$ vsaj 3 števila, ki so manjša od $\sigma_i$. Kar zopet pomeni, da ima $\sigma_i$ vsaj $3$ sosede v grafu $G_{\sigma}$, kar pa je v protislovju s tem, da je $G_{\sigma}$ pot.


    Zaradi lastnosti (a) in (b) je $\sigma_1 \in \{ 2, 3\}$.
    Naslednji konstrukciji permutacije $\sigma$ sta predstavljeni na sliki \ref{dve_permutaciji_poti}. 
    Poglejmo si najprej primer, ko je $\sigma_1 = 2$. Zaradi lastnosti (a) in (b) je $\sigma_2 \in \{ 3, 4 \}$. Če je $\sigma_2 = 3$, potem graf $G_{\sigma}$ vsebuje pot na vozliščih $2, 3, 1$, kjer imata vozlišči $2$ in $3$ stopnjo enako $1$, kar se lahko zgodi samo, če je $n = 3$ in $\sigma = (2, 3, 1)$ (v nasprotnem primeru bi bil graf $G_{\sigma}$ nepovezan). Če je $\sigma_2 = 4$, potem mora biti $\sigma_3 = 1$ (drugače bi imelo vozlišče $1$ stopnjo vsaj $3$), se pravi $\sigma = (2, 4, 1, \dots)$. Če je $n = 4$, potem je $\sigma = (2, 4, 1, 3)$. Sicer je $n > 4$ in zaradi lastnosti (a) in (b) je $\sigma_4 \in \{ 5, 6 \}$. Če je $\sigma_4 = 5$, potem imata vozlišči $2$ in $5$ stopnjo enako $1$, kar se lahko zgodi samo ko je $n = 5$ in $\sigma = (2, 4, 1, 5, 3)$. Sicer je $\sigma_4 = 6$ in $\sigma_5 = 3$ (drugače bi imelo vozlišče $3$ stopnjo vsaj $3$), se pravi $\sigma = (2, 4, 1, 6, 3,\dots)$. Tako lahko nadaljujemo do poljubne dolžine. Induktivno lahko pokažemo, da je $\sigma$ enolično določena. Prvi element je $\sigma_1 = 2$, sodi elementi so $\sigma_{2i} = 2i + 2$, kjer je $i > 0$, lihi elementi so $\sigma_{2i + 1} = 2i - 1$, kjer je $i > 0$. Če je $n$ sodo število, potem je zadnji element $\sigma_n = n-1$. Če je $n$ liho število, potem je predzadnji element $\sigma_{n-1} = n$. 
    
    Poglejmo si sedaj še primer, ko je $\sigma_1 = 3$. Potem je $\sigma_2 \in \{ 1, 2, 4\}$. Element $\sigma_2 \neq 2$, saj bi tako v permutaciji imeli podzaporedje $3, 2, 1$, kar nam da cikel dolžine 3. Prav tako $\sigma_2 \neq 4$, saj bi tako graf vseboval povezave $32, 31, 42, 41$, kar nam da cikel dolžine 4. Zato je $\sigma_2 = 1$. Če je $n = 3$, potem je $\sigma = (3, 1, 2)$. Sicer je $n > 3$ in zaradi lastnosti (a) in (b) je $\sigma_3 \in \{ 4, 5 \}$. Če je $\sigma_3 = 4$, potem je $\sigma_4 = 2$ (drugače bi imelo vozlišče $2$ stopnjo vsaj $3$). Se pravi $n = 4$ in $\sigma = (3, 1, 4, 2)$. Če je $\sigma_3 = 5$, potem je $\sigma_4 = 2$ (drugače bi imelo vozlišče $2$ stopnjo vsaj $3$) in $\sigma = (3, 1, 5, 2,\dots)$. Podobno, kot prej lahko nadaljujemo do poljubne dolžine. Induktivno lahko pokažemo, da je $\sigma$ enolično določena. Drugi element je $\sigma_2 = 1$, sodi elementi so $\sigma_{2i} = 2i - 2$, kjer je $i > 1$, lihi elementi so $\sigma_{2i - 1} = 2i + 1$, kjer je $i > 0$. Če je $n$ sodo število, potem je predzadnji element $\sigma_{n-1} = n$. Če je $n$ liho število, potem je zadnji element $\sigma_n = n-1$. 

    Dobimo sklep, da obstajata natanko dve permutaciji iz $S_n$, katerih permutacijski graf je pot na $n$ vozliščih.
\end{dokaz}

\begin{posledica}
    Naj bo $\sigma \in S_n$ taka, da je njen permutacijski graf pot na $n$ vozliščih. Potem nam zaporedje vozlišč na poti da permutacijo $\sigma^*$, ki je sestavljena iz sosednjih transpozicij. Še več $\sigma^* \preceq_b \sigma$.
\end{posledica}
\begin{dokaz}
    Po dokazu izreka \ref{izrek_dve_permutaciji_poti} imata edini dve poti na $n$ vozliščih permutaciji $\sigma_1 = (2, 4, 1, 6, 3,\dots)$ in $\sigma_2 = (3, 1, 5, 2, 7, 4,\dots)$. Zaporedja vozlišč na poti sta $\sigma_1^* = (2, 1, 4, 3,\dots)$ in $\sigma_2^* = (1, 3, 2, 5, 4,\dots)$. Vidimo, da je permutacija $\sigma^*$ pridobljena iz permutacije $\sigma$ z zaporedjem transpozicij sosednjih elementov, kjer vsaka transpozicija odstrani eno inverzjo.
\end{dokaz}

\begin{izrek}    
    Naj bo $n \geq 3$ in $C$ gosenica na $n$ vozliščih. Potem obstajata natanko dve permutaciji iz $S_n$, katerih permutacijski graf je izomorfen grafu gosenice $C$.
\end{izrek}
\begin{dokaz}
    Privzamimo, da ima pot gosenice maksimalno dolžino (se pravi $m_1 = m_k = 0$). Naj bo $N$ število permutacij iz $S_n$, katerih permutacijski graf je izomorfen grafu $C$. Po trditvi \ref{trditev_gosenica_vsaj_dve_permutaciji} je $N \geq 2$. Naj bo $\pi$ permutacija, katere graf je izomorfen grafu $C$. Vidimo, da je $\pi$ unikatno določena z zaporedjem vozlišč na poti. To je res, saj morajo biti listi urejeni naraščajoče, da preprečimo dodatne inverzije v permutaciji. Po izreku \ref{izrek_dve_permutaciji_poti} obstajata natanko dve zaporedji vozlišč na poti. Torej je $N \leq 2$. Zato je $N = 2$.
\end{dokaz}

\section{ Konstrukcija permutacijskih grafov}

\begin{definicija}
    Naj bo $G$ graf z množico vozlišč $V(G) = \{x_1, x_2, \dots, x_n\}$ in naj bodo $H_1, H_2, \dots, H_n$ poljubni grafi. Kompozicija grafov $H_1, H_2, \dots, H_n$ z grafom $G$, označena z $G(H_1, H_2, \dots, H_n)$, je graf sestavljen iz disjunktne unije grafov $H_1, H_2, \dots, H_n$ in dodanih povezav $a_ib_j$, kjer je $a_i \in V(H_i)$ in $b_j \in V(H_j)$, kadar je $x_ix_j \in E(G)$ (sliki \ref{graf_kompozicija_primer1} in \ref{graf_kompozicija_primer2}). Če je $H_i$ fiksen graf $H$, potem kompozicijo označimo z $G(H)$. 
\end{definicija}

\begin{figure}[h]
    \begin{center}        
        \begin{tikzpicture}[shorten >=1pt,-]
            \tikzstyle{vertex}=[circle,draw=black,fill=white!25,minimum size=20pt,inner sep=0pt]
                    \node[vertex] (V-0) at (1, 0){$1_{P_2}$};
                    \node[vertex] (V-1) at (0, 1){$2_{P_2}$};
        
                    \node[vertex] (V-2) at (2, 3){$3_{P_3}$};
                    \node[vertex] (V-3) at (3, 3){$4_{P_3}$};
                    \node[vertex] (V-4) at (4, 3){$5_{P_3}$};
                    
                    \node[vertex] (V-5) at (5, 0){$6_{P_1}$};
                    
                \foreach \from/\to in {0/1, 2/3, 3/4, 0/2, 0/3, 0/4, 1/2, 1/3, 1/4, 2/5, 3/5, 4/5}
                    \draw (V-\from) -- (V-\to);
        \end{tikzpicture}
    \end{center}
    \caption{Graf $P_3(P_2, P_3, P_1)$.}
    \label{graf_kompozicija_primer1}
\end{figure}

Vsota grafov $L$ in $M$, označena z $L + M$, je sestavljena iz disjunktne unije grafov $L$ in $M$ ter dodanih povezav $ab$, kjer $a \in V(L)$ in $b \in V(M)$. Se pravi, kompozicija $G(H_1, H_2, \dots, H_n)$ je sestavljena iz disjunktne unije grafov $H_i$ in potem iz vsote $H_i + H_j$ za vsako pripadajočo povezavo $x_ix_j \in E(G)$.

\begin{izrek}
\label{izrek_konstrukcija_permutacijskega_grafa}
    Naj bo $G$ graf z $n$ vozlišči in naj bodo $H_1, H_2, \dots, H_n$ poljubni grafi. Potem je $G(H_1, H_2, \dots, H_n)$ permutacijski graf natanko tedaj, ko so $G, H_1, H_2, \dots, H_n$ permutacijski grafi.
\end{izrek}
\begin{dokaz}
    $(\Rightarrow)$ Privzamimo, da je $G(H_1, H_2, \dots, H_n)$ permutacijski graf. Ker so $H_1, H_2, \dots, H_n$ inducirani podgrafi grafa $G(H_1, H_2, \dots, H_n)$, so permutacijski grafi po izreku \ref{izrek_ekvivalentne_trditve_permutacijski_graf}. Prav tako lahko vzamemo eno vozlišče iz vsakega od grafov $H_i$ in ga označimo z $x_i$. Tako dobimo induciran podgraf izomorfen grafu $G$. To pomeni, da je tudi $G$ permutacijski. 
    
    $(\Leftarrow)$ Obratno privzamemo, da so $G, H_1, H_2, \dots, H_n$ permutacijski grafi. Naj bo $(v_1, v_2, \dots, v_n)$ kohezivno zaporedje grafa $G$. Naj bo $n_i$ število vozlišč grafa $H_j$, ki pripada vozlišču $v_i$. Potem ima graf $H_j$, ki pripada vozlišču $v_i$, kohezivno zaporedje $l_i = (x_1^i, x_2^i, \dots, x_{n_i}^i)$. Se pravi je 
    \[
        l = (l_1, l_2, \dots, l_n) = (x_1^1, x_2^1, \dots, x_{n_1}^1, x_1^2, x_2^2, \dots, x_{n_2}^2, \dots, x_1^n, x_2^n, \dots, x_{n_n}^n)
    \]
    kohezivno zaporedje grafa $G(H_1, H_2, \dots, H_n)$ in graf $G(H_1, H_2, \dots, H_n)$ je permutacijski graf.
\end{dokaz}

Izrek \ref{izrek_konstrukcija_permutacijskega_grafa} nam podaja enostaven način konstrukcije permutacijskih grafov. Naj bo $G$ polni graf $K_3$. Poglejmo si kompozicijo $K_3(K_2, K_3, K_1)$ (slika \ref{graf_kompozicija_primer2}). Ker so $K_3, K_2, K_1$ polni grafi dobimo polni graf $K_6$.

\begin{figure}[h]
    \begin{center}        
        \begin{tikzpicture}[shorten >=1pt,-]
            \tikzstyle{vertex}=[circle,draw=black,fill=white!25,minimum size=20pt,inner sep=0pt]
                    \node[vertex] (V-0) at (1.5, 0){$1_{K_2}$};
                    \node[vertex] (V-1) at (-0.5, 1){$2_{K_2}$};
        
                    \node[vertex] (V-2) at (2, 2){$3_{K_3}$};
                    \node[vertex] (V-3) at (3, 3){$4_{K_3}$};
                    \node[vertex] (V-4) at (4, 2){$5_{K_3}$};
                    
                    \node[vertex] (V-5) at (4.5, 0){$6_{K_1}$};
                    
                \foreach \from/\to in {0/1, 2/3, 3/4, 4/2, 0/2, 0/3, 0/4, 1/2, 1/3, 1/4, 2/5, 3/5, 4/5, 0/5, 1/5}
                    \draw (V-\from) -- (V-\to);
        \end{tikzpicture}
    \end{center}
    \caption{Graf $K_3(K_2, K_3, K_2)$.}
    \label{graf_kompozicija_primer2}
\end{figure}

Vsi grafi na največ $4$ vozliščih so permutacijski grafi. Zato sta grafa $P_3(P_2, P_3, P_1)$ in $K_3(K_2, K_3, K_1)$ permutacijska (sliki \ref{graf_kompozicija_primer1} in \ref{graf_kompozicija_primer2}).

% Vsak graf $G$ z $n$ vozlišči se lahko zapiše kot $G(\overset{n}{\overline{P_1, \dots, P_1}})$ in $K_1(G)$. Če sta to edina načina za zapis grafa $G$ kot kompozicija, potem je graf primaren.
Vsak graf $G$ z $n$ vozlišči se lahko zapiše kot $G(K_1, \dots, K_1)$ in $K_1(G)$.

\begin{definicija}
    Graf $G$ je primaren, če sta $G(K_1, \dots, K_1)$ in $K_1(G)$ edina načina za zapis grafa $G$ kot kompozicija. Graf $G$ je sestavljen, če ni primaren.
\end{definicija}

Med polnimi grafi sta primarna samo grafa $K_1$ in $K_2$. Graf $K_3$ ni primaren, saj je $K_3 = K_2(K_2, K_1)$. Vsak polni graf $K_n$, kjer je $n > 2$, vsebuje $K_3$ kot induciran podgraf. 

\begin{figure}[h]
    \begin{center}        
        \begin{tikzpicture}[shorten >=1pt,-]
            \tikzstyle{vertex}=[circle,draw=black,fill=white!25,minimum size=20pt,inner sep=0pt]
                \node[vertex] (V-1) at (0, 0.75){};
        
                \node[vertex] (V-2) at (1.5, 0){};
                \node[vertex] (V-3) at (1.5, 1.5){};
                
                \node[vertex] (V-4) at (3, 0.75){};
                \node[vertex] (V-5) at (4.5, 0){};
                \node[vertex] (V-6) at (4.5, 1.5){};
                
                \node[vertex] (V-7) at (6.5, 0){};
                \node[vertex] (V-8) at (6.5, 1.5){};
                \node[vertex] (V-9) at (8, 0.75){};
                \node[vertex] (V-10) at (9.5, 0.75){};
                \node[vertex] (V-11) at (11, 0){};
                \node[vertex] (V-12) at (11, 1.5){};
                    
                \foreach \from/\to in {2/3, 4/5, 4/6, 7/9, 8/9, 9/10, 10/11, 10/12}
                    \draw (V-\from) -- (V-\to);
        
                    
                \filldraw [black] (4.5,0.55) circle (1pt);
                \filldraw [black] (4.5,0.75) circle (1pt);
                \filldraw [black] (4.5,0.95) circle (1pt);

                \filldraw [black] (6.5,0.55) circle (1pt);
                \filldraw [black] (6.5,0.75) circle (1pt);
                \filldraw [black] (6.5,0.95) circle (1pt);
                \filldraw [black] (11,0.55) circle (1pt);
                \filldraw [black] (11,0.75) circle (1pt);
                \filldraw [black] (11,0.95) circle (1pt);
                
        \end{tikzpicture}
    \end{center}
    \caption{Grafi dreves s premerom $\leq 3$.}
    \label{graf_dreves_premer_leq3}
\end{figure}

\begin{figure}[h]
    \begin{center}        
        \begin{tikzpicture}[shorten >=1pt,-]
            \tikzstyle{vertex}=[circle,draw=black,fill=white!25,minimum size=20pt,inner sep=0pt]
                \node[vertex] (V-1) at (0, 0.75){};
                \node[vertex] (V-2) at (1.5, 1.5){};
                \node[vertex] (V-3) at (1.5, 0){};
        
                \foreach \from/\to in {1/2, 1/3}
                    \draw (V-\from) -- (V-\to);
        \end{tikzpicture}\qquad
        \begin{tikzpicture}[shorten >=1pt,-]
            \tikzstyle{vertex}=[circle,draw=black,fill=white!25,minimum size=20pt,inner sep=0pt]
                
                \node[vertex] (V-1) at (0, 0.75){};
                \node[vertex] (V-2) at (1.5, 0.75){};
                \node[vertex] (V-3) at (3, 0.75){};
                \node[vertex] (V-4) at (4.5, 0){};
                \node[vertex] (V-5) at (4.5, 1.5){};
                    
                \foreach \from/\to in {1/2, 2/3, 3/4, 3/5}
                    \draw (V-\from) -- (V-\to);            
        \end{tikzpicture}
    \end{center}
    \caption{Neprimarna grafa $P_3 = K_2(K_1, \overline{K_2})$ in $P_3(K_1, K_1, K_1, \overline{K_2})$.}
    \label{graf_P3_graf_kompozicije_P3_P1_P1_P1_K2c}
\end{figure}

Med drevesi s premerom, ki ni večji od 3 (slika \ref{graf_dreves_premer_leq3}), lahko pokažemo, da so primarni grafi samo poti $P_1, P_2$ in $P_4$. To so vse gosenice, ki nimajo dveh listov z istim sosednjim vozliščem. Graf $P_3$ ni primaren, saj ima dva lista, ki imata isto sosednje vozlišče. Poleg trivialnih kompozicij ima pot na treh vozliščih tudi kompozicijo $P_3 = K_2(K_1, \overline{K_2})$ (slika \ref{graf_P3_graf_kompozicije_P3_P1_P1_P1_K2c}). Prav tako nobena zvezda $K_{1,n}$ ni primarna, saj ima kompozicijo $K_{1,n} = K_2(K_1, \overline{K_n})$. Na podoben način se prepričamo tudi, da noben graf s premerom $3$, razen $P_4$, ni primaren. Nasledni izrek \ref{izrek_primarna_drevesa} nam karakterizira drevesa, ki so primarni grafi.

\begin{izrek}
\label{izrek_primarna_drevesa}
    Drevo je primaren permutacijski graf natanko tedaj, ko je gosenica brez dveh listov z istim sosednjim vozliščem.
\end{izrek}
\begin{dokaz}
    Ker smo že pogledali drevesa s premeri, ki ne presegajo $3$, privzamimo, da imamo drevo $T$ s premerom vsaj $4$. 
    
    $(\Rightarrow)$ Naj bo $T$ drevo z $n$ vozlišči. Privzemimo, da je $T$ primaren permutacijski graf. Po izreku \ref{izrek_gosenica_permutacijski_graf} je $T$ gosenica. Predpostavimo, da imamo dva lista $x_1$ in $x_2$ z istim sosedom $y$. Naj bo $G$ graf, ki ga dobimo, če identificiramo ti dve vozlišči ($x_1$ in $x_2$ zamenjamo z enim listom $y_1$, ki je povezan s sosedom vozlišč $x_1$ in $x_2$). Naj bodo $y_1, y_2, \dots, y_{n-1}$ vozlišča grafa $G$. Naj bo $H_1 = \overline{K_2}$ in $H_i$ trivialen graf za $i = 2, 3, \dots, n-1$. Potem je $T = G(H_1, H_2, \dots, H_n)$. To je v protislovju s tem, da je $T$ primaren. 
    
    $(\Leftarrow)$ Privzemimo zdaj, da je $T$ gosenica brez dveh listov z istim sosednim vozliščem, in predpostavimo, da $T$ ni primaren permutacijski graf. Potem je za nek netrivialen graf $G$ z vozlišči $y_1, y_2, \dots, y_k$ graf $T$ enak kompoziciji $T = G(H_1, H_2, \dots, H_k)$. Brez izgube splošnosti lahko privzamemo, da ima $H_1$ vsaj $2$ vozlišči. Ker je drevo $T$ povezan graf, mora biti tudi graf $G$ povezan. Zato mora $y_1$ imeti soseda. Privzemimo, da sta $y_1$ in $y_2$ sosednji vozlišči. Potem je $H_1 + H_2$ podgraf grafa $T$. Če bi $H_2$ imel vsaj $2$ vozlišči, bi graf $H_1 + H_2$ vseboval cikel, kar je v protislovju s tem, da je $T$ drevo. Torej ima $H_2$ samo eno vozlišče. Če bi $y_1$ imel še kakšnega soseda v grafu $G$, bi graf $T$ tako imel cikel dolžine $4$. Prav tako v $H_1$ ne sme biti povezav, saj bi tako podgraf $H_1 + H_2$ vseboval cikel dolžine $3$. Ampak potem so vsa vozlišča grafa $H_1$ listi grafa $T$ s skupnim sosedom, ki je edino vozlišče grafa $H_2$. To je v protislovju s predpostavko, da je $T$ gosenica brez dveh listov z istim sosedom. Torej je $T$ primaren permutacijski graf.
\end{dokaz}

\begin{izrek}
\label{izrek_neprimaren_graf_je_kompozicija_z_primarnim}
    Naj bo $G$ sestavljen (neprimaren) permutacijski graf. Potem obstajajo takšni netrivialen primaren permutacijski graf $U$ in permutacijski grafi $H_1, H_2, \dots, H_k$, ki so podgrafi grafa $G$, da je $G = U(H_1, H_2, \dots, H_k)$.
\end{izrek}
\begin{dokaz}
    Naj bo $G = U(H_1, H_2, \dots, H_k)$ (neprimaren) permutacijski graf, kjer je $U$ netrivialen. Če vzamemo eno vozlišče $x_i$ iz vsakega izmed $H_i$, potem je induciran podgraf izomorfen grafu $U$. Zato mora biti $U$ permutacijski graf po izreku \ref{izrek_ekvivalentne_trditve_permutacijski_graf}. Prav tako so grafi $H_i$ permutacijski, saj so inducirani podgrafi grafa $G$. Privzemimo, da ima $U$ najmanjše število vozlišč med vsemi takimi kompozicijami. Dokazali bi radi, da je $U$ primaren. Denimo, da $U$ ni primaren. Naj bo $U = V(L_1, L_2, \dots, L_p)$ kompozicija, kjer je $V$ netrivialen. Ker je $U$ kompozicija in vozlišča grafa $U$ predstavljajo inducirane podgrafe $H_i$ v grafu $G$, potem vsak $L_i$ predstavlja neko disjunktno unijo podmnožice grafov $\{ H_1, H_2, \dots, H_k \}$. Označimo to disjunktno unijo grafov z $A_i$. Graf $A_i$ je tako tudi induciran podgraf grafa $G$. Zato je $G = V(A_1, A_2, \dots, A_p)$. Ampak to predstavlja protislovje z izborom grafa $U$, saj smo našli graf $V$, ki ima manj vozlišč, kot graf $U$. Torej je $U$ primaren.
\end{dokaz}

Izrek \ref{izrek_neprimaren_graf_je_kompozicija_z_primarnim} nam opiše strukturo permutacijskih grafov. Vsak permutacijski graf je primaren ali neprimaren. Če je neprimaren je kompozicija permutacijskih grafov z primarnim permutacijskim grafom. Vidimo, da lahko vsak permutacijski graf izrazimo s kompozicijami in primarnimi grafi.


%%%%%%%%%%%%%%%%%%%%%%%%%%%%%%%%%%%%%%%%%
\chapter{ Tekmovalnostni grafi }

Tekmovalnostni graf je generiran z množico permutacij $R$. Ima množico vozlišč $[n]$ in povezave med vozlišči, ki v permutacijah iz $R$ zamenjata svoji relativni poziciji.

Tekmovalnostni grafi so razširitev permutacijskih grafov, saj je vsak tekmovalnostni graf, ki je generiran z dvema permutacijama $\pi$ in $\sigma$ izomorfen ravno permutacijskemu grafu permutacije $\pi^{-1} \circ \sigma$. 

V tem poglavju bomo spoznali kaj so rangiranja, tekmovalnostni grafi, primerljivostni grafi, množice tekmovalcev, množice posrednih in neposrednih tekmovalcev. Nato si bomo ogledali algoritem za izračun množice posrednih in neposrednih tekmovalcev ter kako lahko množice posrednih in neposrednih tekmovalcev uredimo. Na koncu si bomo pogledali še primer na resničnih podatkih.

\section{ Tekmovalnostni grafi ter množice posrednih in neposrednih tekmovalcev }

\begin{definicija}
    Rangiranje $c = (i_1, \dots, i_n)$ množice $[n]$ je permutacija iz $S_n$. Pisali bomo $i \prec_c j$, kadar se vozlišče $i$ pojavi pred vozliščem $j$ v vektorju rangiranja $c$, to je, ko $c^{-1}(i) < c^{-1}(j)$. Zato rangiranje $c$ definira zaporedje (urejenost) množice $[n]$: $i_1 \prec_c i_2 \prec_c \cdot\cdot\cdot \prec_c i_n$.
    
\end{definicija}

\begin{definicija}
\label{def_tekmovalnosti}
    Naj bo $R = \{c_1, c_2, \dots, c_r\}$ končna množica rangiranj. Potem rečemo, da par vozlišč $(i, j) \in [n] \times [n]$ (neposredno) tekmuje, če obstajata takšni rangiranji $c_s, c_t \in R$, da je $i \prec_{c_s} j$ ampak $j \prec_{c_t} i$ (slika \ref{tekmovalnost_ij}).
\end{definicija}

Če par vozlišč $(i, j)$ tekmuje, potem tekmuje tudi par vozlišč $(j, i)$.
Tekmovalnost med dvema vozliščema $i, j \in [n]$ je močno povezano z dejstvom, da je $(i, j)$ inverzija rangiranja množice. Spomnimo se, da je inverzija v rangiranju $c$ par vozlišč $(i, j)$ tako, da je $(i-j)(c^{-1}(i) - c^{-1}(j)) < 0$.

\begin{figure}[h]
    \begin{center}        
        \begin{tikzpicture}[shorten >=1pt,-]
            \tikzstyle{vertex}=[rectangle,draw=black!0,fill=white!25,minimum size=20pt,inner sep=0pt]
            
            \node[vertex] (V-cs) at (0, 1.1) {$c_s:$};
            \node[vertex] (V-ct) at (0, 0) {$c_t:$};

            \foreach \num/\x in {1/1, 2/2, i/3, j/4, n/5}
                \node[vertex] (V-\num) at (\x, 1.1) {$\num$};
            \foreach \num/\x in {1/1, 2/2, j/3, i/4, n/5}
                \node[vertex] (VV-\num) at (\x, 0) {$\num$};
        
            \foreach \from/\to in {1/1, 2/2, i/i, j/j, n/n}
                \draw[-latex] (V-\from) -- (VV-\to);

                
            \filldraw [black] (2.4, 0) circle (.2pt);
            \filldraw [black] (2.5, 0) circle (.2pt);
            \filldraw [black] (2.6, 0) circle (.2pt);
            \filldraw [black] (2.4, 1.1) circle (.2pt);
            \filldraw [black] (2.5, 1.1) circle (.2pt);
            \filldraw [black] (2.6, 1.1) circle (.2pt);
            \filldraw [black] (3.5, 0) circle (.2pt);
            \filldraw [black] (3.4, 0) circle (.2pt);
            \filldraw [black] (3.6, 0) circle (.2pt);
            \filldraw [black] (3.4, 1.1) circle (.2pt);
            \filldraw [black] (3.5, 1.1) circle (.2pt);
            \filldraw [black] (3.6, 1.1) circle (.2pt);
            \filldraw [black] (4.4, 0) circle (.2pt);
            \filldraw [black] (4.5, 0) circle (.2pt);
            \filldraw [black] (4.6, 0) circle (.2pt);
            \filldraw [black] (4.4, 1.1) circle (.2pt);
            \filldraw [black] (4.5, 1.1) circle (.2pt);
            \filldraw [black] (4.6, 1.1) circle (.2pt);
        \end{tikzpicture}     
    \end{center}
    \caption{Par vozlišč $(i, j)$ tekmuje.}
    \label{tekmovalnost_ij}
\end{figure}

\begin{figure}[h]
    \begin{center}        
        \begin{tikzpicture}[shorten >=1pt,-]
            \tikzstyle{vertex}=[circle,draw=black!0,fill=white!25,minimum size=20pt,inner sep=0pt]
            
            \node[vertex] (V-c1) at (0, 1) {$c_1:$};
            \node[vertex] (V-cs) at (0, 0) {$c_s:$};
            \foreach \num/\x in {i_1/1, i_2/2, i_3/3}
                \node[vertex] (V-\num) at (\x, 1) {$\num$};
            \foreach \num/\x in {i_2/1, i_3/2, i_1/3}
                \node[vertex] (VV-\num) at (\x, 0) {$\num$};
        
            \foreach \from/\to in {i_1/i_1, i_2/i_2, i_3/i_3}
                \draw[-latex] (V-\from) -- (VV-\to);

            \node[vertex] (V-c1-id) at (5, 1) {$c_1:$};
            \node[vertex] (V-cs-id) at (5, 0) {$c_s:$};
            \foreach \num/\x in {1/1, 2/2, 3/3}
                \node[vertex] (V-\num-id) at (5+\x, 1) {$\num$};
            \foreach \num/\x in {2/1, 3/2, 1/3}
                \node[vertex] (VV-\num-id) at (5+\x, 0) {$\num$};
        
            \foreach \from/\to in {1/1, 2/2, 3/3}
                \draw[-latex] (V-\from-id) -- (VV-\to-id);

        \end{tikzpicture}     
    \end{center}
    \caption{Preimenovanje vozlišč tako, da je $c_1 = id$.}
    \label{preimenovanje_vozlisc_id}
\end{figure}

\begin{lema}
    Če imamo podano končno množico $R = \{ c_1, c_2, \dots, c_r \}$ rangiranj, so naslednje trditve ekvivalentne:
    \begin{enumerate}[label=(\roman*)]
        \item Par vozlišč $(i, j)$ tekmuje.
        \item Obstaja tak $c_s \in \{ c_1, \dots, c_{r-1} \}$, da $i$ in $j$ zamenjata svoji relativni poziciji med rangiranji $c_s$ in $c_{s+1}$.
        \item Obstaja preimenovanje vozlišč tako, da je $c_1 = id$ (slika \ref{preimenovanje_vozlisc_id}) in nek $c_s \in \{c_2, \dots, c_r\}$ z inverzijo $(i, j)$.
    \end{enumerate}
\end{lema}
\begin{dokaz}
    $((ii) \Leftrightarrow (i))$ To sledi iz definicije \ref{def_tekmovalnosti}. 
    
    $((i) \Rightarrow (iii))$ Preimenujmo vozlišča tako, da bo $c_1 = id$. Naj $i$ in $j$ zamenjata svoji relativni poziciji med rangiranji $c_s$ in $c_t$. Potem je v enem izmed $c_s$ ali $c_t$ inverzija $(i, j)$. 
    
    $((iii) \Rightarrow (ii))$ Preimenujmo vozlišča tako, da je $c_1 = id$ in $(i, j)$ inverzija v $c_s$. Potem $i$ in $j$ zamenjata relativno pozicijo med $c_s$ in $c_{s-1}$ ali pa $c_{s-1}$ prav tako vsebuje inverzijo $(i, j)$ in se zamenjava zgodi prej. To sledi iz dejstva, da je $R$ končna množica.
\end{dokaz}

\begin{definicija}
    Naj bo $R = \{ c_1, c_2, \dots, c_r \}$ množica rangiranj množice $[n]$. Tekmovalnostni graf množice rangiranj $R$ definiramo kot neusmerjen graf $G_c(R) = ([n], E)$, kjer je množica povezav $E$ podana na nasledni način: med $i$ in $j$ je povezava, če $(i, j)$ tekmujeta.
\end{definicija}

\begin{primer}
\label{primer_tekmovalnostni_graf}
    Naj bo $R = \{ c_1, c_2, c_3 \}$ množica rangiranj množice $[4]$.
    \begin{align*}
        c_1 = (1, 2, 3, 4) \quad c_2 = (1, 2, 4, 3) \quad c_3 = (3, 1, 2, 4) \notag
    \end{align*}
    Ker je $c_1=id$, so povezave grafa $G_c(R)$ ravno inverzije rangiranj $c_2$ in $c_3$. Rangiranje $c_2$ ima inverzijo $(4, 3)$, medtem ko ima rangiranje $c_3$ inverziji $(3, 1), \ (3, 2)$. Graf $G_c(R)$ je prikazan na sliki \ref{graf_tekmovalnosti_primer}.
    \begin{figure}[h]
        \begin{center}
            \begin{tikzpicture}[shorten >=1pt,-]
                \tikzstyle{vertex}=[circle,draw=black,fill=white!25,minimum size=20pt,inner sep=0pt]
                \foreach \num/\x in {1/1, 2/3, 3/5, 4/7}
                        \node[vertex] (V-\num) at (\x,0) {\num};
            
                    \foreach \from/\to in {2/3, 3/4}
                        \draw (V-\from) -- (V-\to);
                    \draw (V-1) .. controls (1.5, 1) and (4.5, 1) .. (V-3);
            \end{tikzpicture}
        \end{center}
        \caption{Graf tekmovalnosti $G_c(R)$.}
        \label{graf_tekmovalnosti_primer}
    \end{figure}
\end{primer}


\begin{definicija}
    Če vzamemo množico rangiranj $R = \{ c_1, \dots, c_r \}$ množice $[n]$ in fiksiramo $i \in [n]$, je tekmovalnostna množica $C[i]$ vozlišča $i$ enaka množici elementov množice $[n]$, ki tekmuje z $i$ vključno z $i$ (to je zaprta soseščina $N_{G_c(R)}[i]$ vozlišča $i$ v grafu $G_c(R)$):
    \[
        C[i] = \{ j \in [n] \ | \ (i, j) \ tekmujeta \} \cup \{ i \} = N_{G_c(R)}[i].
    \]
\end{definicija}

\begin{primer}
    Naj bo $R$ tak kot v primeru \ref{primer_tekmovalnostni_graf}. Potem je:
    \[
        C[1] = \{ 1, 3 \}, \ C[2] = \{ 2, 3 \}, \ C[3] = \{ 1, 2, 3, 4 \}, \ C[4] = \{ 3, 4 \}.
    \]
\end{primer}

\begin{definicija}
    Naj bo $R = \{ c_1, c_2, \dots, c_r\}$ množica rangiranj množice $[n]$. Množici  vozlišč $C \subseteq [n]$ rečemo množica tekmovalcev, če vsaka dva elementa $i, j \in C$ tekmujeta in $C$ je maksimalna glede na to lastnost.
\end{definicija}

\begin{opomba}
    Množice tekmovalcev so ravno največji polni podgrafi grafa $G_c(R)$. Opazimo, da dve vozlišči tekmujeta natanko tedaj, ko pripadata isti množici tekmovalcev. Še več, lahko preverimo, da je množica vozlišč $C \subseteq [n]$ množica tekmovalcev natanko tedaj, ko je $C = \underset{i \in C}{\cap}C(i)$.
\end{opomba}

\begin{primer}
    Naj bo $R$ tak kot v primeru \ref{primer_tekmovalnostni_graf}. Največji polni podgraf grafa $G_c(R)$ je $K_2$, zato imajo množice tekmovalcev grafa $G_c(R)$ dva elementa. Množice tekmovalcev grafa $G_c(R)$ so:
    \[
        C_1 = \{ 1, 3 \}, \ C_2 = \{ 2, 3 \}, \ C_3 = \{ 3, 4 \}.
    \]
\end{primer}

\begin{definicija}
    Če vzamemo množico rangiranj $R = \{ c_1, \dots, c_r\}$ množice $[n]$, rečemo, da par vozlišč $(i, j) \in [n] \times [n]$ posredno ali neposredno tekmuje, če obstaja tak $k \in \mathbb{N}$ in vozlišča $i_1, \dots, i_k \in [n]$, da $(i, i_1)$ tekmujeta, $(i_1, i_2)$ tekmujeta, $\dots$, in $(i_k, j)$ tekmujeta.

    Množici vozlišč $D \subseteq [n]$ rečemo množica posrednih in neposrednih tekmovalcev, če vsaka dva elementa $i, j \in D$ posredno ali neposredno tekmujeta in $D$ je maksimalna glede na to lastnost.
\end{definicija}

\begin{opomba}
    Očitno je, da, če par vozlišč $(i, j)$ tekmuje, potem tudi posredno ali neposredno tekmuje. Še več par $(i, j)$ posredno ali neposredno tekmuje natanko tedaj, ko sta $i$ in $j$ povezana s potjo v grafu $G_c(R)$. Opazimo, da so množice posrednih ali neposrednih tekmovalcev iz $[n]$ povezane komponente grafa $G_c(R)$ in dve vozlišči posredno ali neposredno tekmujeta natanko tedaj, ko pripadata isti množici posrednih in neposrednih tekmovalcev. Seveda dve vozlišči, ki pripadata različnim množicam posrednih in neposrednih tekmovalcev, ne moreta tekmovati.
\end{opomba}

\begin{primer}
    Naj bo $R$ tak kot v primeru \ref{primer_tekmovalnostni_graf}. Graf $G_c(R)$ je povezan graf, zato vsi pari vozlišč posredno ali neposredno tekmujejo. Edina množica posrednih in neposrednih tekmovalcev grafa $G_c(R)$ je množica:
    \[
        D = \{ 1, 2, 3, 4 \}.
    \]
\end{primer}

\begin{definicija}
    Delno urejeni množici $(N, \preceq)$ lahko priredimo usmerjen graf $G_{\preceq}$ tako, da je množica vozlišč enaka $N$, vozlišči $i$ in $j$ pa sta povezani s povezavo $(i, j)$, če $i \neq j$ in $i \preceq j$.
    Graf $G = (N, E)$ delno urejene množice $N$ je primerljivostni graf, če je neusmerjen graf pridobljen po odstranitvi orientacije grafa $G_{\preceq}$ za neko delno urejenost $\preceq$  množice $N$.
\end{definicija}
Graf $G = (N, E)$ je primerljivosten natanko tedaj, ko dopušča tranzitivno orientacijo svojih povezav. To pomeni, da je usmerjen graf $\overset{\rightarrow}{G} = (N, \overset{\rightarrow}{E})$ pridobljen iz $G$ s takšno orientacijo vseh povezav v $E$, da če sta povezavi $(i, j), (j, k) \in \overset{\rightarrow}{E}$, potem je povezava $(i, k) \in \overset{\rightarrow}{E}$.

Permutacijski grafi so karakterizirani tudi s primerljivostnimi grafi. Graf $G$ je permutacijski graf natanko tedaj ko sta grafa $G$ in $\overline{G}$ primerljivostna grafa, to je, dovoljujeta tranzitivno orientacijo svojih povezav.

Opazimo, da so permutacijski grafi tako primerljivostni grafi kot tekmovalnostni. Namreč permutacijski graf permutacije $\sigma$ je ravno tekmovalnostni graf generiran z množico rangiranj $R = \{ c_1, c_2 \} = \{ id, \sigma\}$.

\begin{definicija}
    Graf $G$ ima delno kohezivno zaporedje vozlišč (ali enostavneje delno kohezivno zaporedje), če obstaja takšno preimenovanje vozlišč, da velja $(b)$ iz definicije \ref{def_kohezivno_zaporedje}, to je, če obstaja povezava $ab$, kjer $a < b$, potem mora za vsak $x$, za katerega velja $a < x < b$ obstajati povezava $ax$ ali $xb$. Graf $G$ je delno koheziven, če ima delno kohezivno zaporedje.
\end{definicija}

Medtem, ko je pogoj $(a)$ iz definicije \ref{def_kohezivno_zaporedje} povezan s primerljivostnimi grafi, je pogoj $(b)$ (delna kohezivnost) povezan s tekmovalnostnimi grafi, kot pokaže izrek \ref{tekmovalnosni_graf_delno_koheziven}.

\begin{izrek}
\label{tekmovalnosni_graf_delno_koheziven}
    Vsak tekmovalnostni graf je delno koheziven.
\end{izrek}
\begin{dokaz}
    Naj bo $G_c(R)$ tekmovalnostni graf, ki je generiran z množico rangiranj $R$. Brez izgube splošnosti privzemimo, da je rangiranje $id \in R$. Naj bo $ab \in E(G_c(R))$, kjer je $a \prec_{id} b$, in $x \in [n]$ tak, da je $a \prec_{id} x \prec_{id} b$. Ker je $ab$ povezava, vozlišči $(a, b)$ tekmujeta. To pomeni, da obstaja tako rangiranje $c_m \in R$, da je $b \prec_{c_m} a$. Če $x \prec_{c_m} a$, potem tekmujeta $(x, a)$ in je $ax \in E(G_c(R))$, v nasprotnem primeru je $b \prec_{c_m} a \prec_{c_m} x$, kar pomeni, da tekmujeta $(b, x)$ in $xb \in E(G_c(R))$.
\end{dokaz}

V \cite{setsOfRankings} avtorji domnevajo, da velja naslednja domneva.

\begin{domneva}
    Izrek \ref{tekmovalnosni_graf_delno_koheziven} je karakterizacija tekmovalnostnih grafov, to pomeni $G$ je tekmovalnostni graf natanko tedaj, ko ima delno kohezivno zaporedje.
\end{domneva}

\section{ Algoritem za izračun množice posrednih in neposrednih tekmovalcev }

\begin{lema}
\label{lema_algoritem_posredni_tekmovalci_1}
    Naj bo $R = \{ c_1, \dots, c_r \}$ množica rangiranj množice $[n]$. Če je $D \subseteq [n]$ množica posrednih in neposrednih tekmovalcev in $a, b \in D$, potem za vsak $x \in [n]$ in vsako tako rangiranje $c_m \in R$, da je $a \prec_{c_m} x \prec_{c_m} b$, sledi $x \in D$.
\end{lema}
\begin{dokaz}
    Če vozlišči $(a, b)$ tekmujeta, potem zaradi delne kohezivnosti tekmovalnostnega grafa tekmujeta tudi $(a, x)$ ali $(x, b)$, se pravi $x \in D$. Če vozlišči $(a, b)$ ne tekmujeta, potem obstaja tak $k \in \mathbb{N}$ in vozlišča $i_1, \dots, i_k \in [n]$, da $(a, i_1)$ tekmujeta, $(i_1, i_2)$ tekmujeta, $\dots$, in $(i_k, b)$ tekmujeta, saj sta $a,b \in D$. Če $a \prec_{c_m} x \prec_{c_m} i_1$, potem $x \in D$, ker $(a, i_1)$ tekmujeta. V nasprotnem primeru je $i_1 \prec_{c_m} x \prec_{c_m} b$. Če $i_1 \prec_{c_m} x \prec_{c_m} i_2$, potem $x \in D$, ker $(i_1, i_2)$ tekmujeta. V nasprotnem primeru je $i_2 \prec_{c_m} x \prec_{c_m} b$. Razmislek nadaljujemo, če $i_{k-1} \prec_{c_m} x \prec_{c_m} i_k$, potem $x \in D$, ker $(i_{k-1}, i_k)$ tekmujeta. V nasprotnem primeru je $i_k \prec_{c_m} x \prec_{c_m} b$. Ker $(i_k, b)$ tekmujeta, sledi $x \in D$.
\end{dokaz}

\begin{primer}
    Naj bo $R = \{ c_1, c_2, c_3\}$ množica rangiranj množice $[5]$.
    \begin{align}
        c_1 &= (1, 2, 3, 4, 5) \notag\\
        c_2 &= (2, 1, 3, 4, 5) \notag\\
        c_3 &= (1, 4, 2, 3, 5) \notag
    \end{align}
    Vidimo, da par $(4, 1)$ posredno tekmujeta, saj $(1, 2)$ in $(2, 4)$ tekmujeta, zato sta v isti množici posrednih in neposrednih tekmovalcev. Ker je $1 \prec_{c_1} 3 \prec_{c_1} 4$, je tudi $3$ v isti množici posrednih in neposrednih tekmovalcev. To je res, saj $2 \prec_{c_1} 3 \prec_{c_1} 4$ in $(2, 4)$ tekmujeta. Iz delne kohezivnost sledi, da tekmujeta tudi $(2, 3)$ ali $(3, 4)$. Par $(3, 4)$ res tekmuje, saj je $4 \prec_{c_3} 3$. Vidimo, da je $3$ res v isti množici posrednih in neposrednih tekmovalcev, kot sta $4$ in $1$.
\end{primer}

\begin{lema}
\label{lema_algoritem_posredni_tekmovalci_2}
    Naj bo $R = \{ c_1, \dots, c_r \}$ množica rangiranj množice $[n]$. Če je $D \subseteq [n]$ množica posrednih in neposrednih tekmovalcev ter obstajata taka $a \in D$ in  $c_m \in R$, da je $c_m^{-1}(a) = 1$, potem 
    \[
        \{ x \in [n] \ | \ c_s^{-1}(x) = 1 \ za \ nek \ c_s \in R \} \subseteq D.    
    \]
    To pomeni, da vsi elementi na prvi poziciji rangiranj iz $R$ pripadajo $D$.
\end{lema}
\begin{dokaz}
    Če $c_m \neq c_s$, $a \in D$, $a \neq x$ in $c_m^{-1}(a) = 1 = c_s^{-1}(x)$, potem je $a \prec_{c_m} x$ in $x \prec_{c_s} a$, se pravi $(a, x)$ tekmujeta in $x \in D$. 
\end{dokaz}

\begin{izrek}
    Naj bo $R = \{ c_1, \dots, c_r \}$ množica rangiranj vozlišč $[n]$. Množico posrednih in neposrednih tekmovalcev lahko identificiramo z zaprtimi intervali naravnih števil $[p, q]$ na naslednji način: 
    \[
        D_{[p, q]} = \{ x \in [n] \ | \ c_s^{-1}(x) \in [p, q] \ za \ nek \ c_s \in R\}.
    \]
    Še več, $p$ in $q$ sta prvi na levi in zadnji na desni poziciji elementov iz $D_{[p, q]}$ glede na vsa rangiranja.
\end{izrek}
\begin{dokaz}
    Pokazali bomo, da ima vsaka množica posrednih in neposrednih tekmovalcev obliko $D_{[p, q]}$ za neki naravni števili $p$ in $q$. Naj bo $a \in [n]$, $c_m \in R$ tako, da $c_m^{-1}(a) = 1$, in naj bo $D$ tista množica posrednih in neposrednih tekmovalcev, ki vsebuje $a$. Iz leme \ref{lema_algoritem_posredni_tekmovalci_2} sledi, da je 
    \[ 
        D_{[1,1]} = \{ x \in [n] \ | \ c_s^{-1}(x) = 1 \ za \ nek \ c_s \in R \} \subseteq D.
    \] 
    Definirajmo zaporedje $p_1\!=\!1, p_2, p_3, \dots$, kjer je $p_{k+1}$ zadnja pozicija (na desni) vseh elementov $D_{[1, p_k]}$ v vseh rangiranjih. 
    Pokazali bomo, da če 
    \[
        D_{[1, p_k]} = \{ x \in [n] \ | \ c_s^{-1}(x) \in [1, p_k] \ za \ nek \ c_s \in R\} \subseteq D,
    \]
    potem je tudi 
    \[
        D_{[1, p_{k+1}]} = \{ x \in [n] \ | \ c_s^{-1}(x) \in [1, p_{k+1}] \ za \ nek \ c_s \in R\} \subseteq D.
    \] 
    
    Naj bo $x \in D_{[1, p_{k+1}]}$, potem je $x \in [n]$ z $c_s^{-1}(x) \in [1, p_{k+1}]$ za nek $c_s$. Če je $c_s^{-1}(x) \in [1, p_k]$, sledi $x \in D_{[1, p_k]} \subseteq D$. V nasprotnem je $c_s^{-1} \in [p_k + 1, p_{k+1}]$. V tem primeru naj bo $b$ element $D_{[1,p_k]}$, ki se pojavi na poziciji $p_{k+1}$ v nekem rangiranju $c_{m_b}$, to pomeni $c_{m_b}^{-1}(b) = p_{k+1}$. Če $x \prec_{c_{m_b}} b$, potem je po lemi \ref{lema_algoritem_posredni_tekmovalci_1} element $x \in D$. V nasprotnem je $b \prec_{c_{m_b}} x$. Vsi elementi levo od $b$ v rangiranju $c_{m_b}$ pripadajo množici $D$ po lemi \ref{lema_algoritem_posredni_tekmovalci_1}. Naj bo teh elementov $t$. Če je $x \prec_{c_s} b$, potem $(c, b)$ tekmujeta in $x \in D$. Zato predpostavimo, da $b \prec_{c_s} x$. Na levi od $x$ v rangiranju $c_s$ je tako največ $t$ elementov, ampak en od njih je $b$, kar pomeni, da obstaja element $z$, za katerega velja $z \prec_{c_{m_b}} b \prec_{c_{m_b}} x$ in $b \prec_{c_s} x \prec_{c_s} z$. To pomeni, da $(x, z)$ tekmujeta, zato $x \in D$.

    Ker je $[n]$ končna množica in $D_{[1, p_m]} \subseteq [n]$, se veriga množic 
    \[
        D_{[1, 1]} \subseteq D_{[1, p_1]} \subseteq D_{[1, p_2]} \subseteq \cdot\cdot\cdot \subseteq D
    \]
    stabilizira za nek $D_{[1, p_m]} \subseteq D$. To pomeni, da je $D_{[1, p_m]} = D_{[1, p_{m+1}]}$. Še več $D \subseteq D_{[1, p_m]}$: po hipotezi je element $a \in D$, zato za vsak drug elemet $x \in D$ obstaja takšno končno število elementov $a_1, a_2, \dots, a_k$, da pari $(a, a_1),(a_1, a_2), \dots, (a_k, x)$ tekmujejo. Zaradi dejstva, da je $a \in D_{[1, 1]}$ in $(a, a_1)$ tekmujeta, dobimo, da je $a_1 \in D_{[1, p_1]}$, podobno ker $a_1 \in D_{[1, p_1]}$ in $(a_1, a_2)$ tekmujeta, dobimo, da je $a_2 \in D_{[1, p_2]}$, $\dots$, in ker $a_{k} \in D_{[1, p_k]}$ in $(a_k, x)$ tekmujeta, dobimo, da je $x \in D_{[1, p_{k+1}]} \subseteq D_{[1, p_m]}$. Zato je $D \subseteq D_{[1, p_m]}$ in $D = D_{[1, p_m]}$.

    Izbrišimo elemente iz $[n]$, ki se pojavijo v množici posrednih in neposrednih tekmovalcev $D = D_{[1, p_m]}$, in ponovimo postopek, da odkrijemo ostale množice posrednih in neposrednih tekmovalcev.
\end{dokaz}

\begin{primer}
     Naj bo $R = \{ c_1, c_2, c_3\}$ množica rangiranj množice $[5]$.
    \begin{align}
        c_1 &= (1, 2, 3, 4, 5) \notag\\
        c_2 &= (2, 1, 3, 4, 5) \notag\\
        c_3 &= (1, 4, 2, 3, 5) \notag
    \end{align}
    Poiščimo množice posrednih in neposrednih tekmovalcev. Najprej si poglejmo množico posrednih in neposrednih tekmovalcev $D_1$, ki vsebuje elemente, ki se v vsaj enem rangiranju pojavijo na prvem mestu:
    \[
        D_{[1,1]} = \{ x \in [n] \ | \ c_s^{-1}(x) = 1 \ za \ nek \ c_s \in R \} = \{ 1, 2 \} \subseteq D_1.
    \]
    Elementa $1$ in $2$ se v rangiranjih nahajata na $1., 2.$ in $3.$ mestu. Ker je $max\{ 1, 2, 3\} = 3$, si sedaj poglejmo $D_{[1, 3]}$:
    \[
        D_{[1,1]} \subset D_{[1,3]} = \{ x \in [n] \ | \ c_s^{-1}(x) \in [1, 3] \ za \ nek \ c_s \in R \} = \{ 1, 2, 3, 4\} \subseteq D_1.
    \]
    Elementi $1, 2, 3$ in $4$ se v rangiranjih nahajajo na $1., 2., 3.$ in $4.$ mestu. Ker je $max\{ 1, 2, 3, 4\} = 4$, si sedaj poglejmo $D_{[1, 4]}$:
    \[
        D_{[1,3]} = D_{[1,4]} = \{ x \in [n] \ | \ c_s^{-1}(x) \in [1, 4] \ za \ nek \ c_s \in R \} = \{ 1, 2, 3, 4\} = D_1.
    \]
    Vidimo, da se je veriga stabilizirala in je $D_1 = D_{[1, 4]} = \{ 1, 2, 3, 4\}$. Sedaj si poglejmo množico posrednih in neposrednih tekmovalcev $D_2$, ki vsebuje vse elemente, ki se v vsaj enem rangiranju pojavijo na $5.$ mestu ($5 = 4 + 1$):
    \[
        D_{[5,5]} = \{ x \in [n] \ | \ c_s^{-1}(x) = 5 \ za \ nek \ c_s \in R \} = \{ 5 \} \subseteq D_2.
    \]
    Element $5$ se v rangiranjih vedno nahaja na $5.$ mestu. Ker $max\{ 5 \} = 5$, se nam zgorna meja ne poveča. Zato je $D_2 = D_{[5, 5]} = \{ 5 \}$.
\end{primer}

Dokaz zadnjega izreka nam podaja algoritem za izračun množice posrednih in neposrednih tekmovalcev direktno iz množice rangiranj brez predhodnega izračuna tekmovalnostnega grafa.

\begin{algorithm}[H]
    \SetKwInOut{Input}{input}\SetKwInOut{Output}{output}

    \BlankLine
    \KwIn{
        \\        
        $N = \{1, \dots, n \}$ končna množica vozlišč\\
        $R = \{c_1, \dots, c_r \}$ končna množica rangiranj\\
    }
    \BlankLine

    \Begin{
        $j:= 1$\;
        $p_0 := 0$\;
        $p_j := 1$\;
        \While{$|N| > 0$}{            
            $D_j := \emptyset$\;
            $q_0 := p_{j-1}$\;
            $q_1 := p_j$\;
            $i := 0$\;
            \While{ $q_i \neq q_{i+1}$ }{
                $i := i + 1$\;
                $D_j := D_{[p_j, q_i]}$\;
                $q_{i+1} := \underset{x \in D_j, c \in R}{max} c^{-1}(x)$\;
            }
            $N := N \setminus D_j$\;
            $j := j + 1$\;
            $p_j := q_{i+1} + 1$\;
        }   
    }
    \BlankLine
    \KwOut{
        \\
        Disjunktne množice posrednih in neposrednih tekmovalcev $D_1, \dots, D_k$, kjer je $\underset{i \in [k]}{\cup}D_i = [n]$
    }
    \BlankLine
    \caption{Psevdo koda algoritma za izračun množic 
    posrednih in neposrednih tekmovalcev}
\end{algorithm}
\newpage 

% \begin{minipage}{\linewidth}    
% \begin{lstlisting} [language=Matlab, numbers=left, stepnumber=1, numbersep=10pt, tabsize=4, showspaces=false, showstringspaces=false]
%     Psevdo koda algoritma za izračun množic 
%     posrednih in neposrednih tekmovalcev:

%     Vhod:
%     $N = \{1, \dots, n \}$ končna množica vozlišč
%     $R = \{c_1, \dots, c_r \}$ končna množica rangiranj
    
%     begin
%         $j:= 1$;
%         $p_0 := 0$;
%         $p_j := 1$;
%         while $|N| > 0$ do
%             $D_j := \emptyset$;
%             $q_0 := p_{j-1}$;
%             $q_1 := p_j$;
%             $i := 0$;
%             while $q_i \neq q_{i+1}$ do
%                 $i := i + 1$;
%                 Construct $D_j := D_{[p_j, q_i]}$;
%                 $q_{i+1} := \underset{x \in D_j, c \in R}{max} c^{-1}(x)$;
%             end
%             $N := N \setminus D_j$;
%             $j := j + 1$;
%             $p_j := q_i + 1$;
%         end
%     end
    
%     Izhod: 
%     Množice posrednih in neposrednih 
%     tekmovalcev $D_1, \dots, D_k$
% \end{lstlisting}
% \end{minipage}

\begin{definicija}
\label{definicija_gdr}
    Naj bo $R = \{ c_1, \dots, c_r \}$ množica $r$ rangiranj $(r \geq 2)$ množice $[n]$. Definirajmo usmerjen graf $G_d(R)$  na naslednji način:
    \begin{enumerate}[label=(\roman*)]
        \item Vozlišča grafa $G_d(R)$ so elementi množice $[n]$.
        \item Če $i, j \in [n]$, $i \neq j$ potem je $(i, j)$ usmerjena povezava v grafu $G_d(R)$, če obstaja takšno rangiranje $c_m \in R$, da je $i \preceq_{c_m} j$.
    \end{enumerate}
\end{definicija}

\begin{opomba}
    Opazimo, da je usmerjen graf $G_d(R)$ izomorfen usmerjenemu grafu $G_{\preceq}$, ki ga definiramo z (refleksivno in antisimetrično) relacijo $\preceq$ podano z:
    \begin{enumerate}[label=(\roman*)]
        \item $i \preceq i$ za vsak $i \in [n]$
        \item $i \preceq j$ $(i,j \in [n], i \neq j)$, če obstaja takšno rangiranje $c_m \in R$, da je $i \preceq_{c_m} j$.
    \end{enumerate}
    Tekmovalnostni graf $G_c(R)$ je izomorfen neusmerjenemu grafu z enakimi vozlišči kot graf $G_d(R)$ in neusmerjenimi povezavami $ij \in E(G_c(R))$, kadar sta usmerjeni povezavi $(i, j), (j, i) \in E(G_d(R))$.
\end{opomba}

\begin{trditev}
\label{trditev_ekvivalenc_mnozic_posrednih_tekmovalcev}
    Naj bosta $D_1$ in $D_2$ dve različni množici posrednih in neposrednih tekmovalcev. Naslednji trditvi o usmerjenem grafu $G_d(R)$ sta ekvivalentni:
    \begin{enumerate}[label=(\roman*)]
        \item Obstaja takšna usmerjena povezava $(a, b)$, da je $a \in D_1$ in $b \in D_2$.
        \item Vsa vozlišča iz $D_1$ imajo usmerjeno povezavo proti vsem vozliščem iz $D_2$.
    \end{enumerate}
\end{trditev}
\begin{dokaz}
    Trditev $(i)$ je posebni primer trditve $(ii)$, zato iz $(ii)$ sledi $(i)$. Pokažimo sedaj, da iz $(i)$ sledi $(ii)$. Najprej pokažimo naslednji trditvi.

    \begin{enumerate}
        \item Pokazali bomo, da, če je $a \in D_1$, $b_1, b_2 \in D_2$, par $(b_1, b_2)$ tekmuje in obstaja usmerjena povezava od $a$ do $b_1$, potem obstaja usmerjena povezava od $a$ do $b_2$. Po hipotezi obstaja takšno rangiranje $c_m$, da je $a \prec_{c_m} b_1$. Če $a \prec_{c_m} b_2$, potem trditev velja. V nasprotnem primeru je $b_2 \prec_{c_m} a \prec_{c_m} b_1$. Ampak, ker $(b_1, b_2)$ tekmujeta, obstaja takšno rangiranje $c_{m'}$, da $b_1 \prec_{c_{m'}} b_2$, in ker $a$ ne tekmuje z $b_1$, mora biti $a \prec_{c_{m'}} b_1 \prec_{c_{m'}} b_2$, kar pomeni, da $(a, b_2)$ tekmujeta. To je protislovje in zato $a \prec_{c_m} b_2$.
        \item Pokazali bomo, da, če je $a \in D_1$, $b \in D_2$ in obstaja usmerjena povezava od $a$ proti $b$, potem za vsak $b' \in D_2$ obstaja usmerjena povezava od $a$ do $b'$. Ker sta $b, b' \in D_2$, obstaja tak $k \in \mathbb{N}$ in $b_1, \dots, b_k \in D_2$, da $(b, b_1)$ tekmujeta, $(b_1, b_2)$ tekmujeta, $\dots$, $(b_k, b')$ tekmujeta. Vozlišča $a, b, b_1$ so v takem razmerju kot v koraku $1.$, zato obstaja usmerjena povezava od $a$ do $b_1$, podobno vozlišča $a, b_1, b_2$, zato obstaja usmerjena povezava $a$ do $b_2$, $\dots$, podobno vozlišča $a, b_k, b'$, zato obstaja usmerjena povezava od $a$ do $b'$.
    \end{enumerate}
        
    Predpostavimo sedaj trditev $(i)$, torej, da obstaja usmerjena povezava od $a \in D_1$ do $b \in D_2$. Po trditvi $2$ obstaja usmerjena povezava od $a$ do vseh elementov v $D_2$. 
    Naj bo $a' \in D_1$ in $a \neq a'$. Elementa $a$ in $a'$ posredno ali neposredno tekmujeta. Če (neposredno) tekmujeta, potem po podobnem premisleku, kot v trditvi $1$, obstaja usmerjena povezava od $a'$ do $b$. Če posredno tekmujeta, potem po podobnem premisleku, kot v trditvi $2$, obstaja usmerjena povezava od $a'$ do $b$. Tako dobimo, da obstaja povezava od vsakega elementa iz $D_1$ do vsakega elementa iz $D_2$.
\end{dokaz}

\begin{definicija}
\label{definicija_relacije_mnozic_posrednih_tekmovalcev}
    Naj bo $R = \{ c_1, \dots, c_r \}$ množica r rangiranj vozlišč $[n]$, $r \geq 2$, katerih množice posrednih in neposrednih tekmovalcev označimo z $D_1, \dots, D_k$, kjer je$\underset{i \in [k]}{\cup} D_i = [n]$. Definirajmo binarno relacijo $\rightarrow$ med dvema množicama posrednih in neposrednih tekmovalcev na nasledni način:
    \begin{enumerate}[label=(\roman*)]
        \item $D_i \rightarrow D_i$ za vsako množico posrednih in neposrednih tekmovalcev $D_i$.
        \item za vsaki različni množici $D_i, D_j$ posrednih in neposrednih tekmovalcev, je $D_i \rightarrow D_j$ natanko tedaj, ko velja katerakoli od trditev iz \ref{trditev_ekvivalenc_mnozic_posrednih_tekmovalcev}.
    \end{enumerate}
\end{definicija}

\begin{lema}
    Binarna relacija $\rightarrow$ je tranzitivna.
\end{lema}
\begin{dokaz}
    Predpostavimo, da je $D_1 \rightarrow D_2$ in $D_2 \rightarrow D_3$. Radi bi pokazali, da je potem $D_1 \rightarrow D_3$. Predpostavimo, da je $D_3 \rightarrow D_1$ in pokažimo, da nas to privede do protislovja. Vzamimo vozlišče $x \in D_1$. Ker je $D_3 \rightarrow D_1$, obstaja takšno rangiranje $c_m$ tako, da je $a \prec_{c_m} x$ za vse $a \in D_3$. Še več, ker $D_1 \rightarrow D_2$, $x \prec_{c_m} b$ za vsak $b \in D_2$ in zato $a \prec_{c_m} b$ za vse $a \in D_3$ in $b \in D_2$, kar pomeni $D_3 \rightarrow D_2$. To je protislovje. Iz definicije \ref{definicija_gdr} vidimo, da ima graf $G_d(R)$ usmerjeno povezavo med vsakim parom različnih vozlišč. Zato je $D_1 \rightarrow D_3$.
\end{dokaz}

\begin{posledica}
    Binarna relacija $\rightarrow$ je linearna urejenost množic posrednih in neposrednih tekmovalcev iz $[n]$.
\end{posledica}


\section{Uporaba algoritma na resničnih podatkih}

Poglejmo si sezono 2014 v prvenstvu MotoGP. V tej sezoni je bil najboljši dirkač Marc Márquez. Zmagal je na prvih desetih dirkah sezone. V celi sezoni pa je zmagal na trinajstih od skupaj osemnajst dirk. Na treh dirkah je padel, vendar se je v trenutku padca potegoval za zmago. Poleg tega je bil še enkrat drugi in enkrat četrti. Poleg Marca Márqueza so bili veliko boljši od ostalih še Valentino Rossi, Jorge Lorenzo in Dani Pedrosa. Na stopničkah so bili trije od njih (štirih) na trinajstih dirkah, vsaj dva na sedemnajstih dirkah, vsaj en pa na vseh osemnajstih dirkah te sezone. Opazimo, da so se to sezono izoblikovale vsaj tri kakovostne skupine. V prvi skupini je Marc Márquez, ki je bil to sezono veliko boljši od ostalih. V drugi skupini so Valentino Rossi, Jorge Lorenzo in Dani Pedrosa. V ostalih skupinah pa so ostali dirkači.

Če bi uporabili algoritem za izračun množic posrednih in neposrednih tekmovalcev na rezultatih vseh dirk sezone 2014, bi naleteli na težave. Ena od težav je, da ni na vseh dirkah tekmovalo enako število tekmovalcev, oziroma ni vsake dirke zaključilo enako število tekmovalcev. Zato rezultati dirk niso iz iste simetrične grupe $S_n$. Poleg tega opazimo, da je Marc Márquez bil na prvem in na zadnjem mestu (je odstopil), iz česar sledi, da tekmuje z vsemi ostalimi dirkači in imamo samo eno množico posrednih in neposrednih tekmovalcev.

Zato bomo izbrali neko podmnožico dirk $A$ in neko podmnožico dirkačev, ki so na vseh dirkah iz podmnožice $A$ dirko zaključili. Tako dobimo $|A|$ rangiranj/permutacij neke simetrične grupe.

Izberimo dirke v Argentini ($c_{arg}$), Španiji ($c_{esp}$), Kataloniji ($c_{cat}$), Nemčiji ($c_{ger}$) in Veliki Britaniji ($c_{gbr}$). Izberimo še vse dirkače, ki so na teh dirkah zaključili dirko. Teh dirkačev je $14$. Uredimo jih relativno glede na to, kako so bili na koncu sezone uvrščeni v skupnem vrstem redu. Tako dobimo dobimo vektor = (1, 2, 3, 4, 5, 6, 7, 8, 9, 10, 11, 12, 13, 14) = (Marc Márquez, Valentino Rossi, Jorge Lorenzo, Dani Pedrosa, Andrea Dovizioso, Pol Espargaró, Aleix Espargaró, Bradley Smith, Stefan Bradl, Scott Redding, Hiroshi Aoyama, Yonny Hernández, Héctor Barberá, Broc Parkes). Sedaj si oglejmo rangiranja teh dirkačev na dirkah iz $R = \{ c_{arg}, c_{esp}, c_{cat}, c_{ger}, c_{gbr} \}$:
\begin{align}
    c_{arg} &= (1, 4, 3, 2, 9, 8, 6, 5, 11, 12, 10, 7, 13, 14) \notag\\
    c_{esp} &= (1, 2, 4, 3, 5, 7, 8, 6, 9, 11, 10, 12, 13, 14) \notag\\
    c_{cat} &= (1, 2, 4, 3, 9, 7, 6, 5, 8, 12, 10, 11, 14, 13) \notag\\
    c_{ger} &= (1, 4, 3, 2, 7, 6, 5, 10, 11, 9, 12, 13, 8, 14) \notag\\
    c_{gbr} &= (1, 3, 2, 4, 5, 6, 9, 7, 10, 12, 11, 13, 14, 8) \notag
\end{align}
To pomeni, da je bil na dirki v Argentini ($c_{arg}$) prvi Marc Márquez, drugi Dani Pedrosa, tretji Jorge Lorenzo, četrti Valentino Rossi,...

Sedaj poženemo algoritem za izračun množice posrednih in neposrednih tekmovalcev in dobimo naslednje množice posrednih in neposrednih tekmovalcev:
\begin{align}
    D_1 &= \{ 1 \},  \notag\\
    D_2 &= \{ 2, 3, 4 \}, \notag\\
    D_3 &= \{ 5, 6, 7, 8, 9, 10, 11, 12, 13, 14 \}. \notag
\end{align}
Tako vidimo, da so za to izbiro dirk in dirkačev, ki so zaključili te dirke, dobimo pričakovane ugotovitve. Torej je Marc Márquez v svoji množici posrednih in neposrednih tekmovalcev, v drugi množici posrednih in neposrednih tekmovalcev so Valentino Rossi, Jorge Lorenzo in Dani Pedrosa. Ostali dirkači pa so v tretji množici posrednih in neposrednih tekmovalcev.

Poglejmo si še en pristop. Dirkače označimo z zaporedno številko glede na to, kako so bili na koncu sezone uvrščeni v skupnem vrstem redu. Tako dobimo dobimo vektor = (1, 2, 3, 4, 5, 6, 7, 8, 9, 10, 11, 12, 13, 14, 15, 16, 17, 18, 19, 20, 21, 22, 23, 24, 25, 26, 27, 28, 29) = (Marc Márquez, Valentino Rossi, Jorge Lorenzo, Dani Pedrosa, Andrea Dovizioso, Pol Espargaró, Aleix Espargaró, Bradley Smith, Stefan Bradl, Andrea Iannone, Álvaro Bautista, Scott Redding, Cal Crutchlow, Hiroshi Aoyama, Yonny Hernández, Nicky Hayden, Karel Abraham, Héctor Barberá, Michele Pirro, Danilo Petrucci, Alex de Angelis, Colin Edwards, Broc Parkes, Michael Laverty, Mike Di Meglio, Katsuyuki Nakasuga, Leon Camier, Michel Fabrizio, Randy de Puniet). Tokrat si za podmnožice dirk izbirajmo po 3 zaporedne dirke in glejmo samo dirkače, ki so na vseh teh 3 dirkah zaključili dirko. Množice posrednih in neposrednih tekmovalcev, ki jih dobimo z algoritmom za izračun množic posrednih in neposrednih tekmovalcev, so zbrane v tabeli \ref{tbl:tabelaPosrednihNeposrednihTekmovalcev3}. Ponovno vidimo, da se Marc Márquez velikokrat pojavi v prvi množici posrednih in neposrednih tekmovalcev, kot edini tekmovalec te množice. Dvakrat se nam kot samostojna množica pojavi množica najboljših štirih te sezone. Množice $\{ 24 \}$, $\{ 25 \}$, $\{ 18 \}$, $\{ 23 \}$ nam pokažejo, da nekateri dirkači to sezono niso bili prav zares konkurenčni ostalim dirkačem. Vidimo tudi, da se nam pri skoraj vseh izbirah treh zaporednih dirk ustvari nekaj množic posrednih in neposrednih tekmovalcev. To se ne zgodi samo takrat, ko je Marc Márquez padel, vendar nadaljeval dirko in jo zaključil na 15 oziroma 13 mestu (dirki trinajst in štirinajst). To nam pove, da so dirkači v različnih množicah verjetno res različno konkurenčni. Na to seveda vpliva tudi to, da dirkajo z opremo različnih kvalitet. 

% \begin{table}
%     \begin{center}
%         \begin{tabular}{ |c|c| } 
%         \hline
%             k=2 & Množice posrednih in neposrednih tekmovalcev \\ 
%         \hline
%             [1,2] & $\{ 1 \}, \{ 2, 4, 5, 7, 10, 16 \}, \{ 14 \}, \{ 15 \}, \{ 17 \}, \{ 20 \}, \{ 24 \}, \{ 25 \}$ \\ 
%         \hline
%             [2,3] & $\{ 1 \}, \{ 4 \}, \{ 2, 3, 5, 6, 7, 8, 9, 10, 14, 15, 16, 17 \}, \{ 18 \}, \{ 24 \}, \{ 25 \}$\\
%         \hline
%             [3,4] & $\{ 1 \}, \{ 2, 3, 4 \}, \{ 5, 6, 7, 8, 9, 12, 14, 15, 16 \}, \{ 18 \}, \{ 24 \}, \{ 23 \}$\\
%         \hline
%             [4,5] & $\{ 1 \}, \{ 2 \}, \{ 3, 4, 5, 6, 7, 8, 9, 11 \}, \{ 12, 14, 15 \}, \{ 24 \}, \{ 23 \}$\\
%         \hline
%             [5,6] & $\{ 1 \}, \{ 2, 3, 4, 5, 6, 11 \}, \{ 7 \}, \{ 12, 14, 15, 17 \}, \{ 22, 24 \}, \{ 23 \} \{ 25 \}$\\
%         \hline
%             [6,7] & $\{1\}, \{2, 3, 4\}, \{5, 6, 7, 10\}, \{15\}, \{12, 19\}, \{14\}, \{22, 23, 24\}$\\
%         \hline
%             [7,8] & $\{1\}, \{2, 3, 4, 5, 7, 8, 9, 10, 12, 14, 15, 16, 18, 22, 23, 24\}$\\
%         \hline
%             [8,9] & $\{1\}, \{2, 3, 4, 5, 7, 8, 9, 10, 11, 12, 13, 14, 15, 16, 17, 18, 20, 22, 23, 25\}$\\
%         \hline
%             [9,10] & $\{1\}, \{2, 3, 4\}, \{6\}, \{5, 8, 12, 13, 14, 17\}, \{22, 23, 25\}$\\
%         \hline
%             [10,11] & $\{1, 2, 3, 4\}, \{5, 8\}, \{12\}, \{14\}, \{17\}, \{25\}, \{23\}$\\
%         \hline
%             [11,12] & $\{1, 2, 3, 4\}, \{5, 9, 10\}, \{7\}, \{8, 12, 14, 17, 18, 21, 23, 25, 27\}$\\
%         \hline
%             [12,13] & $\{1, 2, 3, 4, 5, 6, 8, 10, 12, 13, 14, 15, 17, 18, 21, 23, 24, 27\}$\\
%         \hline
%             [13,14] & $\{3\}, \{1, 4, 6, 8, 11, 12, 13, 14, 15, 21\}, \{24\}, \{23\}, \{18\}$\\
%         \hline
%             [14,15] & $\{3\}, \{1, 4, 6, 7, 8, 9, 11, 12, 14, 16, 18, 21, 23, 24, 25\}$\\
%         \hline
%             [15,16] & $\{2, 3\}, \{5, 8\}, \{11, 12, 14, 16, 18, 21\}, \{24\}, \{25\}$\\
%         \hline
%             [16,17] & $\{2\}, \{3\}, \{8\}, \{5, 12, 14, 15, 18\}, \{24\}, \{25\}$\\
%         \hline
%             [17,18] & $\{1\}, \{2\}, \{5, 6, 8, 9, 12, 18\}, \{14\}, \{24\}, \{23, 25\}$\\
%         \hline        
%     \end{tabular}
% \end{center}
% \caption{ Množice posrednih in neposrednih tekmovalcev za 2 zaporedni dirki. }
% \label{tbl:tabelaPosrednihNeposrednihTekmovalcev2}
% \end{table}

\begin{table}
\begin{center}
    \begin{tabular}{ |c|c| } 
        \hline
            Dirke & Množice posrednih in neposrednih tekmovalcev \\ 
        \hline
            1,2,3 & $\{1\}, \{2, 4, 5, 7, 10, 14, 15, 16, 17\}, \{24\}, \{25\}$ \\ 
        \hline
            2,3,4 & $\{1\}, \{2, 3, 4, 5, 6, 7, 8, 9, 14, 15, 16\}, \{18\}, \{24\}$ \\ 
        \hline
            3,4,5 & $\{1\}, \{2, 3, 4, 5, 6, 7, 8, 9, 12, 14, 15\}, \{24\}, \{23\}$ \\ 
        \hline
            4,5,6 & $\{1\}, \{2, 3, 4, 5, 6, 7, 11\}, \{12, 14, 15\}, \{24\}, \{23\}$ \\ 
        \hline
            5,6,7 & $\{1\}, \{2, 3, 4, 5, 6, 7\}, \{12, 15\}, \{14\}, \{22, 23, 24\}$ \\ 
        \hline
            6,7,8 & $\{1\}, \{2, 3, 4, 5, 7, 10, 12, 14, 15, 22, 23, 24\}$ \\ 
        \hline
            7,8,9 & $\{1\}, \{2, 3, 4, 5, 7, 8, 9, 10, 12, 14, 15, 16, 18, 22, 23\}$ \\ 
        \hline
            8,9,10 & $\{1\}, \{2, 3, 4, 5, 8, 12, 13, 14, 17, 22, 23, 25\}$ \\ 
        \hline
            9,10,11 & $\{1, 2, 3, 4\}, \{5, 8, 12, 14, 17\}, \{23, 25\}$ \\ 
        \hline
            10,11,12 & $\{1, 2, 3, 4\}, \{5, 8, 12, 14, 17, 23, 25\}$ \\ 
        \hline
            11,12,13 & $\{1, 2, 3, 4, 5, 8, 10, 12, 14, 17, 18, 21, 23, 27\}$ \\ 
        \hline
            12,13,14 & $\{1, 3, 4, 6, 8, 12, 13, 14, 15, 18, 21, 23, 24\}$ \\ 
        \hline
            13,14,15 & $\{3\}, \{1, 4, 6, 8, 11, 12, 14, 18, 21, 23, 24\}$ \\ 
        \hline
            14,15,16 & $\{3\}, \{8\}, \{11, 12, 14, 16, 18, 21, 24, 25\}$ \\ 
        \hline
            15,16,17 & $\{2, 3\}, \{5, 8\}, \{12, 14, 18\}, \{24\}, \{25\}$ \\ 
        \hline
            16,17,18 & $\{2\}, \{5, 8, 12, 18\}, \{14\}, \{24\}, \{25\}$ \\ 
        \hline
    \end{tabular}
\end{center}
\caption{ Množice posrednih in neposrednih tekmovalcev za 3 zaporedne dirke. }
\label{tbl:tabelaPosrednihNeposrednihTekmovalcev3}
\end{table}

% \begin{table}
% \begin{center}
%     \begin{tabular}{ |c|c| } 
%         \hline
%             k=4 & Množice posrednih in neposrednih tekmovalcev \\ 
%         \hline
%             [1,4] & $\{1\}, \{2, 4, 5, 7, 14, 15, 16\}, \{24\}$ \\
%         \hline
%             [2,5] & $\{1\}, \{2, 3, 4, 5, 6, 7, 8, 9, 14, 15\}, \{24\}$ \\
%         \hline
%             [3,6] & $\{1\}, \{2, 3, 4, 5, 6, 7, 12, 14, 15\}, \{24\}, \{23\}$ \\
%         \hline
%             [4,7] & $\{1\}, \{2, 3, 4, 5, 6, 7\}, \{12, 14, 15\}, \{23, 24\}$ \\
%         \hline
%             [5,8] & $\{1\}, \{2, 3, 4, 5, 7, 12, 14, 15, 22, 23, 24\}$ \\
%         \hline
%             [6,9] & $\{1\}, \{2, 3, 4, 5, 7, 10, 12, 14, 15, 22, 23\}$ \\
%         \hline
%             [7,10] & $\{1\}, \{2, 3, 4, 5, 8, 12, 14, 22, 23\}$ \\
%         \hline
%             [9,12] & $\{1, 2, 3, 4\}, \{5, 8, 12, 14, 17, 23, 25\}$ \\
%         \hline
%             [10,13] & $\{1, 2, 3, 4, 5, 8, 12, 14, 17, 23\}$ \\
%         \hline
%             [11,14] & $\{1, 3, 4, 8, 12, 14, 18, 21, 23\}$ \\
%         \hline
%             [12,15] & $\{1, 3, 4, 6, 8, 12, 14, 18, 21, 23, 24\}$ \\
%         \hline
%             [13,16] & $\{3\}, \{8\}, \{11, 12, 14, 18, 21, 24\}$ \\
%         \hline
%             [14,17] & $\{3\}, \{8\}, \{12, 14, 18, 24, 25\}$ \\
%         \hline
%             [15,18] & $\{2\}, \{5, 8, 12, 14, 18\}, \{24\}, \{25\}$ \\
%         \hline        
%     \end{tabular}
% \end{center}
% \caption{ Množice posrednih in neposrednih tekmovalcev za 4 zaporedne dirke. }
% \label{tbl:tabelaPosrednihNeposrednihTekmovalcev4}
% \end{table}

% \begin{table}
% \begin{center}
%     \begin{tabular}{ |c|c| } 
%         \hline
%             k=5 & Množice posrednih in neposrednih tekmovalcev \\ 
%         \hline
%             [1,5] & $\{1\}, \{2, 4, 5, 7, 14, 15\}, \{24\}$ \\
%         \hline
%             [2,6] & $\{1\}, \{2, 3, 4, 5, 6, 7, 14, 15\}, \{24\}$ \\
%         \hline
%             [3,7] & $\{1\}, \{2, 3, 4, 5, 6, 7, 12, 14, 15\}, \{23, 24\}$ \\
%         \hline
%             [4,8] & $\{1\}, \{2, 3, 4, 5, 7, 12, 14, 15, 23, 24\}$ \\
%         \hline
%             [5,9] & $\{1\}, \{2, 3, 4, 5, 7, 12, 14, 15, 22, 23\}$ \\
%         \hline
%             [6,10] & $\{1\}, \{2, 3, 4, 5, 12, 14, 22, 23\}$ \\
%         \hline
%             [7,11] & $\{1, 2, 3, 4, 5, 8, 12, 14, 23\}$ \\
%         \hline
%             [8,12] & $\{1, 2, 3, 4, 5, 8, 12, 14, 17, 23, 25\}$ \\
%         \hline
%             [9,13] & $\{1, 2, 3, 4, 5, 8, 12, 14, 17, 23\}$ \\
%         \hline
%             [10,14] & $\{1, 3, 4, 8, 12, 14, 23\}$ \\
%         \hline
%             [11,15] & $\{1, 3, 4, 8, 12, 14, 18, 21, 23\}$ \\
%         \hline
%             [12,16] & $\{3\}, \{8, 12, 14, 18, 21, 24\}$ \\
%         \hline
%             [13,17] & $\{3\}, \{8\}, \{12, 14, 18, 24\}$ \\
%         \hline
%             [14,18] & $\{8, 12, 14, 18, 24, 25\}$ \\
%         \hline
%     \end{tabular}
% \end{center}
% \caption{ Množice posrednih in neposrednih tekmovalcev za 5 zaporednih dirk. }
% \label{tbl:tabelaPosrednihNeposrednihTekmovalcev5}
% \end{table}

% \begin{table}
% \begin{center}
%     \begin{tabular}{ |c|c| } 
%         \hline
%             k=6 & Množice posrednih in neposrednih tekmovalcev \\ 
%         \hline
%             [1,6] & $\{1\}, \{2, 4, 5, 7, 14, 15\}, \{24\}$ \\
%         \hline
%             [2,7] & $\{1\}, \{2, 3, 4, 5, 6, 7, 14, 15\}, \{24\}$ \\
%         \hline
%             [3,8] & $\{1\}, \{2, 3, 4, 5, 7, 12, 14, 15, 23, 24\}$ \\
%         \hline
%             [4,9] & $\{1\}, \{2, 3, 4, 5, 7, 12, 14, 15, 23\}$ \\
%         \hline
%             [5,10] & $\{1\}, \{2, 3, 4, 5, 12, 14, 22, 23\}$ \\
%         \hline
%             [6,11] & $\{1, 2, 3, 4, 5, 12, 14, 23\}$ \\
%         \hline
%             [7,12] & $\{1, 2, 3, 4, 5, 8, 12, 14, 23\}$ \\
%         \hline
%             [8,13] & $\{1, 2, 3, 4, 5, 8, 12, 14, 17, 23\}$ \\
%         \hline
%             [9,14] & $\{1, 3, 4, 8, 12, 14, 23\}$ \\
%         \hline
%             [10,15] & $\{1, 3, 4, 8, 12, 14, 23\}$ \\
%         \hline
%             [11,16] & $\{3\}, \{8, 12, 14, 18, 21\}$ \\
%         \hline
%             [12,17] & $\{3\}, \{8, 12, 14, 18, 24\}$ \\
%         \hline
%             [13,18] & $\{8, 12, 14, 18, 24\}$ \\
%         \hline
%     \end{tabular}
% \end{center}
% \caption{ Množice posrednih in neposrednih tekmovalcev za 6 zaporednih dirk. }
% \label{tbl:tabelaPosrednihNeposrednihTekmovalcev6}
% \end{table}

% \begin{table}
% \begin{center}
%     \begin{tabular}{ |c|c| } 
%         \hline
%             k=7 & Množice posrednih in neposrednih tekmovalcev \\ 
%         \hline
%             [1,7] & $\{1\}, \{2, 4, 5, 7, 14, 15\}, \{24\}$ \\ 
%         \hline
%             [2,8] & $\{1\}, \{2, 3, 4, 5, 7, 14, 15\}, \{24\}$ \\ 
%         \hline
%             [3,9] & $\{1\}, \{2, 3, 4, 5, 7, 12, 14, 15, 23\}$ \\ 
%         \hline
%             [4,10] & $\{1\}, \{2, 3, 4, 5, 12, 14, 23\}$ \\ 
%         \hline
%             [5,11] & $\{1, 2, 3, 4, 5, 12, 14, 23\}$ \\ 
%         \hline
%             [6,12] & $\{1, 2, 3, 4, 5, 12, 14, 23\}$ \\ 
%         \hline
%             [7,13] & $\{1, 2, 3, 4, 5, 8, 12, 14, 23\}$ \\ 
%         \hline
%             [8,14] & $\{1, 3, 4, 8, 12, 14, 23\}$ \\ 
%         \hline
%             [9,15] & $\{1, 3, 4, 8, 12, 14, 23\}$ \\ 
%         \hline
%             [10,16] & $\{3\}, \{8, 12, 14\}$ \\ 
%         \hline
%             [11,17] & $\{3\}, \{8, 12, 14, 18\}$ \\ 
%         \hline
%             [12,18] & $\{8, 12, 14, 18, 24\}$ \\ 
%         \hline
%     \end{tabular}
% \end{center}
% \caption{ Množice posrednih in neposrednih tekmovalcev za 7 zaporednih dirk. }
% \label{tbl:tabelaPosrednihNeposrednihTekmovalcev7}
% \end{table}

% \begin{table}
% \begin{center}
%     \begin{tabular}{ |c|c| } 
%         \hline
%             k=8 & Množice posrednih in neposrednih tekmovalcev \\ 
%         \hline
%             [1,8] & $\{1\}, \{2, 4, 5, 7, 14, 15\}, \{24\}$ \\
%         \hline
%             [2,9] & $\{1\}, \{2, 3, 4, 5, 7, 14, 15\}$ \\
%         \hline
%             [3,10] & $\{1\}, \{2, 3, 4, 5, 12, 14, 23\}$ \\
%         \hline
%             [4,11] & $\{1, 2, 3, 4, 5, 12, 14, 23\}$ \\
%         \hline
%             [5,12] & $\{1, 2, 3, 4, 5, 12, 14, 23\}$ \\
%         \hline
%             [6,13] & $\{1, 2, 3, 4, 5, 12, 14, 23\}$ \\
%         \hline
%             [7,14] & $\{1, 3, 4, 8, 12, 14, 23\}$ \\
%         \hline
%             [8,15] & $\{1, 3, 4, 8, 12, 14, 23\}$ \\
%         \hline
%             [9,16] & $\{3\}, \{8, 12, 14\}$ \\
%         \hline
%             [10,17] & $\{3\}, \{8, 12, 14\}$ \\
%         \hline
%             [11,18] & $\{8, 12, 14, 18\}$ \\
%         \hline
%     \end{tabular}
% \end{center}
% \caption{ Množice posrednih in neposrednih tekmovalcev za 8 zaporednih dirk. }
% \label{tbl:tabelaPosrednihNeposrednihTekmovalcev8}
% \end{table}

% \begin{table}
% \begin{center}
%     \begin{tabular}{ |c|c| } 
%         \hline
%             k=9 & Množice posrednih in neposrednih tekmovalcev \\ 
%         \hline
%             [1,9] & $\{1\}, \{2, 4, 5, 7, 14, 15\}$ \\
%         \hline
%             [2,10] & $\{1\}, \{2, 3, 4, 5\}, \{14\}$ \\
%         \hline
%             [3,11] & $\{1, 2, 3, 4, 5, 12, 14, 23\}$ \\
%         \hline
%             [4,12] & $\{1, 2, 3, 4, 5, 12, 14, 23\}$ \\
%         \hline
%             [5,13] & $\{1, 2, 3, 4, 5, 12, 14, 23\}$ \\
%         \hline
%             [6,14] & $\{1, 3, 4, 12, 14, 23\}$ \\
%         \hline
%             [7,15] & $\{1, 3, 4, 8, 12, 14, 23\}$ \\
%         \hline
%             [8,16] & $\{3, 8, 12, 14\}$ \\
%         \hline
%             [9,17] & $\{3\}, \{8, 12, 14\}$ \\
%         \hline
%             [10,18] & $\{8, 12, 14\}$ \\
%         \hline
%     \end{tabular}
% \end{center}
% \caption{ Množice posrednih in neposrednih tekmovalcev za 9 zaporednih dirk. }
% \label{tbl:tabelaPosrednihNeposrednihTekmovalcev9}
% \end{table}

% \begin{table}
% \begin{center}
%     \begin{tabular}{ |c|c| } 
%         \hline
%             k=10 & Množice posrednih in neposrednih tekmovalcev \\ 
%         \hline
%             [1,10] & $\{1\}, \{2, 4, 5\}, \{14\}$ \\
%         \hline
%             [2,11] & $\{1, 2, 3, 4, 5\}, \{14\}$ \\
%         \hline
%             [3,12] & $\{1, 2, 3, 4, 5, 12, 14, 23\}$ \\
%         \hline
%             [4,13] & $\{1, 2, 3, 4, 5, 12, 14, 23\}$ \\
%         \hline
%             [5,14] & $\{1, 3, 4, 12, 14, 23\}$ \\
%         \hline
%             [6,15] & $\{1, 3, 4, 12, 14, 23\}$ \\
%         \hline
%             [7,16] & $\{3, 8, 12, 14\}$ \\
%         \hline
%             [8,17] & $\{3, 8, 12, 14\}$ \\
%         \hline
%             [9,18] & $\{8, 12, 14\}$ \\
%         \hline
%     \end{tabular}
% \end{center}
% \caption{ Množice posrednih in neposrednih tekmovalcev za 10 zaporednih dirk. }
% \label{tbl:tabelaPosrednihNeposrednihTekmovalcev10}
% \end{table}

% \begin{table}
% \begin{center}
%     \begin{tabular}{ |c|c| } 
%         \hline
%             k=11 & Množice posrednih in neposrednih tekmovalcev \\ 
%         \hline
%             [1,11] & $\{1, 2, 4, 5\}, \{14\}$ \\
%         \hline
%             [2,12] & $\{1, 2, 3, 4, 5\}, \{14\}$ \\
%         \hline
%             [3,13] & $\{1, 2, 3, 4, 5, 12, 14, 23\}$ \\
%         \hline
%             [4,14] & $\{1, 3, 4, 12, 14, 23\}$ \\
%         \hline
%             [5,15] & $\{1, 3, 4, 12, 14, 23\}$ \\
%         \hline
%             [6,16] & $\{3, 12, 14\}$ \\
%         \hline
%             [7,17] & $\{3, 8, 12, 14\}$ \\
%         \hline
%             [8,18] & $\{8, 12, 14\}$ \\
%         \hline
%     \end{tabular}
% \end{center}
% \caption{ Množice posrednih in neposrednih tekmovalcev za 11 zaporednih dirk. }
% \label{tbl:tabelaPosrednihNeposrednihTekmovalcev11}
% \end{table}

% \begin{table}
% \begin{center}
%     \begin{tabular}{ |c|c| } 
%         \hline
%             k=12 & Množice posrednih in neposrednih tekmovalcev \\ 
%         \hline
%             [1,12] & $\{1, 2, 4, 5\}, \{14\}$ \\
%         \hline
%             [2,13] & $\{1, 2, 3, 4, 5, 14\}$ \\
%         \hline
%             [3,14] & $\{1, 3, 4, 12, 14, 23\}$ \\
%         \hline
%             [4,15] & $\{1, 3, 4, 12, 14, 23\}$ \\
%         \hline
%             [5,16] & $\{3, 12, 14\}$ \\
%         \hline
%             [6,17] & $\{3, 12, 14\}$ \\
%         \hline
%             [7,18] & $\{8, 12, 14\}$ \\
%         \hline
%     \end{tabular}
% \end{center}
% \caption{ Množice posrednih in neposrednih tekmovalcev za 12 zaporednih dirk. }
% \label{tbl:tabelaPosrednihNeposrednihTekmovalcev12}
% \end{table}

% \begin{table}
% \begin{center}
%     \begin{tabular}{ |c|c| } 
%         \hline
%             k=13 & Množice posrednih in neposrednih tekmovalcev \\ 
%         \hline
%             [1,13] & $\{1, 2, 4, 5, 14\}$ \\
%         \hline
%             [2,14] & $\{1, 3, 4, 14\}$ \\
%         \hline
%             [3,15] & $\{1, 3, 4, 12, 14, 23\}$ \\
%         \hline
%             [4,16] & $\{3, 12, 14\}$ \\
%         \hline
%             [5,17] & $\{3, 12, 14\}$ \\
%         \hline
%             [6,18] & $\{12, 14\}$ \\
%         \hline
%     \end{tabular}
% \end{center}
% \caption{ Množice posrednih in neposrednih tekmovalcev za 13 zaporednih dirk. }
% \label{tbl:tabelaPosrednihNeposrednihTekmovalcev13}
% \end{table}

% \begin{table}
% \begin{center}
%     \begin{tabular}{ |c|c| } 
%         \hline
%             k=14 & Množice posrednih in neposrednih tekmovalcev \\ 
%         \hline
%             [1,14] & $\{1, 4, 14\}$ \\
%         \hline
%             [2,15] & $\{1, 3, 4, 14\}$ \\
%         \hline
%             [3,16] & $\{3, 12, 14\}$ \\
%         \hline
%             [4,17] & $\{3, 12, 14\}$ \\
%         \hline
%             [5,18] & $\{12, 14\}$ \\
%         \hline
%     \end{tabular}
% \end{center}
% \caption{ Množice posrednih in neposrednih tekmovalcev za 14 zaporednih dirk. }
% \label{tbl:tabelaPosrednihNeposrednihTekmovalcev14}
% \end{table}

% \begin{table}
% \begin{center}
%     \begin{tabular}{ |c|c| } 
%         \hline
%             k=15 & Množice posrednih in neposrednih tekmovalcev \\ 
%         \hline
%             [1,15] & $\{1, 4, 14\}$ \\
%         \hline
%             [2,16] & $\{3\}, \{14\}$ \\
%         \hline
%             [3,17] & $\{3, 12, 14\}$ \\
%         \hline
%             [4,18] & $\{12, 14\}$ \\
%         \hline
%     \end{tabular}
% \end{center}
% \caption{ Množice posrednih in neposrednih tekmovalcev za 15 zaporednih dirk. }
% \label{tbl:tabelaPosrednihNeposrednihTekmovalcev15}
% \end{table}

% \begin{table}
% \begin{center}
%     \begin{tabular}{ |c|c| } 
%         \hline
%             k=16 & Množice posrednih in neposrednih tekmovalcev \\ 
%         \hline
%             [1,16] & $\{14\}$ \\
%         \hline
%             [2,17] & $\{3\}, \{14\}$ \\
%         \hline
%             [3,18] & $\{12, 14\}$ \\
%         \hline
%     \end{tabular}
% \end{center}
% \caption{ Množice posrednih in neposrednih tekmovalcev za 16 zaporednih dirk. }
% \label{tbl:tabelaPosrednihNeposrednihTekmovalcev16}
% \end{table}

% \begin{table}
% \begin{center}
%     \begin{tabular}{ |c|c| } 
%         \hline
%             k=17 & Množice posrednih in neposrednih tekmovalcev \\ 
%         \hline
%             [1,17] & $\{14\}$ \\
%         \hline
%             [2,18] & $\{14\}$ \\
%         \hline
%     \end{tabular}
% \end{center}
% \caption{ Množice posrednih in neposrednih tekmovalcev za 17 zaporednih dirk. }
% \label{tbl:tabelaPosrednihNeposrednihTekmovalcev17}
% \end{table}

% \begin{table}
% \begin{center}
%     \begin{tabular}{ |c|c| } 
%         \hline
%             k=18 & Množice posrednih in neposrednih tekmovalcev \\ 
%         \hline
%             [1,18] & $\{14\}$ \\
%         \hline
%         \end{tabular}
%     \end{center}
%     \caption{ Množice posrednih in neposrednih tekmovalcev za 18 zaporednih dirk. }
%     \label{tbl:tabelaPosrednihNeposrednihTekmovalcev18}
% \end{table}

%%%%%%%%%%%%%%%%%%%%%%%%%%%%%%%%%%%%%%%%%
%%%%%%%%%%%%%%%%%%%%%%%%%%%%%%%%%%%%%%%%%
%%%%%%%%%%%%%%%%%%%%%%%%%%%%%%%%%%%%%%%%%
%%%%%%%%%%%%%%%%%%%%%%%%%%%%%%%%%%%%%%%%%
%%%%%%%%%%%%%%%%%%%%%%%%%%%%%%%%%%%%%%%%%
%%%%%%%%%%%%%%%%%%%%%%%%%%%%%%%%%%%%%%%%%
%%%%%%%%%%%%%%%%%%%%%%%%%%%%%%%%%%%%%%%%%
%%%%%%%%%%%%%%%%%%%%%%%%%%%%%%%%%%%%%%%%%
%%%%%%%%%%%%%%%%%%%%%%%%%%%%%%%%%%%%%%%%%
%%%%%%%%%%%%%%%%%%%%%%%%%%%%%%%%%%%%%%%%%
%%%%%%%%%%%%%%%%%%%%%%%%%%%%%%%%%%%%%%%%%

% ra\-ču\-nal\-ni\-štvo

\chapter{Sklep}

V diplomskem delu smo si pogledali, kaj so inverzije permutacij, njihove lastnosti in kako definirajo permutacijske in tekmovalnostne grafe.

Diplomsko delo smo začeli s ponovitvijo nekaterih osnovnih pojmov, oznak in definicij s področja teorije grafov, algebre in teorije množic, ki so ključni za nadaljno razumevanje.

Nato smo definirali permutacije in pokazali nekaj načinov, kako jih lahko zapišemo. Povedali smo, da je množica vseh permutacij z $n$ elementi skupaj z operacijo kompozitum simetrična grupa. Permutacijska grupa je vsaka podgrupa simetrične grupe ter po Cayleyevem izreku je vsaka grupa izomorfna neki permutacijski grupi.

V nadaljevanju smo povedali kaj so inverzije neke permutacije. Pokazali smo kako lahko permutacije delno uredimo s pomočjo Bruhatovih delnih urejenosti. Šibka Bruhatova delna urejenost nam pove, da lahko vsako permutacijo z $i$ inverzijami uredimo z natanko $i$ transpozicijami sosednih elementov. Zatem izračunamo rodovno funkcijo, ki nam šteje število permutacij množice $[n]$ z $i$ inverzijami. Pokazali smo, kako lahko množico permutacij $S_n$ uredimo in tako vsaki permutaciji iz množice $S_n$ dodelimo celo število $N$, kjer je $0 \leq N \leq n!$. To lahko denimo naredimo s pomočjo Lehmerjeve kode ali vektorja inverzij.

V tretjem poglavju smo se posvetili permutacijskim grafom. Najprej smo jih karakterizirali s pomočjo kohezivnega zaporedja grafa. Potem smo se osredotočili na drevesa, ki so permutacijski grafi. Ugotovili smo, da so to ravno gosenice. Pokazali smo, da obstajata natanko dve permutaciji iz $S_n$, ki imata permutacijski graf izomorfen neki gosenici na $n \geq 3$ vozliščih. Nato smo predstavili, kako lahko konstruiramo permutacijske grafe s pomočjo kompozicije grafov.

V zadnjem delu smo si pogledali kaj so tekmovalnostni grafi. Povedali smo, kaj so rangiranja, kdaj par vozlišč tekmuje, kako je tekmovalnost dveh vozlišč povezana z inverzijami permutacij, kaj je tekmovalnostna množica, kaj je množica tekmovalcev ter kaj je množica posrednih in neposrednih tekmovalcev. Nato smo povedali, kaj so primerljivostni grafi in ugotovili, da so permutacijski grafi tako primerljivostni kot tekmovalnostni grafi. Definirali smo tudi delno kohezivno zaporedje vozlišč ter pokazali, da je vsak tekmovalnostni graf delno koheziven. Zatem smo predstavili algoritem za izračun množice posrednih in neposrednih tekmovalcev direktno iz množice rangiranj brez predhodnega izračuna tekmovalnostnega grafa. Pokazali smo tudi, kako so množice posrednih in neposrednih tekmovalcev urejene. Na koncu smo algoritem uporabili na resničnih podatkih. Ker so bili podatki iz špotra, kjer so velike spremembe rangiranj pogoste, smo ugotovili, da če želimo pridobiti uporabne informacije, moramo algoritem uporabiti na manjšem številu rangiranj. Prvi prikazani način je izbira nekaterih rangiranj. Drugi prikazani način pa je izbira nekaj zaporednih rangiranj. Pri drugem načinu tako lahko vidimo formo športnikov skozi čas. Treba pa je poudariti, da sta oba načina pomanjkljiva. Pri prvem je težava, da dobimo informacije iz neke podmnožice rangiranj. Te informacije so lahko zelo variabilne. Pri drugem načinu je težava v tem, da ne dobimo veliko informacij o športnikih, ki so v tem zaporedju imeli en slab rezultat. Rezultate algoritma bi morda lahko izboljšali s pomočjo predprocesiranja podatkov. Na primer, v primeru odstopa športnika, ga lahko kaznujemo z uvrstitvijo na njegovo rahlo podpovprečno mesto, namesto da ga diskvalifikaciramo.

\cleardoublepage
\addcontentsline{toc}{chapter}{Literatura}
\bibliography{literatura}
\bibliographystyle{plainnat}

\end{document}

