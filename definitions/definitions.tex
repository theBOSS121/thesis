\documentclass{article}

\usepackage{tikz}
\usepackage{verbatim}
\usepackage{parskip}
\usepackage{amsthm}
\usepackage{xpatch}

\setlength\parindent{0pt}

\newtheorem*{definition}{Definition}
\newtheorem*{theorem}{Theorem}
\newtheorem*{lemma}{Lemma}
\newtheorem*{corollary}{Corollary}

\makeatletter
\xpatchcmd{\@thm}{\thm@headpunct{.}}{\thm@headpunct{}}{}{}
\makeatother

\begin{document}
\pagestyle{empty}

\begin{comment}
definitions
\end{comment}

%%%%%%%%%%%%%%%%%%%%%%%%%%%%%%%%%%%%%%%%%%%%%%%%%%%%%%%%%%%%%%%%%
\section{ Article: Characterization and Constriction of Permutation Graphs }
An inversion of $\sigma$ is an ordered pair $(a_i, a_j)$ where $i < j$ but $a_i > a_j$. Equivalently $(x, y)$ is an inversion if and only if $x > y$ and $\sigma^{-1}(x) < \sigma^{-1}(y)$ 

\begin{definition}[1.1]
    Let $\sigma \in S_n$. The graph of inverions of $\sigma$, denoted by $G_{\sigma}$, is the graph with vertices $1,2,...,n$ where $xy$ is an edge of $G_{\sigma} \Leftrightarrow (x, y)$ or $(y, x)$ is an inversion of $\sigma$. 
\end{definition}

\begin{definition}[2.1]
    Let G be a graph of order n. An arrangement $l = (v_1, v_2, ..., v_n)$ of the vertices of G is called a cohesive vertex-set order of G (or simply cohesive order G) if the following conditions are satisfied:\\
    (a) if $i < k < j$ and $v_iv_k, v_kv_j \in E(G)$, then $v_iv_j \in E(G)$ \\
    (b) if $i < k < j$ and $v_iv_j \in E(G)$, then $v_iv_k \in E(G)$ or $v_kv_j \in E(G)$ \\
\end{definition}

\begin{lemma}[2.1]
    Let G be a graph. Then $l$ is a cohesive order of G $\Leftrightarrow l$ is a cohesive order of $\overline{G}$.
\end{lemma}

\begin{theorem}[2.1]
    Let $\sigma \in S_n$. Then $\sigma = (\sigma(1),\sigma(2),...,\sigma(n))$ is a cohesive order of the permutation graph $G_{\sigma}$
\end{theorem}

Note that $(v_1, v_2, ..., v_n)$ is a cohesive order of a graph G $\Leftrightarrow (v_n, v_{n-1}, ..., v_1)$ is a cohesive order of G.

\begin{definition}[2.2]
    An orientation of a graph G is the digraph (directed graph) obtained by assigning a direction to each edge of G. The directed edges, which are ordered pairs, are called arcs.
\end{definition}

A digraph D is said to be transitive if $(x, z)$ is an arc of D whenever $(x, y)$ and $(y, z)$ are arcs in D.

An oriented complete graph is called a tournament.

The score of a vertex x in a tournament, denoted be $s(x)$, is defined by $s(x) = deg^{+}(x)$ (out-degree of x)

The score sequence of a tournament is the sequence of scores arranged in non-decreasing order.

\begin{theorem}[2.2]
    Let T be a tournament of order n. The following statements are equivalent.\\
    1) T is transitive. \\
    2) For all vertices x and y in T, if $(x, y)$ is an arc of T, then $s(x) > s(y)$ \\
    3) For all vertices x and y in T, if $s(x) > s(y)$, then $(x, y)$ is an arc of T. \\
    4) The score sequence of T is $(0, 1, 2,..., n-1)$

\end{theorem}

\begin{theorem}[2.3]
    A graph G is a permutation graph $\Leftrightarrow$ G has a cohesive order.
\end{theorem}

\begin{theorem}[3.1]
    Let G be a graph. The following are equivalent:\\
    (a) $G$ is a permutation graph. \\
    (b) $\overline G$ is a permutation graph. \\
    (c) Every induced subgraph of G is a permutation graph. \\
    (d) Every conneted component of G is a permutation graph. \\
\end{theorem}

A caterpillar is a tree with the property that the removal of all pendant vertices results into a path.

\begin{lemma}[3.1]
    A tree is a caterpillar $\Leftrightarrow$ it does not contain $K^*_{1, 3}$ as a subgraph.
\end{lemma}

\begin{theorem}[3.2]
    A tree is a permutation graph $\Leftrightarrow$ it is a caterpillar.
\end{theorem}

\begin{definition}[3.1]
    Let G be a graph with vertices $x_1, x_2, ..., x_n$ and let $H_1, H_2, ,..., H_n$ be a collection of arbitrary graphs. The composition by G of $H_1, H_2, ..., H_n$, denoted by $G(H_1, H_2, ..., H_n)$ is the graph formed by taking the disjoint union of the graphs $H_i$ and then adding all edges of the form $a_ib_j$ where $a_i$ is in $H_i$, $b_j$ is in $H_j$ whenever $x_ix_j$ is an edge of G.
\end{definition}

The sum of two graphs L and M, denoted by $L + M$ is formed by taking the disjoint union of L and M and then adding all edges of the form $ab$ where $a \in V(L)$ and $b \in V(M)$. Thus, the composition $G(H_1, H_2, ..., H_n)$ is formed by taking the disjoint union of the graphs $H_i$ and then forming the sum $H_i + H_j$ if the associated vertices $x_i$ and $x_j$ of G are adjacent.

\begin{theorem}[3.3]
    Let G be a graph of order n and let $H_1, H_2, ..., H_n$ be arbitrary graphs. Then $G(H_1, H_2, ..., H_n)$ is a permutation graph $\Leftrightarrow$ $G, H_1, H_2, ..., H_n$ are permutation graphs
\end{theorem}

If the only ways G can be written as a composition are $G = G(P_1,...,P_1)$, $G = K_1(G)$, then we say that G is prime.

\begin{theorem}[3.4]
    A tree is a prime permutation graph $\Leftrightarrow$ A tree is a caterpillar where no two pendant vertices have a common neighbor.
\end{theorem}

\begin{theorem}[3.5]
    Let G be a decomposable permutation graph. Then there exists a non-trivial prime permutation graph U and pemutation graphs $H_1, H_2, ..., H_k$ which are subgraphs of G such that $G = U(H_1, H_2, ..., H_k)$
\end{theorem}
%%%%%%%%%%%%%%%%%%%%%%%%%%%%%%%%%%%%%%%%%%%%%%%%%%%%%%%%%%%%%%%%%
\section { Article: Permutation graphs and the weak Bruhat order }

Let $I(\sigma) = \{ (i_k,i_l): i_k > i_l, 1 \leq k < l \leq n \}$ be the inversion set of $\sigma$.

Each inversion $(i_k, i_l)$ of $\sigma$ has an associated unordered pair $\{ i_k, i_l \}$ called unordered inversion of $\sigma$, the set of which is denoted by $E(\sigma)$

The permutation $\sigma$ thus determines: \\
(a) a labeled graph $G(\sigma)$ with vertex set $\{ 1, 2, ..., n\}$ and edge set $E(\sigma)$ with the edges of $G(\sigma)$ corresponding to the unordered inversions of the permutation $\sigma$; \\
(b) a labeled digraph $\Gamma(\sigma)$ with vertex set $\{ 1, 2, ..., n \}$ and edge set $I(\sigma)$ whose (directed) edges thus correspond to the inversions of $\sigma$.

Thus $\Gamma(\sigma)$ is an orientation of $G(\sigma)$

The set $I(\sigma)$ of inversions also determines a partially ordered set $(P(\sigma), \preceq_{\sigma})$ on $\{1, 2, ..., n\}$, since $1 \leq k < l < t \leq n$ and $(i_k, i_l)$, $(i_l, i_t) \in I(\sigma) \Rightarrow (i_k, i_t) \in I(\sigma)$  

Note that the graph complement of $G(\sigma)$ is $G(\overleftarrow{\sigma})$ where $\overleftarrow{\sigma}$ is the permutation obtained from $\sigma$ by reversing the sequence of its elements to get $\overleftarrow{\sigma} = (i_n, ..., i_2, i_1)$

\begin{theorem}[1.1]
    A graph G is a permutation graph $\Leftrightarrow$ G and $\overline{G}$ are comparability graphs.
\end{theorem}

A poset (partially ordered set) $P = (X, \leq)$ determines a graph $G(P)$ with vertex set $X$ whose edges are all those pairs $\{ a, b \}$ such that $a \neq b$ and either $a \leq b$ or $b \leq a$. The transitive property of a partial order implies that $G(P)$ is a comparability graph. A linear extension of a poset $P = (X, \leq)$ is a linearly ordered set $P = (X, \leq^{'})$ on the same set X of elements such that $x \leq y$ always implies $x \leq^{'} y$. The dimension of a poset $(P, \leq)$ is the smallest number of its linear extensions whose intersection is $(P, \leq)$. A poset of dimension 1 is a linearly ordered set.

\begin{theorem}[1.2]
    A poset $P = (X, \leq)$ has dimension at most 2 $\Leftrightarrow$ the complement of its comparability graph is also a comparability graph
\end{theorem}

\begin{corollary}[1.3]
    Let $\sigma$ be a permutation of $\{ 1, 2, ..., n\}$ not equal to the identity permutation $(1, 2, ..., n)$ or its reversal $(n, n-1, ..., 1)$. Then the dimension of $(P(\sigma), \preceq)$ equals 2.
\end{corollary}

\begin{lemma}[2.1]
    Assume that $1 \leq k < l \leq n$. Let $\pi = (i_1, ..., i_{k-1}, i_k, ..., i_l, i_{l+1}, ..., i_n)$ where $i_k > i_l$, and let $\sigma = (i_1, ..., i_{k-1}, i_l, ..., i_k, i_{l+1}, ..., i_n)$ be obtained from $\pi$ by the transposition that interchanges $i_k$ and $i_l$. Consider the partition of $L = \{ k, k+1, ..., l \}$ given by $L = L_1 \cup L_2 \cup L_3 \cup  \{ k, l \}$ where \\
    $L_1 = \{ s \in L: i_s > i_k \}$, $L_2 = \{ s \in L: i_k > i_s > i_l \}$, $L_3 = \{ s \in L: i_l > i_s \}$ \\
    Consider the graphs $G(\sigma)$ and $G(\pi)$ with edge sets $E(\sigma)$ and $E(\pi)$. Then \\
    $E(\sigma) \setminus E(\pi) = \{ (i_k, i_l) \} \cup \{ (i_k, i_s): s \in L_2 \cup L_3 \}$ \\
    $E(\pi) \setminus E(\sigma) = \{ (i_l, i_s): s \in L_3 \} \cup \{ (i_s, i_k): s \in L_1 \}$ \\
    In particular, $\sigma \preceq_b \pi \Leftrightarrow i_k > i_s > i_l$  $(k < s < l)$, and thus $\sigma \preceq_b \pi \Leftrightarrow \sigma$ can be obtained from $\pi$ by a sequence of adjacent transpositions.

\end{lemma}
















%%%%%%%%%%%%%%%%%%%%%%%%%%%%%%%%%%%%%%%%%%%%%%%%%%%%%%%%%%%%%%%%%
\section{ Graph theory definitions }
Induced subgraph of a graph is another graph, formed from a subset of the vertices of the graph and all of the edges (from the original graph) connecting pairs of vertices in that subset.

An isomorphism $f$ of graphs $G$ and $H$ is bijection between the vertex sets of $G$ and $H$ $f: V(G) \rightarrow V(H)$ such that any two vertices u and v of $G$ are adjacent in $G \Leftrightarrow f(u)$ and $f(v)$ are adjacent in $H$. (edge-preserving bijection) $G \simeq H$

Complement or inverse of a graph G is a graph H on the same vertices such that two distinct vertices of H are adjacent if and only if they are not adjacent in G.

An orientation of an undirected graph is an assignment of a direction to each edge, turning the initial graph into a directed graph. A directed graph is called an oriented graph if none of its pairs of vertices is linked by two symmetric edges. Among directed graphs, the oriented graphs are the ones that have no 2-cycles (that is at most one of (x, y) and (y, x) may be arrows of the graph).

A tournament is an orientation of a complete graph.

A directed graph is said to be strongly connected if every vertex is reachable from every other vertex. The strongly connected components of an arbitrary directed graph form a partition into subgraphs that are themselves strongly connected.

A strong orientation is an orientation that results in a strongly connected graph.

An acyclic graph is a graph having no graph cycles. Acyclic graphs are bipartite. A connected acyclic graph is known as a tree, and a possibly disconnected acyclic graph is known as a forest.

A directed acyclic graph (DAG) is a directed graph with no directed cycles.

A transitive orientation is an orientation such that the resulting directed graph is its own transitive closure. The graphs with transitive orientations are called comparability graphs; they may be defined from a partially ordered set by making two elements adjacent whenever they are comparable in the partial order.

Comparability graph is an undirected graph that connects pairs of elements that are comparable to each other in a partial order. Comparability graphs have also been called transitively orientable graphs.
Equivalently, a comparability graph is a graph that has a transitive orientation, an assignment of directions to the edges of the graph (i.e. an orientation of the graph) such that the adjacency relation of the resulting directed graph is transitive: whenever there exist directed edges (x,y) and (y,z), there must exist an edge (x,z).

Every complete graph is a comparability graph. All acyclic orientations of a complete graph are transitive. Every bipartite graph is also a comparability graph. Orienting the edges of a bipartite graph from one side of the bipartition to the other results in a transitive orientation.

An undirected graph H is called a minor of the graph G if H can be formed from G by deleting edges, vertices and by contracting edges. An edge contraction is an operation that removes an edge from a graph while simultaneously merging the two vertices that it previously joined.

Disjoint union of graphs is an operation that combines two or more graphs to form a larger graph. It is analogous to the disjoint union of sets, and is constructed by making the vertex set of the result be the disjoint union of the vertex sets of the given graphs, and by making the edge set of the result be the disjoint union of the edge sets of the given graphs. Any disjoint union of two or more nonempty graphs is necessarily disconnected.

The diameter of a tree is the length of the longest path between any 2 nodes of a tree. 

%%%%%%%%%%%%%%%%%%%%%%%%%%%%%%%%%%%%%%%%%%%%%%%%%%%%%%%%%%%%%%%%%
\section{ Relacije in urejenosti: }
Relacije: $ R\subseteq A\times A, A\neq\emptyset$ \\
R je refleksivna $\Leftrightarrow \forall x\in A: xRx$ \\
R je irefleksivna $\Leftrightarrow \forall x\in A: \neg xRx \Leftrightarrow \overline{R}$ relfeksina \\
R je simetricna $\Leftrightarrow \forall x,y\in A: xRy \Rightarrow yRx \Leftrightarrow R = R^{-1}$ \\
R je asimetricna $\Leftrightarrow \forall x,y\in A: xRy \Rightarrow \neg yRx \Leftrightarrow R \cap R^{-1} = \emptyset$ \\ Ce asimetricna $\Rightarrow$ irefleksina \\
R je antisimetricna $\Leftrightarrow \forall x,y\in A: xRy \Rightarrow x = y \lor \neg yRx \Leftrightarrow R \cap R^{-1} \subseteq I = \{ (x,x): x \in A\} $ \\
R je tranzitivna $\Leftrightarrow \forall x,y,z\in A: xRy \land yRz \Rightarrow xRz$ \\
R je sovisna $\Leftrightarrow \forall x,y\in A: x \neq y \Rightarrow xRy \lor yRx $ (x in y sta primerljiva)\\
R je strogosovisna $\Leftrightarrow \forall x,y\in A: xRy \lor yRx $\\

Relacije urejenosti: \\
Delna urejenost: R je refleksivna, antisimetricna in tranzitivna [$\subseteq$]\\
Linearna urejenost: R je antisimetricna, strogosovisna, transitivna (refleksivna) [$\leq$]\\ 
Stroga delna urejenost: R je asimetricna in tranzitivna (irefleksivna) [$\subset$] \\
Stroga linearna urejenost: R je asimetricna, sovisna in transitivna [$<$] \\
linearna urejenost $\Rightarrow$ delna urejenost \\
stroga linearna urejenost $\Rightarrow$ stroga delna urejenost \\

Potence relacije R: $xR^{n}y \Leftrightarrow x(R \circ R \circ ... \circ R)y$

Transitivna ovojnica relacije R: $R^{+} = \bigcup_{n=1}^{\infty} R^{n}$

Transitive closure $R^{+}$ of a homogeneous binary relation R on a set X is the smallest relation on X that contains R and is transitive.\\
For example, if X is a set of airports and xRy means "there is a direct flight from airport x to airport y" (for x and y in X), then the transitive closure of R on X is the relation $R^+$ such that x$R^+$y means "it is possible to fly from x to y in one or more flights".


\section { Funkcije }
Funkcija $f: X \rightarrow Y$ je: \\
injektivna : $x_1 \neq x_2 \Rightarrow f(x_1) \neq f(x_2) \Leftrightarrow f(x_1) = f(x_2) \Rightarrow x_1 = x_2$ \\
surjektivna : $\forall y \in Y$ $\exists x \in X: f(x) = y$ \\
bijektivna: injektivna in surjektivna \\
$\exists$ injekcija $f: X \rightarrow Y \Rightarrow |X| \leq |Y|$ \\
$\exists$ surjekcija $f: X \rightarrow Y \Rightarrow |X| \geq |Y|$ \\
$\exists$ bijekcija $f: X \rightarrow Y \Rightarrow |X| = |Y|$ \\

\section { Other }
Partition is division of a set of objects into a family of subsets that are mutually exclusive and jointly exhaustive.

\end{document}